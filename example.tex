% General example LaTeX file for including Sage calculations and plots
% Build with: (pdf)latex example.tex; sage example.sage; pdflatex example.tex 
% Please read README.txt and the documentation of the SageTeX package
% for more information!

\documentclass{article}
\title{Examples of embedding Sage in \LaTeX}
\author{\relax}

\usepackage{sagetex}
% 
% If you want sagetex to use Imagemagick's convert utility to make eps
% files from png files when generating a dvi file, add the
% "imagemagick" option above:
%
%    \usepackage[imagemagick]{sagetex}

\begin{document}
\maketitle

\section{Inline Sage, code blocks}

This is an example $2+2=\sage{2+2}$. If you raise the current year
($\the\year$) to the power of the current day ($\the\day$), you get
$\sage{\the\year^\the\day}$.

Code block which uses a variable \texttt{s} to store the solutions:
\begin{sageblock}
 var('a,b,c')
 eqn = [a+b*c==1, b-a*c==0, a+b==5]
 s = solve(eqn, a,b,c)
\end{sageblock}

Solutions of $\mbox{eqn}=\sage{eqn}$:
$$
\sage{s[0]}
$$
$$
\sage{s[1]}
$$

Now we evaluate the following block:
\begin{sageblock}
E = EllipticCurve("37a")
\end{sageblock}
You can't do assignment inside \verb|\sage| macros, since Sage doesn't
know how to typeset the output of such a thing. So you have to use a
code block. The elliptic curve $E$ given by $\sage{E}$ 
has discriminant $\sage{E.discriminant()}$. 

You can do anything in a code block that you can do in Sage and/or
Python. Here we save an elliptic curve into a file.
\begin{sageblock}
try:
    E = load('E2')
except IOError:
    E = EllipticCurve([1,2,3,4,5])
    E.anlist(100000)
    E.save('E2')
\end{sageblock}
\label{ellipticcurve}

The 9999th Fourier coefficient of $\sage{E}$ is 
$\sage{E.anlist(100000)[9999]}$. 

The following code block doesn't appear in the typeset file\dots
\begin{sagesilent}
  e = 2
  e = 3*e + 1
\end{sagesilent}
but we can refer to whatever we did in that code block: $e=\sage{e}$.

\begin{sageblock}
  var('x')
  f = log(sin(x)/x)
\end{sageblock}
The Taylor Series of $f$ is: $\sage{ f.taylor(x, 0, 10) }$.

\section{Plotting}

Here's a plot of the elliptic curve $E$.

\sageplot{E.plot(-3,3)}

\begin{sagesilent}
  var('x')
  f=-x^3+3*x^2+7*x-4
\end{sagesilent}

You can use variables to hold plot objects and do stuff with them.
\begin{sageblock}
  p = plot(f, x, -5, 5)
\end{sageblock}

Here's a small plot of $f$ from $-5$ to $5$, which I've centered:

\centerline{\sageplot[scale=.2]{p}}

On second thought, use the default size of $3/4$ the \verb|\textwidth|
and don't use axes:

\sageplot{p, axes=False}

Remember, you're using Sage, and can therefore call upon any of the
software packages Sage is built out of.
\begin{sageblock}
f = maxima('sin(x)^2*exp(x)')
g = f.integrate('x')
\end{sageblock}
Plot $g(x)$, but don't typeset it.
\begin{sagesilent}
  # g is a Maxima thingy, it needs to get converted into a Sage object
  plot1 = plot(g.sage(),x,-1,2*pi)
\end{sagesilent}

You can specify a file format and options for \verb|includegraphics|.
The default is for EPS and PDF files, which are the best choice in
almost all situations. (Although see the section on 3D plotting.)

\sageplot[angle=45, width=.5\textwidth][png]{plot1}

When using \verb|\sageplot|, you can pass in just about anything that
Sage can call \verb|.save()| on to produce a graphics file:

\begin{center}
\sageplot{plot1 + plot(f.sage(),x,-1,2*pi,rgbcolor=hue(0.4)), figsize=[1,2]}
\end{center}

\sageplot{graphs.FlowerSnark().plot()}

\begin{sageblock}
G4 = DiGraph({1:[2,2,3,5], 2:[3,4], 3:[4], 4:[5,7], 5:[6]},\
             multiedges=True)
G4plot = G4.plot(layout='circular')
\end{sageblock}

\sageplot{G4plot, axes=False}

Indentation and so on works fine.
\begin{sageblock}
 s     = 7
 s2    = 2^s
 P.<x> = GF(2)[]
 M     = matrix(parent(x),s2)
 for i in range(s2):
    p  = (1+x)^i
    pc = p.coeffs()
    a  = pc.count(1)
    for j in range(a):
        idx        = pc.index(1)
        M[i,idx+j] = pc.pop(idx)

 matrixprogram = matrix_plot(M,cmap='Greys')
\end{sageblock}
And here's the picture:

\sageplot{matrixprogram}

\subsection{3D plotting}

3D plotting right now is problematic because there's no convenient way
to produce vector graphics. We can make PNGs, though, and since the
\verb|sageplot| command defaults to EPS and PDF, \emph{you must specify
a valid format for 3D plotting}. Sage right now can't produce EPS or PDF
files from plot3d objects, so if you don't specify a valid format,
things will go badly. You can specify the ``\texttt{imagemagick}''
option, which will use the Imagemagick \texttt{convert} utility to make
EPS files. See the documentation for details.

Here's the famous Sage cube graph:

\begin{sageblock}
  G = graphs.CubeGraph(5)
\end{sageblock}

% need empty [] so sageplot knows you want png format, and aren't
% passing an option to includegraphics
\sageplot[][png]{G.plot3d()}

\end{document}
