% General example LaTeX file for including Sage calculations and plots
% Build with:
%
%   (pdf)latex example.tex; sage example.sage; pdflatex example.tex
%
% Please read README and the documentation of the SageTeX package for
% more information!

\documentclass{article}
\title{Examples of embedding Sage in \LaTeX{} with \textsf{Sage\TeX}}
\author{Dan Drake and others}
\usepackage{hyperref}
\usepackage{tkz-berge}

\usepackage{sagetex}
%
% If you want SageTeX to use Imagemagick's `convert' utility to make eps
% files from png files when generating a dvi file, add the "imagemagick"
% option above:
%
%    \usepackage[imagemagick]{sagetex}

\setlength{\sagetexindent}{10ex}

\begin{document}
\maketitle

\section{Inline Sage, code blocks}

This is an example $2+2=\sage{2+2}$. If you raise the current year mod
$100$ (which equals $\sage{mod(\the\year, 100)}$) to the power of the
current day ($\the\day$), you get $\sage{Integer(mod(\the\year,
100))^\the\day}$. Also, $\the\year$ modulo $42$ is $\sage{\the\year
\percent 42}$.

Code block which uses a variable \texttt{s} to store the solutions:
\begin{sageblock}
 1+1
 var('a,b,c')
 eqn = [a+b*c==1, b-a*c==0, a+b==5]
 s = solve(eqn, a,b,c)
\end{sageblock}

Solutions of $\mbox{eqn}=\sage{eqn}$:
\[
\sage{s[0]}
\]
\[
\sage{s[1]}
\]

Now we evaluate the following block:
\begin{sageblock}
E = EllipticCurve("37a")
\end{sageblock}
You can't do assignment inside \verb|\sage| macros, since Sage doesn't
know how to typeset the output of such a thing. So you have to use a
code block. The elliptic curve $E$ given by $\sage{E}$ has discriminant
$\sage{E.discriminant()}$.

You can do anything in a code block that you can do in Sage and/or
Python. Here we save an elliptic curve into a file.
\begin{sageblock}
try:
    E = load('E2')
except IOError:
    E = EllipticCurve([1,2,3,4,5])
    E.anlist(100000)
    E.save('E2')
\end{sageblock}

The 9999th Fourier coefficient of $\sage{E}$ is
$\sage{E.anlist(100000)[9999]}$.

The following code block doesn't appear in the typeset file\dots
\begin{sagesilent}
  e = 2
  e = 3*e + 1
\end{sagesilent}
but we can refer to whatever we did in that code block: $e=\sage{e}$.

\begin{sageblock}
  var('x')
  f(x) = log(sin(x)/x)
\end{sageblock}
The Taylor Series of $f$ begins: $\sage{ f.taylor(x, 0, 10) }$.

\section{Plotting}

Here's a plot of the elliptic curve $E$.

\sageplot{E.plot(-3,3)}

\begin{sagesilent}
  # the var line is unecessary unless you've defined x to be something
  # other than a symbolic variable
  var('x')
  f(x) = -x^3+3*x^2+7*x-4
\end{sagesilent}

You can use variables to hold plot objects and do stuff with them.
\begin{sageblock}
  p = plot(f, x, -5, 5)
\end{sageblock}

Here's a small plot of $f$ from $-5$ to $5$, which I've centered:

\begin{center} \sageplot[scale=.2]{p} \end{center}

On second thought, use the default size of $3/4$ the \verb|\textwidth|
and don't use axes:

\sageplot{p, axes=False}

Remember, you're using Sage, and can therefore call upon any of the
software packages Sage is built out of.
\begin{sageblock}
f = maxima('sin(x)^2*exp(x)')
g = f.integrate('x')
\end{sageblock}
Plot $g(x)$, but don't typeset it.
\begin{sagesilent}
  # g is a Maxima thingy, it needs to get converted into a Sage object
  plot1 = plot(g.sage(),x,-1,2*pi)
\end{sagesilent}

You can specify a file format and options for \verb|includegraphics|.
The default is for EPS and PDF files, which are the best choice in
almost all situations. (Although see the section on 3D plotting.)

\sageplot[angle=45, width=.5\textwidth][png]{plot1}

If you use regular \verb|latex| to make a DVI file, you'll see a box,
because DVI files can't include PNG files. If you use \verb|pdflatex|
that will work. See the documentation for details.

When using \verb|\sageplot|, you can pass in just about anything that
Sage can call \verb|.save()| on to produce a graphics file:

\begin{center}
\sageplot{plot1 + plot(f.sage(),x,-1,2*pi,rgbcolor=hue(0.4)), figsize=[1,2]}
\end{center}

To fiddle with aspect ratio, first save the plot object:

\begin{sageblock}
  p = plot(x, 0, 1) + circle((0,0), 1)
  p.set_aspect_ratio(1)
\end{sageblock}

Now plot it and see the circular circle and nice 45 degree angle:

\sageplot[scale=.33]{p}

Indentation and so on works fine.
\begin{sageblock}
 s     = 7
 s2    = 2^s
 P.<x> = GF(2)[]
 M     = matrix(parent(x),s2)
 for i in range(s2):
    p  = (1+x)^i
    pc = p.coeffs()
    a  = pc.count(1)
    for j in range(a):
        idx        = pc.index(1)
        M[i,idx+j] = pc.pop(idx)

 matrixprogram = matrix_plot(M,cmap='Greys')
\end{sageblock}
And here's the picture:

\sageplot{matrixprogram}

Reset \texttt{x} in Sage so that it's not a generator for the polynomial
ring: \sage{var('x')}


\subsection{Plotting (combinatorial) graphs with TikZ}
\label{sec:plotting-graphs-with}

Sage now includes some nice support for plotting graphs using
\href{http://www.texample.net/tikz/}{TikZ}. Here, we mean things with
vertices and edges, not graphs of a function of one or two variables.

First define our graph:

\begin{sageblock}
  g = graphs.PetersenGraph()
  g.set_latex_options(tkz_style='Art')
\end{sageblock}

Now just do \verb|\sage{}| on it to plot it. You'll need to use the
\href{http://altermundus.com/pages/graphtheory.html}{\texttt{tkz-berge}}
package for this to work; that package in turn depends on
\href{http://altermundus.com/pages/graph.html}{\texttt{tkz-graph}} and
TikZ. See
\href{http://sagemath.org/doc/reference/sage/graphs/graph_latex.html}{``\LaTeX{}
  Options for Graphs''} in the Sage reference manual for more details.

\begin{center}
  \sage{g}
\end{center}

The above command just outputs a \texttt{tikzpicture} environment, and
you can control that environment using anything supported by
TikZ---although the output of \verb|\sage{g}| explicitly hard-codes a
lot of things and cannot be flexibly controlled in its current form.

\tikzstyle{every picture}=[rotate=45, scale=1/2]

\begin{center}
  \sage{g}
\end{center}

Here's some more graphs, plotted using the usual plot routines.

\sageplot[scale=.5]{graphs.FlowerSnark().plot()}

\begin{sageblock}
G4 = DiGraph({1:[2,2,3,5], 2:[3,4], 3:[4], 4:[5,7], 5:[6]},\
             multiedges=True)
G4plot = G4.plot(layout='circular')
\end{sageblock}

\sageplot[scale=.5]{G4plot, axes=False}

\subsection{3D plotting}

3D plotting right now (Sage version 4.3.4) is problematic because
there's no convenient way to produce vector graphics. We can make PNGs,
though, so if you pass \verb|sageplot| a graphics object that cannot be
saved to EPS or PDF format, we will automatically save to a PNG file,
which can be used when typesetting a PDF file, but not when creating a
DVI file. However, you can specify the ``\texttt{imagemagick}'' option,
which will use the Imagemagick \texttt{convert} utility to make EPS
files. See the documentation for details.

% FIXME: not sure this works with remote sagetex

\begin{sagesilent}
  x, y = var('x y')
\end{sagesilent}

Here's a 3D plot whose format we do not specify; it will automatically
get saved as a PNG file and won't work when using \texttt{latex} to make
a DVI file.

\sageplot[scale=.5]{plot3d(sin(pi*(x^2+y^2))/2,(x,-1,1),(y,-1,1))}

Here's the (perhaps-not-so-) famous Sage cube graph in 3D.

\begin{sageblock}
  G = graphs.CubeGraph(5)
\end{sageblock}

% need empty [] so sageplot knows you want png format, and aren't
% passing an option to includegraphics
\sageplot[][png]{G.plot3d()}

\section{Pausing Sage\TeX}
\label{sec:pausing-sagetex}

Sometimes you want to ``pause'' for a bit while writing your document if
you have embedded a long calculation or just want to concentrate on the
\LaTeX{} and ignore any Sage stuff. You can use the \verb|\sagetexpause|
and \verb|\sagetexunpause| macros to do that.

\sagetexpause

A calculation: $\sage{factor(2^325 + 1)}$ and a code environment that
simulates a time-consuming calculation. While paused, this will get
skipped over.
\begin{sageblock}
  import time
  time.sleep(15)
\end{sageblock}

Graphics are also skipped: \sageplot{plot(2*sin(x^2) + x^2, (x, 0, 5))}

\sagetexunpause

\section{Make Sage write your \LaTeX{} for you}

With \textsf{Sage\TeX}, you can not only have Sage do your math for you,
it can write parts of your \LaTeX{} document for you! For example, I
hate writing \texttt{tabular} environments; there's too many fiddly
little bits of punctuation and whatnot\ldots and what if you want to add
a column? It's a pain---or rather, it \emph{was} a pain. Just write a
Sage/Python function that outputs a string of \LaTeX{} code, and use
\verb|\sagestr|. Here's how to make Pascal's triangle.

\begin{sageblock}
def pascals_triangle(n):
    # start of the table
    s  = [r"\begin{tabular}{cc|" + "r" * (n+1) + "}"]
    s.append(r"  & & $k$: & \\")
    # second row, with k values:
    s.append(r"  & ")
    for k in [0..n]:
        s.append("& {0} ".format(k))
    s.append(r"\\")
    # the n = 0 row:
    s.append(r"\hline" + "\n" + r"$n$: & 0 & 1 & \\")
    # now the rest of the rows
    for r in [1..n]:
        s.append(" & {0} ".format(r))
        for k in [0..r]:
            s.append("& {0} ".format(binomial(r, k)))
        s.append(r"\\")
    # add the last line and return
    s.append(r"\end{tabular}")
    return ''.join(s)

# how big should the table be?
n = 8
\end{sageblock}

Okay, now here's the table. To change the size, edit \texttt{n} above.
If you have several tables, you can use this to get them all the same
size, while changing only one thing.

\begin{center}
  \sagestr{pascals_triangle(n)}
\end{center}

\section{Include doctest-like examples in your document}

Here are some examples of using the \texttt{sageexample} environment:
\begin{sageexample}
  sage: 1+1
  2
  sage: factor(x^2 + 2*x + 1)
  (x + 1)^2
\end{sageexample}
If you want to see the plain-text output as well as the typeset output,
renew the \texttt{sageexampleincludetextoutput} command to True:
\begin{verbatim}
  \renewcommand{\sageexampleincludetextoutput}{True}
\end{verbatim}
\renewcommand{\sageexampleincludetextoutput}{True}
This can be useful to check that the two outputs are consistent.
% fiddle with spacing to demonstrate the kind of thing mentioned in the
% next paragraph:
\vspace{10mm}

When this environment is near the bottom of the page, it may look like
the page number is the output of a command, when in fact the real output
is on the next page. If the output of a command below looks like
\thepage, don't worry, that's just the page number.

\begin{sageexample}
  sage: 1+1
  2
  sage: factor(x^2 + 2*x + 1)
  (x + 1)^2
\end{sageexample}
Multiline statements are supported, as are triple-quoted strings
delimited by single quotes:
\renewcommand{\sageexampleincludetextoutput}{False}
\begin{sageexample}
  sage: def f(a):
  ...       '''This function is really quite nice,
  ...       although perhaps not very useful.'''
  ...       print "f called with a = ", a
  ...       y = integrate(SR(cyclotomic_polynomial(10)) + a, x)
  ...       return y + 1
  sage: f(x)
  f called with a =  x
  1/5*x^5 - 1/4*x^4 + 1/3*x^3 + x + 1
\end{sageexample}
Note that the ``$f$ called with\ldots'' stuff doesn't get typeset, since
when running Sage on \texttt{example.sage}, that gets printed to the
terminal.

When typesetting your document, the validity of the outputs is not
checked. In fact, the provided outputs are completely ignored:
\renewcommand{\sageexampleincludetextoutput}{True}
\begin{sageexample}
  sage: is_prime(57)
  toothpaste
\end{sageexample}
\renewcommand{\sageexampleincludetextoutput}{False}

However, typesetting your document produces a file named
\texttt{example\_test.sage} containing all the doctest-like examples,
and you can have Sage check them for you with:
\begin{verbatim}
> sage -t  example_test.sage
sage -t  "example_test.sage"
**********************************************************************
File "/home/nthiery/Sage-Combinat/sagetex/example_test.sage", line 30:
    sage: is_prime(57)
Expected:
    toothpaste
Got:
    False
**********************************************************************
1 items had failures:
   1 of  11 in __main__.example_0
***Test Failed*** 1 failures.
For whitespace errors, see the file /.../.sage//tmp/.doctest_example_test.py
         [3.7 s]
----------------------------------------------------------------------
The following tests failed:
        sage -t  "example_test.sage"
Total time for all tests: 3.7 seconds
\end{verbatim}
Please look into this file for the original line numbers.

Beware that \texttt{sage -t} does not handle well file names with
special characters in them (even just \texttt{example-test.sage}).
Also, running doctests on files outside the main Sage library does not
always work, so contact \texttt{sage-support} if you run into
troubles.

Some more examples. This environment is implemented a little bit
differently than the other environments, so it's good to make sure that
definitions are preserved across multiple uses. This will correctly
define $a$, but not print its output because the statement is made up of
a sequence of expressions.
\begin{sageexample}
  sage: 1; 2; a=4; 3; a
  1
  2
  3
  4
\end{sageexample}
After that, Sage should remember that $a = \sage{a}$ and be able to use
that in future \texttt{sageexample} blocks:
\begin{sageexample}
  sage: f(a)
  f called with a =  4
  1/5*x^5 - 1/4*x^4 + 1/3*x^3 - 1/2*x^2 + 5*x + 1
\end{sageexample}

\end{document}
