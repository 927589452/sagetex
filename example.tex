% General example LaTeX file for including Sage calculations and plots
% Build with:
%
%   (pdf)latex example.tex; sage example.sagetex.sage; pdflatex example.tex
%
% Please read README and the documentation of the SageTeX package for
% more information!
%

\documentclass{article}
\title{Examples of embedding Sage in \LaTeX{} with \textsf{Sage\TeX}}
\author{Dan Drake and others}
\usepackage{hyperref}
% If you want to see the examples in the section "Plotting
% (combinatorial) graphs with TikZ" remove the \begin{comment}
% and \end{comment} in that section and uncomment the following line.
%\usepackage{tkz-berge}


\usepackage{sagetex}
%
% If you want SageTeX to use Imagemagick's `convert' utility to make eps
% files from png files when generating a dvi file, add the "imagemagick"
% option above:
%
%    \usepackage[imagemagick]{sagetex}

\setlength{\sagetexindent}{10ex}

\begin{document}
\maketitle

\section{Inline Sage, code blocks}

This is an example $2+2=\sage{2+2}$. If you raise the current year mod
$100$ (which equals $\sage{mod(\the\year, 100)}$) to the power of the
current day ($\the\day$), you get $\sage{Integer(mod(\the\year,
100))^\the\day}$. Also, $\the\year$ modulo $42$ is $\sage{\the\year
\percent 42}$.

Code block which uses a variable \texttt{s} to store the solutions:
\begin{sageblock}
 1+1
 var('a,b,c')
 eqn = [a+b*c==1, b-a*c==0, a+b==5]
 s = solve(eqn, a,b,c)
\end{sageblock}

Solutions of $\mbox{eqn}=\sage{eqn}$:
\[
\sage{s[0]}
\]
\[
\sage{s[1]}
\]

Now we evaluate the following block:
\begin{sageblock}
E = EllipticCurve("37a")
\end{sageblock}
You can't do assignment inside \verb|\sage| macros, since Sage doesn't
know how to typeset the output of such a thing. So you have to use a
code block. The elliptic curve $E$ given by $\sage{E}$ has discriminant
$\sage{E.discriminant()}$.

You can do anything in a code block that you can do in Sage and/or
Python. Here we save an elliptic curve into a file.
\begin{sageblock}
try:
    E = load('E2')
except IOError:
    E = EllipticCurve([1,2,3,4,5])
    E.anlist(100000)
    E.save('E2')
\end{sageblock}

The 9999th Fourier coefficient of $\sage{E}$ is
$\sage{E.anlist(100000)[9999]}$.

The following code block doesn't appear in the typeset file\dots
\begin{sagesilent}
  e = 2
  e = 3*e + 1
\end{sagesilent}
but we can refer to whatever we did in that code block: $e=\sage{e}$.

\begin{sageblock}
  var('x')
  f(x) = log(sin(x)/x)
\end{sageblock}
The Taylor Series of $f$ begins: $\sage{ f.taylor(x, 0, 10) }$.

\section{Plotting}

Here's a very large plot of the elliptic curve $E$.

\sageplot{E.plot(-3,3)}

\begin{sagesilent}
  # the var line is unecessary unless you've defined x to be something
  # other than a symbolic variable
  var('x')
  f(x) = -x^3+3*x^2+7*x-4
\end{sagesilent}

You can use variables to hold plot objects and do stuff with them.
\begin{sageblock}
  p = plot(f, x, -5, 5)
\end{sageblock}

Here's a small plot of $f$ from $-5$ to $5$, which I've centered:

\begin{center} \sageplot[scale=.2]{p} \end{center}

On second thought, use a size of $3/4$ the \verb|\textwidth| and don't
use axes:

\sageplot[width=.75\textwidth]{p, axes=False}

Remember, you're using Sage, and can therefore call upon any of the
software packages Sage is built out of.
\begin{sageblock}
f = maxima('sin(x)^2*exp(x)')
g = f.integrate('x')
\end{sageblock}
Plot $g(x)$, but don't typeset it.
\begin{sagesilent}
  # g is a Maxima thingy, it needs to get converted into a Sage object
  plot1 = plot(g.sage(),x,-1,2*pi)
\end{sagesilent}

You can specify a file format and options for \verb|includegraphics|.
The default is for EPS and PDF files, which are the best choice in
almost all situations. (Although see the section on 3D plotting.)

\sageplot[angle=45, width=.5\textwidth][png]{plot1}

If you use regular \verb|latex| to make a DVI file, you'll see a box,
because DVI files can't include PNG files. If you use \verb|pdflatex|
that will work. See the documentation for details.

When using \verb|\sageplot|, you can pass in just about anything that
Sage can call \verb|.save()| on to produce a graphics file:

\begin{center}
\sageplot[width=.7\textwidth]{plot1 + plot(f.sage(),x,-1,2*pi,rgbcolor=hue(0.4)), figsize=[5,3]}
\end{center}

To fiddle with aspect ratio, first save the plot object:

\begin{sageblock}
  p = plot(x, 0, 1) + circle((0,0), 1)
  p.set_aspect_ratio(1)
\end{sageblock}

Now plot it and see the circular circle and nice 45 degree angle:

\sageplot[scale=.33]{p}

Indentation and so on works fine.
\begin{sageblock}
 s     = 7
 s2    = 2^s
 P.<x> = GF(2)[]
 M     = matrix(parent(x),s2)
 for i in range(s2):
    p  = (1+x)^i
    pc = p.coefficients(sparse=False)
    a  = pc.count(1)
    for j in range(a):
        idx        = pc.index(1)
        M[i,idx+j] = pc.pop(idx)

 matrixprogram = matrix_plot(M,cmap='Greys')
\end{sageblock}
And here's the picture:

\sageplot[scale=.5]{matrixprogram}

Reset \texttt{x} in Sage so that it's not a generator for the polynomial
ring: \sage{var('x')}


\subsection{Plotting (combinatorial) graphs with TikZ}
\label{sec:plotting-graphs-with}

Sage now includes some nice support for plotting graphs using
\href{http://www.texample.net/tikz/}{TikZ}. Here, we mean things with
vertices and edges, not graphs of a function of one or two variables.

The graphics in this section depends on the \texttt{tkz-berge} package,
which is generally only available in newer \TeX{} distributions (for
example, \TeX Live 2011 and newer). That package depends in turn on
TikZ 2.0, which is also only available in newer \TeX{} distributions.
Installing both of those is in some cases nontrivial, so this section is
disabled by default.

If you have TikZ and \texttt{tkz-berge} and friends, remove the
\texttt{comment} environments below.

\begin{comment}

First define our graph:

\begin{sageblock}
  g = graphs.PetersenGraph()
  g.set_latex_options(tkz_style='Art')
\end{sageblock}

Now just do \verb|\sage{}| on it to plot it. You'll need to use the
\texttt{tkz-berge} package for this to work; that package in turn
depends on \texttt{tkz-graph} and TikZ. See
\href{http://altermundus.fr/pages/tkz.html}{\texttt{altermundus.fr/pages/tkz.html}};
if you're using a recent version of \TeX Live, you can use its package
manager to install those packages, or get them from CTAN:
\href{http://www.ctan.org/pkg/tkz-berge}{\texttt{www.ctan.org/pkg/tkz-berge}}.
See
\href{http://doc.sagemath.org/html/en/reference/sage/graphs/graph_latex.html}{``\LaTeX{}
  Options for Graphs''} in the Sage reference manual for more details.

\begin{center}
  \sage{g}
\end{center}

The above command just outputs a \texttt{tikzpicture} environment, and
you can control that environment using anything supported by
TikZ---although the output of \verb|\sage{g}| explicitly hard-codes a
lot of things and cannot be flexibly controlled in its current form.

\tikzstyle{every picture}=[rotate=45, scale=1/2]

\begin{center}
  \sage{g}
\end{center}

\tikzstyle{every picture}=[]

Here's some more graphs, plotted using the usual plot routines.

\sageplot[scale=.5]{graphs.FlowerSnark().plot()}

\begin{sageblock}
G4 = DiGraph({1:[2,2,3,5], 2:[3,4], 3:[4], 4:[5,7], 5:[6]},\
             multiedges=True)
G4plot = G4.plot(layout='circular')
\end{sageblock}

\sageplot[scale=.5]{G4plot, axes=False}

\end{comment}

\subsection{3D plotting}

3D plotting right now (Sage version 4.3.4) is problematic because
there's no convenient way to produce vector graphics. We can make PNGs,
though, so if you pass \verb|sageplot| a graphics object that cannot be
saved to EPS or PDF format, we will automatically save to a PNG file,
which can be used when typesetting a PDF file, but not when creating a
DVI file. However, you can specify the ``\texttt{imagemagick}'' option,
which will use the Imagemagick \texttt{convert} utility to make EPS
files. See the documentation for details.

% FIXME: not sure this works with remote sagetex

\begin{sagesilent}
  x, y = var('x y')
\end{sagesilent}

Here's a 3D plot whose format we do not specify; it will automatically
get saved as a PNG file and won't work when using \texttt{latex} to make
a DVI file.

\sageplot[scale=.5]{plot3d(sin(pi*(x^2+y^2))/2,(x,-1,1),(y,-1,1))}

Here's the (perhaps-not-so-) famous Sage cube graph in 3D.

\begin{sageblock}
  G = graphs.CubeGraph(5)
\end{sageblock}

% need empty [] so sageplot knows you want png format, and aren't
% passing an option to includegraphics
\sageplot[][png]{G.plot3d()}

\section{Pausing Sage\TeX}
\label{sec:pausing-sagetex}

Sometimes you want to ``pause'' for a bit while writing your document if
you have embedded a long calculation or just want to concentrate on the
\LaTeX{} and ignore any Sage stuff. You can use the \verb|\sagetexpause|
and \verb|\sagetexunpause| macros to do that.

\sagetexpause

A calculation: $\sage{factor(2^325 + 1)}$ and a code environment that
simulates a time-consuming calculation. While paused, this will get
skipped over.
\begin{sageblock}
  import time
  time.sleep(15)
\end{sageblock}

Graphics are also skipped: \sageplot{plot(2*sin(x^2) + x^2, (x, 0, 5))}

\sagetexunpause

\section{Make Sage write your \LaTeX{} for you}

With \textsf{Sage\TeX}, you can not only have Sage do your math for you,
it can write parts of your \LaTeX{} document for you! For example, I
hate writing \texttt{tabular} environments; there's too many fiddly
little bits of punctuation and whatnot\ldots and what if you want to add
a column? It's a pain---or rather, it \emph{was} a pain. Just write a
Sage/Python function that outputs a string of \LaTeX{} code, and use
\verb|\sagestr|. Here's how to make Pascal's triangle.

\begin{sageblock}
def pascals_triangle(n):
    # start of the table
    s  = [r"\begin{tabular}{cc|" + "r" * (n+1) + "}"]
    s.append(r"  & & $k$: & \\")
    # second row, with k values:
    s.append(r"  & ")
    for k in [0..n]:
        s.append("& {0} ".format(k))
    s.append(r"\\")
    # the n = 0 row:
    s.append(r"\hline" + "\n" + r"$n$: & 0 & 1 & \\")
    # now the rest of the rows
    for r in [1..n]:
        s.append(" & {0} ".format(r))
        for k in [0..r]:
            s.append("& {0} ".format(binomial(r, k)))
        s.append(r"\\")
    # add the last line and return
    s.append(r"\end{tabular}")
    return ''.join(s)

# how big should the table be?
n = 8
\end{sageblock}

Okay, now here's the table. To change the size, edit \texttt{n} above.
If you have several tables, you can use this to get them all the same
size, while changing only one thing.

\begin{center}
  \sagestr{pascals_triangle(n)}
\end{center}

\section{Include doctest-like examples in your document}

Here are some examples of using the \texttt{sageexample} environment:
\begin{sageexample}
  sage: 2+2
  4
  sage: print('middle')
  middle
  sage: factor(x^2 + 2*x + 1)
  (x + 1)^2
\end{sageexample}
Note above that no output from the \texttt{print} statement appears.
That is because we have to use Python's \texttt{exec} to execute that
statement (and not \texttt{eval()}), and we can't get the output from
that.

That said, if you want to see the plain-text output you put into your
\verb|.tex| file as well as the Sage-computed typeset output, renew the
\texttt{sageexampleincludetextoutput} command to True:
\begin{verbatim}
  \renewcommand{\sageexampleincludetextoutput}{True}
\end{verbatim}
\renewcommand{\sageexampleincludetextoutput}{True}
This can be useful to check that the two outputs are consistent. Here's
the print statement with text output included:
\begin{sageexample}
  sage: print('middle')
  middle
\end{sageexample}
When typesetting your document, the validity of the outputs is not
checked. In fact, the provided outputs are completely ignored:
\renewcommand{\sageexampleincludetextoutput}{True}
\begin{sageexample}
  sage: is_prime(57)
  toothpaste
\end{sageexample}
\renewcommand{\sageexampleincludetextoutput}{False}%
Multiline statements with the ``\verb|....:|'' continuation marks are
supported, as are triple-quoted strings delimited by single quotes
(double quotes won't work):
\begin{sageexample}
  sage: gcd([5656565656,
  ....:      4747474747,
  ....:      123456789])
  1
  sage: mystr = '''my
  ....: string
  ....: has
  ....: several
  ....: lines.'''
  sage: len(mystr)
  28
  sage: def f(a):
  ....:     '''This function is really quite nice,
  ....:     although perhaps not very useful.'''
  ....:     print("f called with a = {}".format(a))
  ....:     y = integrate(SR(cyclotomic_polynomial(10)) + a, x)
  ....:     return y + 1
  sage: f(x)
  f called with a =  x
  1/5*x^5 - 1/4*x^4 + 1/3*x^3 + x + 1
\end{sageexample}
Note that the ``$f$ called with\ldots'' stuff doesn't get typeset, since
when running Sage on \texttt{example.sagetex.sage}, that gets printed to the
terminal.

Typesetting your document produces a file named
\texttt{example\_doctest.sage} containing all the doctest-like examples,
and you can have Sage check them for you with:
\begin{verbatim}
  $ sage -t  example_doctest.sage
\end{verbatim}
You should get a doctest failure from the ``toothpaste'' line above. The
line numbers from \texttt{sage -t} refer to the ``\verb|_doctest.sage|''
file.

Beware that \texttt{sage -t} does not really handle file names with
special characters in them, particularly dashes, dots, and spaces---this
ultimately comes from the way Python interprets \texttt{import}
statements. Also, running doctests on files outside the main Sage
library does not always work, so contact \texttt{sage-support} if you
run into troubles.

Some more examples. This environment is implemented a little bit
differently than the other environments, so it's good to make sure that
definitions are preserved across multiple uses. This will correctly
define $a$, but not print its output because the statement is made up of
a sequence of expressions and we can't use Python's \texttt{eval()}; we
have to use \texttt{exec} and we can't capture the output from that.
\begin{sageexample}
  sage: 1; 2; a=4; 3; a
  1
  2
  3
  4
\end{sageexample}
However, after that, Sage should remember that $a = \sage{a}$ and be
able to use that in future \texttt{sageexample} blocks:
\begin{sageexample}
  sage: f(a)
  f called with a =  4
  1/5*x^5 - 1/4*x^4 + 1/3*x^3 - 1/2*x^2 + 5*x + 1
\end{sageexample}

\section{Plotting functions in Ti\emph{k}Z with Sage\TeX}

(The code in this section should work with any reasonable version of
Ti\emph{k}Z, which means it should work with all but the most terribly
out-of-date \TeX{} installations---but to make sure we can accomodate
everyone, the code here is commented out. You can almost certainly
uncomment and run them. Make sure you do \verb|\usepackage{tikz}| in the
preamble.)

\begin{comment}

The wonderful graphics package TikZ has the ability to plot functions by
reading in a sequence of points from an external file---see chapter 18,
page 193 of the TikZ manual. This facility is designed around files
produced by Gnuplot, but the file format is so simple that it's very
easy to use Sage\TeX{} to generate them. First you need a function that
will evaluate functions and write the results into a file:


% set up plotting stuff
\begin{sageblock}
  def gnuplot(x, y, tvals_, fn):
      """
      Write out a gnuplot-style file of points x(t), y(t).
      """
      tvals = list(tvals_)
      lines = ['#This is a gnuplot-style file written by SageTeX.',
               '#x: {0}'.format(x),
               '#y: {0}'.format(y),
               '#Curve 0, {0} points'.format(len(tvals)),
               '#x y type']
      fmt = lambda _: _.n().str(no_sci=2)
      for t in tvals:
          try:
              lines.append('{0} {1}  i'.format(fmt(x(t)), fmt(y(t))))
          except ValueError, ZeroDivisonError:
              pass
      with open(fn, 'w') as f:
          f.write('\n'.join(lines) + '\n')
\end{sageblock}

There probably should be some more exceptions in that list, and the
above code doesn't check to make sure it's writing real values, but then
again, this is just a file of examples!

Then you define callable functions x and y and pass them in, along with
a sequence of values and a file name. Here's a plot that I used on a
calculus exam:

\begin{sageblock}
  r(t) = 1 - 2*sin(3*t)
  x(t) = r(t)*cos(t)
  y(t) = r(t)*sin(t)
  gnuplot(x, y, srange(0, 2*pi + .05, .05), 'example-tikz1.table')
\end{sageblock}

(Usually you would do that in sagesilent environments, I guess.)

Then you call TikZ with your plot.

\begin{tikzpicture}
 \draw[very thin,->] (-3.25,0) -- (3.25,0);
 \draw[very thin,->] (0,-3.25) -- (0,3.25);
 \draw[smooth] plot file {example-tikz1.table};
\end{tikzpicture}

For regular Cartesian plots, just pass in the identity function for x:

\begin{sageblock}
  x = lambda t: t
  y(t) = t*sin(1/t)
  gnuplot(x, y, [0.01, 0.02..(0.5)] + [0.55, 0.6..2], 'example-tikz2.table')
\end{sageblock}

\begin{center}
\begin{tikzpicture}[scale=3]
 \draw[very thin,->] (-0.25,0) -- (2,0);
 \draw[very thin,->] (0,-1/3) -- (0,1);
 \draw[smooth, red] plot file {example-tikz2.table};
\end{tikzpicture}
\end{center}

This style of plotting will become even more useful and powerful when
the new TikZ Data Visualization library is available---you will be able
to feed TikZ a bunch of data points, and it automatically make a very
nice plot for you, including axes, labels, and so on.

\end{comment}

\section{The \texttt{sagecommandline} environment}

When writing a \TeX{} document about Sage, you may want to show some
examples of commands and their output. But naturally, you are lazy and
don't want to cut and paste the output into your document. ``Why should
I have to do anything? Why can't Sage and \TeX{} cooperate and do it for
me?'' you may cry. Well, they \emph{can} cooperate:

\begin{sagecommandline}
  sage: 1+1
  sage: is_prime(57)
  sage: if is_prime(57):
  ....:     print('prime')
  ....: else:
  ....:     print('composite')
\end{sagecommandline}

Note that the output of the commands is not included in the source file,
but are included in the typeset output.

Because of the way the environment is implemented, not everything is
exactly like using Sage in a terminal: the two commands below (and the
``if is prime'' one above, did you notice that?) would produce some
output, but don't here:

\begin{sagecommandline}
  sage: x = 2010; len(x.divisors())
  sage: print('Hola, mundo!')
\end{sagecommandline}

The difference lies in the Python distinction between statements and
expressions; we can use \texttt{eval()} for an expression and get its
output, but we must use \texttt{exec} for a statement and can't get the
output, if any.

One nice thing is that you can set labels by using an @ sign:

\begin{sagecommandline}
  sage: l = matrix([[1,0,0],[3/5,1,0],[-2/5,-2,1]])
  sage: d = diagonal_matrix([15, -1, 4]) #@\label{diagonal}
  sage: u = matrix([[1,0,1/3],[0,1,2],[0,0,1]]) #@\label{anotherlabel} \# foo
  sage: l*d*u   # this is a comment
\end{sagecommandline}

And then refer to that label: it was on line \ref{diagonal}, which is on
page \pageref{diagonal}. Note that the other text after the hash mark on
that line does not get typeset as a comment, and that you cannot have
any space between the hash mark and the~@. You will also need to typeset
your document \emph{twice}

You can also typeset the output by changing the value of
\verb|\sagecommandlinetextoutput| to False:
\renewcommand{\sagecommandlinetextoutput}{False}
\begin{sagecommandline}
  sage: l*d*u
  sage: x = var('x')
  sage: (1-cos(x)^2).trig_simplify()
\end{sagecommandline}

\renewcommand{\sagecommandlinetextoutput}{True}

The Sage input and output is typeset using the \texttt{listings} package
with the styles \texttt{SageInput} and \texttt{SageOutput},
respectively. If you don't like the defaults you can change them. It is
recommended to derive from \texttt{DefaultSageInput} and
\texttt{DefaultSageOutput}, for example\ldots
\lstdefinestyle{SageInput}{style=DefaultSageInput,basicstyle={\color{red}}}
\lstdefinestyle{SageOutput}{style=DefaultSageOutput,basicstyle={\color{green}}}
makes things overly colorful:
\begin{sagecommandline}
  sage: pi.n(100)
\end{sagecommandline}
\lstdefinestyle{SageInput}{style=DefaultSageInput}
\lstdefinestyle{SageOutput}{style=DefaultSageOutput}

Plotting things doesn't automatically pull in the plot, just the text
representation of the plot (the equivalent of applying \texttt{str()} to
it):

\begin{sagecommandline}
  sage: plot(sin(x), (x, 0, 2*pi))
\end{sagecommandline}

You can include output, but it will be ignored. This is useful for
doctesting, as all the \texttt{sagecommandline} environment things get
put into the ``\texttt{\_doctest.sage}'' file. However, note that if you
don't include any output, then the corresponding doctest will fail if
the command produces output. The doctest output from this file will have
lots of failures because not many of the commands have output included
in the source \texttt{.tex} file.

The command below has incorrect output included in the \texttt{.tex}
file; in the PDF, you see the correct Sage-computed answer, but if you
do \texttt{sage -t example\_doctest.sage} you will get a genuine doctest
failure.
\begin{sagecommandline}
  sage: factor(x^2 + 2*x + 1)
  (x + 999)^2
\end{sagecommandline}

\end{document}
