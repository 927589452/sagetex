% \iffalse meta-comment
%
% Copyright (C) 2008--2010 by Dan Drake <ddrake (at) member (dot) ams (dot) org>
% -------------------------------------------------------
%
% See the "Copying and licenses" section in this file for the terms
% under which this source code and documentation may be modified and
% distributed.
%
% This package is not licensed under the LPPL, but it seems reasonable
% to say:
%
%   This work has the LPPL maintenance status `maintained'.
%
%   The Current Maintainer of this work is Dan Drake.
%
%   This work consists of the files sagetexpackage.dtx, py-and-sty.dtx,
%   scripts.dtx, remote-sagetex.dtx, sagetexpackage.ins, example.tex,
%   and the derived files sagetex.sty, sagetex.py, sagetexparse.py,
%   makestatic.py, extractsagecode.py, and remote-sagetex.py.
%
% \fi
%
% \iffalse
%<*driver>
\ProvidesFile{sagetexpackage.dtx}
%</driver>
%<latex>\NeedsTeXFormat{LaTeX2e}
%<latex>\ProvidesPackage{sagetex}
%<python>__version__ = """
%<*latex|python>
  [2010/02/15 v2.2.4 embedding Sage into LaTeX documents]
%</latex|python>
%<python>""".strip()
%<latex>\newcommand{\ST@ver}{2010/02/15 v2.2.4}
%<*driver>
\documentclass{ltxdoc}
\usepackage{sagetex}
\usepackage{xspace}
\usepackage{tikz}
\usepackage{hyperref}
% \iffalse
% Work around a problem with using Docstrip and hyperref; for macros and
% such described with DescribeMacro and friends, plain Docstrip puts
% something like this into the .idx file:
%
%  \indexentry{sage=\verb!*+\sage+|usage}{5}
%
% and defines \usage{}. hyperref comes along and sneakily alters those
% lines and adds:
%
%  \indexentry{sage=\verb!*+\sage+|usage|hyperpage}{5}
%
% and makeindex gets confused because you can't have two | things in one
% indexentry. I could probably figure out how to fix this in LaTeX, but
% it's easier to run sed on the .idx file to remove the extra | and
% define a new macro that makes the text italic and puts in the
% hyperlink.
%
% Another option is to forget about all this and just pass the
% hyperindex=false option to hyperref, but then you don't get
% hyperlinked page numbers.
% \fi
\newcommand{\usagehyperpage}[1]{\textit{\hyperpage{#1}}}

\renewcommand{\subsubsectionautorefname}{section}
\renewcommand{\subsectionautorefname}{section}
\EnableCrossrefs
\CodelineIndex
\RecordChanges
\begin{document}
\DocInput{sagetexpackage.dtx}
\StopEventually{
\section{Credits and acknowledgments}

According to the original README file, this system was originally done
by Gonzalo Tornaria and Joe Wetherell. Later Harald Schilly made some
improvements and modifications. Almost all the examples in the
|example.tex| file are from Harald.

Dan Drake rewrote and extended the style file (there is effectively zero
original code there), made significant changes to the Python module, put
both files into \textsf{Docstrip} format, and wrote all the
documentation and extra Python scripts.

Many thanks to Jason Grout for his numerous comments, suggestions, and
feedback.

\section{Copying and licenses}
\label{sec:copying-licenses}

If you are unnaturally curious about the current state of the \ST
package, you can visit \url{http://www.bitbucket.org/ddrake/sagetex/}.
There is a Mercurial repository and other stuff there.

As for the terms and conditions under which you can copy and modify \ST:

The \emph{source code} of the \ST package may be redistributed and/or
modified under the terms of the GNU General Public License as published
by the Free Software Foundation, either version 2 of the License, or (at
your option) any later version. To view a copy of this license, see
\url{http://www.gnu.org/licenses/} or send a letter to the Free Software
Foundation, Inc., 51 Franklin Street, Fifth Floor, Boston, MA
02110-1301, USA.

The \emph{documentation} of the \ST package is licensed under the
Creative Commons Attribution-Share Alike 3.0 License. To view a copy of
this license, visit \url{http://creativecommons.org/licenses/by-sa/3.0/}
or send a letter to Creative Commons, 171 Second Street, Suite 300, San
Francisco, California, 94105, USA.

I am not terribly dogmatic about these licenses, so if you would like to
do something with \ST that's not possible under these license
conditions, please contact me. I will likely be receptive to
suggestions.

\iffalse meta-comment
 I've run into a situation where the index wants to start on the very
 last line of the page, and I actually get errors: ``Package multicol
 Error: Error saving partial page.'' The problem goes away if I fiddle
 with some lines so that the index starts elsewhere. Putting in a
 clearpage below makes the index start nicely in the middle of a page
 (until my change history gets too long!) and solves all those
 problems. It can be removed/pulled into this comment if you're
 confident the page break problems won't occur.
\clearpage
\fi
}

\DocInput{py-and-sty.dtx}
\DocInput{scripts.dtx}
\DocInput{remote-sagetex.dtx}
\Finale
\PrintChanges
\PrintIndex
\end{document}
%</driver>
% \fi
%
% \CheckSum{344}
%
% \CharacterTable
%  {Upper-case    \A\B\C\D\E\F\G\H\I\J\K\L\M\N\O\P\Q\R\S\T\U\V\W\X\Y\Z
%   Lower-case    \a\b\c\d\e\f\g\h\i\j\k\l\m\n\o\p\q\r\s\t\u\v\w\x\y\z
%   Digits        \0\1\2\3\4\5\6\7\8\9
%   Exclamation   \!     Double quote  \"     Hash (number) \#
%   Dollar        \$     Percent       \%     Ampersand     \&
%   Acute accent  \'     Left paren    \(     Right paren   \)
%   Asterisk      \*     Plus          \+     Comma         \,
%   Minus         \-     Point         \.     Solidus       \/
%   Colon         \:     Semicolon     \;     Less than     \<
%   Equals        \=     Greater than  \>     Question mark \?
%   Commercial at \@     Left bracket  \[     Backslash     \\
%   Right bracket \]     Circumflex    \^     Underscore    \_
%   Grave accent  \`     Left brace    \{     Vertical bar  \|
%   Right brace   \}     Tilde         \~}
%
% \changes{v1.0}{2008/03/03}{Initial version}
% \changes{v1.1}{2008/03/05}{Wrapped user-provided Sage code in
% try/except clauses; plotting now has optional format argument}
% \changes{v1.2}{2008/03/07}{Imagemagick option; better documentation}
% \changes{v1.3.1}{2008/03/10}{Internal variables renamed; fixed typos}
% \changes{v1.4}{2008/03/11}{MD5 fix, percent sign macro, CTAN upload}
% \changes{v2.0}{2008/12/16}{External Python scripts for parsing
% SageTeX-ified documents, tons of documentation improvements,
% sagetex.py refactored, include in Sage as spkg}
% \changes{v2.0}{2009/01/09}{Miscellaneous fixes, final 2.0 version}
% \changes{v2.1}{2009/05/12}{Add pausing support}
% \changes{v2.1}{2009/05/12}{Get version written to .py file}
% \changes{v2.2}{2009/06/17}{Add remote-sagetex.py script}
%
% \GetFileInfo{sagetex.sty}
%
% \DoNotIndex{\newcommand,\newenvironment,\the}
%
% \newcommand{\ST}{\textsf{Sage\TeX}\xspace}
% \iffalse
% so I don't have to put \ or {} after \LaTeX:
% \fi
% \newcommand{\LTX}{\LaTeX\xspace}
%
% \newcommand{\TikZ}{Ti\emph{k}Z\xspace}
%
% \newcommand{\warningbox}[1]{\colorbox[rgb]{1, 0.6, 0.6}%
% {\parbox{.97\textwidth}{#1}}}
%
% \tikzstyle{box}=[draw, shape=rectangle, thick]
%
% \title{The \ST{} package\thanks{This document corresponds to
%     \ST \fileversion, dated \filedate.}}
%
% \author{Dan Drake and others\thanks{Author's website:
%     \href{http://mathsci.kaist.ac.kr/~drake/}
%     {\texttt{mathsci.kaist.ac.kr/$\sim$drake/}}.}}
%
% \iffalse
% Don't put any other code from this file into the .sty or .py
% file:
%<*!latex&!python>
% This lets us use verbatim environments in the documentation.
% \fi
%
% \maketitle
%
% \section{Introduction}
%
% Why should the Haskell and R folks have all the fun?
% \href{http://www.haskell.org/haskellwiki/Literate_programming}{Literate
%   Haskell} is a popular way to mix Haskell source code and \LTX
% documents. (Actually any kind of text or document, but here we're
% concerned only with \LTX.) You can even embed Haskell code in your
% document that writes part of your document for you. Similarly, the R
% statistical computing environment includes
% \href{http://tug.org/pracjourn/2008-1/zahn/}{Sweave}, which lets you
% do the same thing with R code and \LTX.
%
% The \ST package allows you to do (roughly) the same thing with the
% Sage mathematics software suite (see \url{http://sagemath.org}) and
% \LTX. (If you know how to write literate Haskell: the |\eval| command
% corresponds to |\sage|, and the |code| environment to the |sageblock|
% environment.) As a simple example, imagine in your document you are
% writing about how to count license plates with three letters and three
% digits. With this package, you can write something like this:
% \begin{quote}
%  |There are $26$ choices for each letter, and $10$ choices for|\\
%  |each digit, for a total of $26^3 \cdot 10^3 =|\\
%  |\sage{26^3*10^3}$ license plates.|
% \end{quote}
% and it will produce
% \begin{quote}
%   There are $26$ choices for each letter, and $10$ choices for each
%   digit, for a total of $26^3 \cdot 10^3 = \sage{26^3 * 10^3}$ license
%   plates.
% \end{quote}
% The great thing is, you don't have to do the multiplication. Sage does
% it for you. This process mirrors one of the great aspects of \LTX:
% when writing a \LTX document, you can concentrate on the logical
% structure of the document and trust \LTX and its army of packages to
% deal with the presentation and typesetting. Similarly, with \ST, you
% can concentrate on the mathematical structure (``I need the product of
% $26^3$ and $10^3$'') and let Sage deal with the base-$10$ presentation
% of the number.
%
% A less trivial, and perhaps more useful example is plotting. You can
% include a plot of the sine curve without manually producing a plot,
% saving an EPS or PDF file, and doing the |\includegraphics| business
% with the correct filename yourself. If you write this:
% \begin{quote}
% |Here is a lovely graph of the sine curve:|
%
% |\sageplot{plot(sin(x), x, 0, 2*pi)}|
% \end{quote}
% in your \LTX file, it produces
% \begin{quote}
% Here is a lovely graph of the sine curve:
%
% \sageplot{plot(sin(x), x, 0, 2*pi)}
% \end{quote}
% Again, you need only worry about the logical/mathematical structure of
% your document (``I need a plot of the sine curve over the interval
% $[0, 2\pi]$ here''), while \ST{} takes care of the gritty details of
% producing the file and sourcing it into your document.
%
% \paragraph{But \texttt{\bslash sageplot} isn't magic} I just tried to
% convince you that \ST makes putting nice graphics into your document
% very easy; let me turn around and warn you that using graphics
% \emph{well} is not easy, and no \LTX package or Python script will
% ever make it easy. What \ST does is make it easy to \emph{use Sage} to
% create graphics; it doesn't magically make your graphics good,
% appropriate, or useful. (For instance, look at the sine plot above---I
% would say that a truly lovely plot of the sine curve would not mark
% integer points on the $x$-axis, but rather $\pi/2$, $\pi$, $3\pi/2$,
% and $2\pi$.)
%
% Till Tantau has some good commentary on the use of graphics in
% \href{http://www.ctan.org/tex-archive/graphics/pgf/}{section 6 of the
%   \textsc{pgf} manual}. You should always give careful thought and
% attention to creating graphics for your document; I have in mind that
% a good workflow for using \ST for plotting is something like this:
%
% \begin{enumerate}
%   \item Figure out what sort of graphic you need to communicate your
%   ideas or information.
%   \item Fiddle around in Sage until you get a graphics object and set
%   of options that produce the graphic you need.
%   \item Copy those commands and options into \ST commands in your
%   \LTX document.
% \end{enumerate}
%
% The \ST{} package's plotting capabilities don't help you find those
% Sage commands to make your lovely plot, but they do eliminate the need
% to muck around with saving the result to a file, remembering the
% filename, including it into your document, and so on. In
% \autoref{sec:usage}, we will see what what we can do with \ST.
%
% \section{Installation}
% \label{sec:installation}
% \changes{v2.0}{2009/01/14}{Fixed up installation section, final
% \emph{final} 2.0}
% \changes{v2.2.3}{2009/12/29}{Rewrote installation section to reflect
% inclusion as standard spkg}
%
% \newcommand{\sageroot}{\$SAGE\_ROOT}
%
% \ST needs two parts to work: a Python module known to Sage, and a \LTX
% package known to \TeX. These two parts need to come from the same
% version of \ST to guarantee that everything works properly. As of Sage
% version 4.3.1, \ST comes included with Sage, so you only need to make
% \texttt{sagetex.sty}, the \LTX package, known to \TeX. Full details of
% this are in the Sage Installation guide at
% \href{http://sagemath.org/doc/installation/}{\texttt{sagemath.org/doc/installation/}}
% in the obviously-named section ``Make \ST known to \TeX''. Here's a
% brief summary of how to do that:
%
% \begin{itemize}
% \item Copy \texttt{sagetex.sty} to the same directory as your
%   document. This always works, but requires lots of copies of
%   \texttt{sagetex.sty} and is prone to version skew.
%
% \item Copy the directory containing \texttt{sagetex.sty} to your home
%   directory with a command like
%   \begin{quotation}
%     \texttt{cp -R \sageroot/local/share/texmf \textasciitilde/}
%   \end{quotation}
%   where \texttt{\sageroot} is replaced with the location of your
%   Sage installation.
%
% \item Use the environment variable \texttt{TEXINPUTS} to tell \TeX{}
%   to search the directory containing \texttt{sagetex.sty}; in the bash
%   shell, you can do
%   \begin{quotation}
%     \texttt{export TEXINPUTS=\sageroot/local/share/texmf//:}
%   \end{quotation}
%   You should again replace \texttt{\sageroot} with the location of
%   your Sage installation.
% \end{itemize}
%
% The best method is likely the second; while that does require you to
% recopy the files every time you update your copy of Sage, it does not
% depend on your shell, so if you use, say, Emacs with Auc\TeX{} or some
% other editor environment, everything will still work since \TeX's
% internal path-searching mechanisms can find \texttt{sagetex.sty}.
%
% Note that along with \texttt{sagetex.sty}, this documentation, an
% example file, and other useful scripts are all located in the
% directory \texttt{\sageroot/local/share/texmf}.
%
% \subsection{\ST and \TeX Live}
%
% \ST is included in \TeX Live, which is very nice, but because the Python
% module and \LTX package for \ST need to be synchronized, if you use
% the \LTX package from \TeX Live and the Python module from Sage, they
% may not work together if they are from different versions of \ST.
% Because of this, I strongly recommend using \ST only from what is
% included with Sage and ignoring what's included with \TeX Live.
%
% \subsection{The noversioncheck option}
% \label{sec:noversioncheck}
%
% As of version 2.2.4, \ST automatically checks to see if the versions
% of the style file and Python module match. This is intended to prevent
% strange version mismatch problems, but if you would like to use
% mismatched sources, you can---at your peril---give the
% |noversioncheck| option when you load the \ST package. Don't be
% surprised if things don't work when you do this.
%
% If you are considering using this option because the Sage script
% complained and exited, you really should just get the \LTX and Python
% modules synchronized. Every copy of Sage since version 4.3.2 comes
% with a copy of |sagetex.sty| that is matched up to Sage's baked-in \ST
% support, so you can always use that. See the
% \href{http://sagemath.org/doc/installation/}{\ST section of the Sage
%   installation guide}.
%
% \subsection{Using \TeX Shop}
% \label{sec:using-texshop}
% \changes{v2.0.1}{2009/03/10}{Add \TeX Shop info}
%
% Starting with version 2.25,
% \href{http://www.uoregon.edu/~koch/texshop/}{\TeX Shop} includes
% support for \ST. If you move the file |sage.engine| from
% |~/Library/TeXShop/Engines/Inactive/Sage| to
% |~/Library/TeXShop/Engines| and put the line
% \begin{quotation}
%   |%!TEX TS-program = sage|
% \end{quotation}
% at the top of your document, then \TeX Shop will automatically run Sage
% for you when compiling your document.
%
% Note that you will need to make sure that \LTX can find |sagetex.sty|
% using any of the methods above. You also might need to edit the
% |sage.engine| script to reflect the location of your Sage
% installation.
%
% \subsection{Other scripts included with \ST}
% \label{sec:inst-other-scripts}
%
% \ST includes several Python files which may be useful for working with
% ``\ST-ified'' documents. The |remote-sagetex.py| script allows you to
% use \ST on a computer that doesn't have Sage installed; see
% \autoref{sec:remote-sagetex} for more information.
%
% Also included are |makestatic.py| and |extractsagecode.py|, which are
% convenience scripts that you can use after you've written your
% document. See \autoref{sec:makestatic-usage} and
% \autoref{sec:extractsagecode} for information on using those scripts.
% The file |sagetexparse.py| is a module used by both those scripts.
% These three files are independent of \ST. If you install from a spkg,
% these scripts can be found in \texttt{\sageroot/local/share/texmf/}.
%
% \section{Usage}
% \label{sec:usage}
%
% Let's begin with a rough description of how \ST works. Naturally the
% very first step is to put |\usepackage{sagetex}| in the preamble of
% your document. When you use macros from this package and run \LTX on
% your file, along with the usual zoo of auxiliary files, a |.sage| file
% is written with the same basename as your document. This is a Sage
% source file that uses the Python module from this package and when you
% run Sage on that file, it will produce a |.sout| file. That file
% contains \LTX code that, when you run \LTX on your source file again,
% will pull in all the results of Sage's computation.
%
% All you really need to know is that to typeset your document, you need
% to run \LTX, then run Sage, then run \LTX again. You can even ``run
% Sage'' on a computer that doesn't have Sage installed by using the
% |remote-sagetex.py| script; see \autoref{sec:remote-sagetex}. Whenever
% this manual says ``run Sage'', you can either directly run Sage, or
% use the |remote-sagetex.py| script.
%
% Also keep in mind that everything you send to Sage is done within one
% Sage session. This means you can define variables and reuse them
% throughout your \LTX document; if you tell Sage that |foo| is
% $12$, then anytime afterwards you can use |foo| in your Sage code and
% Sage will remember that it's $12$---just like in a regular Sage
% session.
%
% Now that you know that, let's describe what macros \ST provides and
% how to use them. If you are the sort of person who can't be bothered
% to read documentation until something goes wrong, you can also just
% look through the |example.tex| file included with this
% package.\footnote{Then again, if you're such a person, you're probably
% not reading this, and are already fiddling with
% \texttt{example.tex}\dots}\\
%
% \noindent\warningbox{\textbf{WARNING!} When you run \LTX on a file
%   named \texttt{\meta{filename}.tex}, the file
%   \texttt{\meta{filename}.sage} is created---and will be
%   \emph{automatically overwritten} if it already exists. If you keep
%   Sage scripts in the same directory as your \ST-ified \LTX documents,
%   use a different file name!}
%
% \paragraph{The final option} On a similar note, \ST, like many \LTX
% packages, accepts the |final| option. When passed this option, either
% directly in the |\usepackage| line, or from the |\documentclass| line,
% \ST will not write a |.sage| file. It will try to read in the |.sout|
% file so that the \ST macros can pull in their results. However, this
% will not allow you to have an independent Sage script with the same
% basename as your document, since to get the |.sout| file, you need the
% |.sage| file.
%
% \subsection{Inline Sage}
% \label{sec:sagemacro-usage}
%
% \DescribeMacro{sage}
% \fbox{\texttt{\bslash sage}\marg{Sage code}}
% takes whatever Sage code you give it, runs Sage's |latex| function on
% it, and puts the result into your document.
%
% For example, if you do |\sage{matrix([[1, 2], [3,4]])^2}|, then that
% macro will get replaced by
% \begin{quote}
% |\left(\begin{array}{rr}|\\
% |7 & 10 \\|\\
% |15 & 22|\\
% |\end{array}\right)|
% \end{quote}
% in your document---that \LTX code is exactly exactly what you get
% from doing
% \begin{center}
% |latex(matrix([[1, 2], [3,4]])^2)|
% \end{center}
% in Sage.
%
% Note that since \LTX will do macro expansion on whatever you give
% to |\sage|, you can mix \LTX variables and Sage variables! If
% you have defined the Sage variable |foo| to be $12$ (using, say, the
% |sageblock| environment), then you can do something like this:
% \begin{quote}
% |The prime factorization of the current page number plus foo|\\
% |is $\sage{factor(foo + \thepage)}$|.
% \end{quote}
%
% Here, I'll do just that right now: the prime factorization of the
% current page number plus $12$ is $\sage{factor(\thepage + 12)}$.
% (Wrong answer? See footnote.\footnote{Is the above factorization
%   wrong? If the current page number plus $12$ is one larger than the
%   claimed factorization, another Sage/\LTX cycle on this source file
%   should fix it. Why? The first time you run \LTX on this file, the
%   sine graph isn't available, so the text where I've talked about the
%   prime factorization is back one page. Then you run Sage, and it
%   creates the sine graph and does the factorization. When you run \LTX
%   again, the sine graph pushes the text onto the next page, but it
%   uses the Sage-computed value from the previous page. Meanwhile, the
%   |.sage| file has been rewritten with the correct page number, so if
%   you do another Sage/\LTX cycle, you should get the correct value
%   above. However, in some cases, even \emph{that} doesn't work because
%   of some kind of \TeX{} weirdness in ending the one page a bit short
%   and starting another.}) The |\sage| command doesn't automatically
% use math mode for its output, so be sure to use dollar signs or a
% displayed math environment
% as appropriate.\\
%
% \DescribeMacro{\sagestr}
% \fbox{\texttt{\bslash sagestr}\marg{Sage code}}
% is identical to |\sage|, but is does \emph{not} run Sage's |latex|
% function on the code you give it; it simply runs the Sage code and
% pulls the result into your \LTX file. This is useful for calling
% functions that return \LTX code; see the example file distributed
% along with \ST for a demonstration of using this command to easily
% produce a table.
%
% \DescribeMacro{\percent} If you are doing modular arithmetic or string
% formatting and need a percent sign in a call to |\sage| (or
% |\sageplot|), you can use |\percent|. Using a bare percent sign won't
% work because \LTX will think you're starting a comment and get
% confused; prefixing the percent sign with a backslash won't work
% because then ``|\%|'' will be written to the |.sage| file and Sage
% will get confused. The |\percent| macro makes everyone happy.
%
% Note that using |\percent| inside the verbatim-like environments
% described in \autoref{sec:codeblockenv} isn't necessary; a literal
% ``\percent'' inside such an environment will get written, uh, verbatim
% to the |.sage| file.
%
% \subsection{Graphics and plotting}
% \label{sec:graphics-plotting}
%
% \noindent \DescribeMacro{\sageplot}
% \fbox{\texttt{\bslash sageplot}\oarg{ltx opts}\oarg{fmt}\{\meta{graphics
% obj}, \meta{keyword args}\}}
% plots the given Sage graphics object and runs an
% |\includegraphics| command to put it into your document. It does not
% have to actually be a plot of a function; it can be any Sage graphics
% object. The options are described in \autoref{t:sageplotopts}.
%
% \begin{table}[h]
%   \centering
%   \begin{tabular}{l p{8cm}}
%   Option & Description \\
%   \hline
%   \meta{ltx options} & Any text here is passed directly into the
%   optional arguments (between the square brackets) of an
%   |\includegraphics| command. If not specified,
%   ``|width=.75\textwidth|'' will be used.\\
%   \meta{fmt} & You can optionally specify a file extension here; Sage
%   will then try to save the graphics object to a file with extension
%   \emph{fmt}. If not specified, \ST\ will save to EPS and PDF files.\\
%   \meta{graphics obj} & A Sage object on which you can call |.save()|
%   with a graphics filename.\\
%   \meta{keyword args} & Any keyword arguments you put here will
%   all be put into the call to |.save()|.
%   \end{tabular}
%   \caption{Explanation of options for the \texttt{\bslash sageplot}
%   command.}
%   \label{t:sageplotopts}
% \end{table}
%
% This setup allows you to control both the Sage side of things, and the
% \LTX side. For instance, the command
% \begin{quote}
% |\sageplot[angle=30, width=5cm]{plot(sin(x), 0, pi), axes=False,|\\
% |chocolate=True}|
% \end{quote}
% will run the following command in Sage:
% \begin{quote}
% |sage: plot(sin(x), 0, pi).save(filename=autogen, axes=False,|\\
% |chocolate=True)|
% \end{quote}
% Then, in your \LTX file, the following command will be issued
% automatically:
% \begin{center}
% |\includegraphics[angle=30, width=5cm]{autogen}|
% \end{center}
% You can specify a file format if you like. This must be the
% \emph{second} optional argument, so you must use empty brackets if
% you're not passing anything to |\includegraphics|:
% \begin{center}
% |\sageplot[][png]{plot(sin(x), x, 0, pi)}|
% \end{center}
% The filename is automatically generated, and unless you specify a
% format, both EPS and PDF files will be generated. This allows you to
% freely switch between using, say, a DVI viewer (many of which have
% support for automatic reloading, source specials and make the writing
% process easier) and creating PDFs for posting on the web or emailing
% to colleagues.
%
% If you ask for, say, a PNG file, keep in mind that ordinary |latex|
% and DVI files have no support for PNG files; \ST detects this and will
% warn you that it cannot find a suitable file if using
% |latex|.\footnote{We use a typewriter font here to indicate the
% executables which produce DVI and PDF files, respectively, as
% opposed to ``\LTX'' which refers to the entire typesetting system.}
% If you use |pdflatex|, there will be no problems because PDF files can
% include PNG graphics.
%
% When \ST cannot find a graphics file, it inserts this into your
% document:
%
% \begin{center}
%   \framebox[2cm]{\rule[-1cm]{0cm}{2cm}\textbf{??}}
% \end{center}
%
% \noindent That's supposed to resemble the image-not-found graphics
% used by web browsers and use the traditional ``\textbf{??}'' that \LTX
% uses to indicate missing references.
%
% You needn't worry about the filenames; they are automatically
% generated and will be put into the directory
% |sage-plots-for-filename.tex|. You can safely delete that directory
% anytime; if \ST can't find the files, it will warn you to run Sage to
% regenerate them.\\
%
% \noindent\warningbox{\textbf{WARNING!} When you run Sage on your
%   |.sage| file, all files in the
%   \texttt{sage-plots-for-\meta{filename}.tex} directory \emph{will be
%     deleted!} Do not put any files into that directory that you do not
%   want to get automatically deleted.}
%
% \paragraph{The epstopdf option} One of the graphics-related options
% supported by \ST is |epstopdf|. This option causes \ST to use the
% |epstopdf| command to convert EPS files into PDF files. Like with the
% |imagemagick| option, it doesn't check to see if the |epstopdf|
% command exists or add options: it just runs the command. This option
% was motivated by a bug in the matplotlib PDF backend which caused it
% to create invalid PDFs. Ideally, this option should never be
% necessary; if you do need to use it, file a bug!
%
% \subsubsection{3D plotting}
%
% Right now there is, to put it nicely, a bit of tension between the
% sort of graphics formats supported by |latex| and |pdflatex|, and the
% graphics formats supported by Sage's 3D plotting systems. \LTX is
% happiest, and produces the best output, with EPS and PDF files, which
% are vector formats. Tachyon, Sage's 3D plotting system, produces
% bitmap formats like BMP and PNG.
%
% Because of this, when producing 3D plots with |\sageplot|, \emph{you
% must specify a file format}. The PNG format is compressed and lossless
% and is by far the best choice, so use that whenever possible. (Right
% now, it is always possible.) If you do not specify a file format, or
% specify one that Tachyon does not understand, it will produce files in
% the Targa format with an incorrect extension and \LTX (both |latex|
% and |pdflatex|) will be profoundly confused. Don't do that.
%
% Since |latex| does not support PNGs, when using 3D plotting (and
% therefore a bitmap format like PNG), \ST will always issue a warning
% about incompatible graphics if you use |latex|, provided you've
% processed the |.sage| file and the PNG file exists. The only exception
% is if you're using the |imagemagick| option below. (Running |pdflatex|
% on the same file will work, since PDF files can include PNG files.)
%
% \paragraph{The imagemagick option} As a response to the above issue,
% the \ST package has an |imagemagick| option. If you specify this
% option in the preamble of your document with the usual
% ``|\usepackage[imagemagick]{sagetex}|'', then when you are compiling
% your document using |latex|, any |\sageplot| command which requests a
% non-default format will cause the \ST Python script to convert the
% resulting file to EPS using the Imagemagick |convert| utility. It does
% this by executing ``|convert filename.EXT filename.eps|'' in a
% subshell. It doesn't add any options, check to see if the |convert|
% command exists or belongs to Imagemagick---it just runs the command.
%
% The resulting EPS files are not very high quality, but they will work.
% This option is not intended to produce good graphics, but to allow you
% to see your graphics when you use |latex| and DVI files while writing
% your document.
%
% \subsubsection{But that's not good enough!}
% \label{sec:notgoodenough}
%
% The |\sageplot| command tries to be both flexible and easy to use, but
% if you are just not happy with it, you can always do things manually:
% inside a |sagesilent| environment (see the next section) you could do
% \begin{quote}
%   |your special commands|\\
%   |x = your graphics object|\\
%   |x.save(filename=myspecialfile.ext, options, etc)|
% \end{quote}
% and then, in your source file, do your own |\includegraphics| command.
% The \ST package gives you full access to Sage and Python and doesn't
% turn off anything in \LTX, so you can always do things manually.
%
% \subsection{Verbatim-like environments}
% \label{sec:codeblockenv}
%
% The \ST package provides several environments for typesetting and
% executing blocks of Sage code.\\
%
% \DescribeEnv{sageblock} Any text between |\begin{sageblock}| and
% |\end{sageblock}| will be typeset into your file, and also written into
% the |.sage| file for execution. This means you can do something like
% this:
% \begin{quote}
% |\begin{sageblock}|\\
% |   var('x')|\\
% |   f(x) = sin(x) - 1|\\
% |   g(x) = log(x)|\\
% |   h(x) = diff(f(x) * g(x), x)|\\
% |\end{sageblock}|
% \end{quote}
% and then anytime later write in your source file
% \begin{quote}
%   |We have $h(2) = \sage{h(2)}$, where $h$ is the derivative of|\\
%   |the product of $f$ and $g$.|
% \end{quote}
% and the |\sage| call will get correctly replaced by $\sage{
% diff((sin(x) - 1)*log(x), x)(x=1)}$. You can use any Sage or Python
% commands inside a |sageblock|; all the commands get sent directly to
% Sage.\\
%
% \iffalse meta-comment
%   Sadly, we can't use sageblock or similar environments in this file!
%   If you prefix the lines inside the environment with percent signs,
%   then those percent signs get written to your .sage file. If you
%   *don't* prefix the lines with percent signs, those lines get written
%   into the .sty or .py file. It's just too tricky to get docstrip and
%   the verbatim stuff to play nicely together. I'd have to redefine how
%   those environments work, and get them to strip off initial percents.
% \fi
%
% \DescribeEnv{sagesilent} This environment is like |sageblock|, but it
% does not typeset any of the code; it just writes it to the |.sage|
% file. This is useful if you have to do some setup in Sage that is not
% interesting or relevant to the document you are writing.\\
%
% \DescribeEnv{sageverbatim} This environment is the opposite of the one
% above: whatever you type will be typeset, but not written into the
% |.sage| file. This allows you to typeset psuedocode, code that will
% fail, or take too much time to execute, or whatever.\\
%
% \DescribeEnv{comment} Logically, we now need an environment that
% neither typesets nor executes your Sage code\ldots but the |verbatim|
% package, which is always loaded when using \ST, provides such an
% environment: |comment|. Another way to do this is to put stuff between
% |\iffalse| and |\fi|.\\
%
% \DescribeMacro{\sagetexindent} There is one final bit to our
% verbatim-like environments: the indentation. The \ST package defines a
% length |\sagetexindent|, which controls how much the Sage code is
% indented when typeset. You can change this length however you like
% with |\setlength|: do |\setlength{\sagetexindent}{6ex}| or whatever.
%
% \subsection{Pausing \ST}
% \label{sec:pausing-st-usage}
%
% Sometimes when you are writing a document, you may wish to temporarily
% turn off or pause \ST to concentrate more on your document than on the
% Sage computations, or to simply have your document typeset faster. You
% can do this with the following commands.
%
% \DescribeMacro{\sagetexpause} \DescribeMacro{\sagetexunpause} Use
% these macros to ``pause'' and ``unpause'' \ST. After issuing this
% macro, \ST will simply skip over the corresponding calculations.
% Anywhere a |\sage| macro is used while paused, you will simply see
% \sagetexpause ``\sage{dummy call to sage to illustrate
% pausing}'', and anywhere a |\sageplot| macro is used, you will see:\\
%
% \noindent
% \sageplot{dummy call to sageplot to illustrate pausing}
% \sagetexunpause\\
%
% \noindent Anything in the verbatim-like environments of
% \autoref{sec:codeblockenv} will be typeset or not as usual, but none
% of the Sage code will be executed.
%
% Obviously, you use |\sagetexunpause| to unpause \ST and return to the
% usual state of affairs. Both commands are idempotent; issuing them
% twice or more in a row is the same as issuing them once. This means
% you don't need to precisely match pause and unpause commands: once
% paused, \ST stays paused until it sees |\sagetexunpause| and
% vice versa.
%
% \section{Other notes}
%
% Here are some other notes on using \ST.
%
% \subsection{Using Beamer}
%
% The \textsc{beamer} package does not play nicely with verbatim-like
% environments unless you ask it to. To use code block environments in a
% \textsc{beamer} presentation, do:
% \begin{quote}
%  |\begin{frame}[fragile]|\\
%  |\begin{sageblock}|\\
%  |# sage stuff|\\
%  |# more stuff \end{sageblock}|\\
%  |\end{frame}|\\
% \end{quote}
% For some reason, \textsc{beamer} inserts an extra line break at the
% end of the environment; if you put the |\end{sageblock}| on the same
% line as the last line of your code, it works properly. See section
% 12.9, ``Verbatim and Fragile Text'', in the \textsc{beamer} manual.
%
% Thanks to Franco Saliola for reporting this.
%
% \subsection{Using the \texttt{rccol} package}
%
% If you are trying to use the |\sage| macro inside a table when using
% the
% \href{http://www.ctan.org/tex-archive/macros/latex/contrib/rccol/}{\texttt{rccol}}
% package, you need to use an extra pair of braces or typesetting will
% fail. That is, you need to do something like this:
% \begin{quote}
%   | abc & {\sage{foo.n()}} & {\sage{bar}} \\|
% \end{quote}
% with each ``|\sage{}|'' enclosed in an extra |{}|. Thanks to Sette
% Diop for reporting this.
%
% \subsection{Plotting from Mathematica, Maple, etc.}
%
% Sage can use Mathematica, Maple, and friends and can tell them to do
% plotting, but since it cannot get those plots into a Sage graphics
% object, you cannot use |\sageplot| to use such graphics. You'll need
% to use the method described in ``But that's not good enough!''
% (\autoref{sec:notgoodenough}) with some additional bits to get the
% directory right---otherwise your file will get saved to someplace in a
% hidden directory.
%
% For Mathematica, you can do something like this inside a |sagesilent|
% or |sageblock| environment:
% \begin{quote}
%   |mathematica('myplot = commands to make your plot')|\\
%   |mathematica('Export["%s/graphicsfile.eps", myplot]' % os.getcwd())|
% \end{quote}
% then put |\includegraphics[opts]{graphicsfile}| in your file.
%
% For Maple, you'll need something like
% \begin{quote}
%   |maple('plotsetup(ps, plotoutput=`%s/graphicsfile.eps`, \|\\
%   |  plotoptions=`whatever`);' % os.getcwd())|\\
%   |maple('plot(function, x=1..whatever);')|
% \end{quote}
% and then |\includegraphics| as necessary.
%
% These interfaces, especially when plotting, can be finicky. The above
% commands are just meant to be a starting point.
%
% \subsection{Sending \ST files to others who don't use Sage}
% \label{sec:makestatic-usage}
%
% What can you do when sending a \LTX document that uses \ST to a
% colleague who doesn't use Sage?\footnote{Or who cannot use Sage, since
% currently \ST is not very useful on Windows.} The best option is to
% bring your colleague into the light and get him or her using Sage! But
% this may not be feasible, because some (most?) mathematicians are
% fiercely crotchety about their choice of computer algebra system, or
% you may be sending a paper to a journal or the arXiv, and such places
% will not run Sage just so they can typeset your paper---at least not
% until Sage is much closer to its goal of world domination.
%
% How can you send your \ST-enabled document to someone else who doesn't
% use Sage? The easiest way is to simply include with your document the
% following files:
% \begin{enumerate}
%   \item |sagetex.sty|
%   \item the generated |.sout| file
%   \item the \texttt{sage-plots-for-\meta{filename}.tex} directory and
%     its contents
% \end{enumerate}
% As long as |sagetex.sty| is available, your document can be typeset
% using any reasonable \LTX system. Since it is very common to include
% graphics files with a paper submission, this is a solution that should
% always work. (In particular, it will work with arXiv submissions.)
%
% There is another option, and that is to use the |makestatic.py| script
% included with \ST.
%
% Use of the script is quite simple. Copy it and |sagetexparse.py| to
% the directory with your document, and run
% \begin{quote}
%   |python makestatic.py inputfile [outputfile]|
% \end{quote}
% where |inputfile| is your document. (You can also set the executable
% bit of |makestatic.py| and use |./makestatic.py|.) This script needs
% the \href{http://pyparsing.wikispaces.com}{pyparsing} module to be
% installed.\footnote{If you don't have pyparsing installed, you can
%   simply copy the file
%   \texttt{\sageroot/local/lib/python/matplotlib/pyparsing.py} into
%   your directory.} You may optionally specify |outputfile|; if you do
% so, the results will be written to that file. If the file exists, it
% won't be overwritten unless you also specify the |-o| switch.
%
% You will need to run this after you've compiled your document and run
% Sage on the |.sage| file. The script reads in the |.sout| file and
% replaces all the calls to |\sage| and |\sageplot| with their plain
% \LTX equivalent, and turns the |sageblock| and |sageverbatim|
% environments into |verbatim| environments. Any |sagesilent|
% environment is turned into a |comment| environment. The resulting
% document should compile to something identical, or very nearly so, to
% the original file.
%
% One large limitation of this script is that it can't change anything
% while \ST is paused, since Sage doesn't compute anything for such
% parts of your document. It also doesn't check to see if pause and
% unpause commands are inside comments or verbatim environments. If
% you're going to use |makestatic.py|, just remove all pause/unpause
% statements.
%
% The parsing that |makestatic.py| does is pretty good, but not perfect.
% Right now it doesn't support having a comma-separated list of
% packages, so you can't have |\usepackage{sagetex, foo}|. You need to
% have just |\usepackage{sagetex}|. (Along with package options; those
% are handled correctly.) If you find other parsing errors, please let
% me know.
%
% \subsection{Extracting the Sage code from a document}
% \label{sec:extractsagecode}
%
% This next script is probably not so useful, but having done the above,
% this was pretty easy. The |extractsagecode.py| script does the
% opposite of |makestatic.py|, in some sense: given a document, it
% extracts all the Sage code and removes all the \LTX.
%
% Its usage is the same as |makestatic.py|.
%
% Note that the resulting file will almost certainly \emph{not} be a
% runnable Sage script, since there might be \LTX commands in it, the
% indentation may not be correct, and the plot options just get written
% verbatim to the file. Nevertheless, it might be useful if you just
% want to look at the Sage code in a file.
%
% \section{Using \ST without Sage installed}
% \label{sec:remote-sagetex}
%
% You may want to edit and typeset a \ST-ified file on a computer that
% doesn't have Sage installed. How can you do that? We need to somehow
% run Sage on the |.sage| file. The included script
% \texttt{remote-sagetex.py} takes advantage of Sage's network
% transparency and will use a remote server to do all the computations.
% Anywhere in this manual where you are told to ``run Sage'', instead of
% actually running Sage, you can run
% \begin{center}
%   \texttt{python remote-sagetex.py filename.sage}
% \end{center}
% The script will ask you for a server, username, and password, then
% process all your code and write a |.sout| file and graphics files
% exactly as if you had used a local copy of Sage to process the |.sage|
% script. (With some minor limitations and differences; see below.)
%
% One important point: \emph{the script requires Python 2.6}. It will
% not work with earlier versions. (It will work with Python 3.0 or later
% with some trivial changes.)
%
% You can provide the server, username and password with the
% command-line switches |--server|, |--username|, and |--password|, or
% you can put that information into a file and use the |--file| switch
% to specify that file. The format of the file must be like the
% following:
% \begin{verbatim}
    # hash mark at beginning of line marks a comment
    server = "http://example.com:1234"
    username = 'my_user_name'
    password = 's33krit'\end{verbatim}
% As you can see, it's really just like assigning a string to a variable
% in Python. You can use single or double quotes and use hash marks to
% start comments. You can't have comments on the same line as an
% assignment, though. You can omit any of those pieces of information
% information; the script will ask for anything it needs to know.
% Information provided as a command line switch takes precedence over
% anything found in the file.
%
% You can keep this file separate from your \LTX documents in a secure
% location; for example, on a USB thumb drive or in an automatically
% encrypted directory (like |~/Private| in Ubuntu). This makes it much
% harder to accidentally upload your private login information to the
% arXiv, put it on a website, send it to a colleague, or otherwise make
% your private information public.
%
% \subsection{Limitations of \texttt{remote-sagetex.py}}
% \label{sec:remote-sagetex-limitations}
%
% The |remote-sagetex.py| script has several limitations. It completely
% ignores the |epstopdf| and |imagemagick| flags. The |epstopdf| flag is
% not a big deal, since it was originally introduced to work around a
% matplotlib bug which has since been fixed. Not having |imagemagick|
% support means that you cannot automatically convert 3D graphics to eps
% format; using |pdflatex| to make PDFs works around this issue.
%
%
% \subsection{Other caveats}
% \label{sec:remote-sagetex-caveats}
%
% Right now, the ``simple server API'' that |remote-sagetex.py| uses is
% not terribly robust, and if you interrupt the script, it's possible to
% leave an idle session running on the server. If many idle sessions
% accumulate on the server, it can use up a lot of memory and cause the
% server to be slow, unresponsive, or maybe even crash. For now, I
% recommend that you only run the script manually. It's probably best to
% not configure your \TeX{} editing environment to automatically run
% |remote-sagetex.py| whenever you typeset your document, at least not
% without showing you the output or alerting you about errors.
%


% \iffalse
% Local Variables:
% mode: doctex
% TeX-master: t
% End:
% \fi
