% \iffalse meta-comment
%
% Copyright (C) 2008 by Dan Drake <ddrake@member.ams.org>
% -------------------------------------------------------
%
% See the "Copying and licenses" section at the end of this file for the
% terms under which this source code and documentation may be modified
% and distributed.
%
% This package is not licensed under the LPPL, but it seems reasonable
% to say:
%
%   This work has the LPPL maintenance status `maintained'.
%   
%   The Current Maintainer of this work is Dan Drake.
%
%   This work consists of the files sagetexpackage.dtx,
%   sagetexpackage.ins, example.tex, and the derived files sagetex.sty,
%   sagetex.py, sagetexparse.py, makestatic.py, and extractsagecode.py.
% 
% \fi
%
% \iffalse
%<*driver>
\ProvidesFile{sagetexpackage.dtx}
%</driver>
%<latex>\NeedsTeXFormat{LaTeX2e}
%<latex>\ProvidesPackage{sagetex}
%<*latex>
  [2008/03/12 v1.4 embedding Sage into LaTeX documents]
%</latex>
%<*driver>
\documentclass{ltxdoc}
\usepackage{sagetex}
\usepackage{xspace}
\usepackage{tikz}
\usepackage{hyperref}
\EnableCrossrefs         
\CodelineIndex
\RecordChanges
\begin{document}
  \DocInput{sagetexpackage.dtx}
  \PrintChanges
  \PrintIndex
\end{document}
%</driver>
% \fi
%
% \CheckSum{0}
%
% \CharacterTable
%  {Upper-case    \A\B\C\D\E\F\G\H\I\J\K\L\M\N\O\P\Q\R\S\T\U\V\W\X\Y\Z
%   Lower-case    \a\b\c\d\e\f\g\h\i\j\k\l\m\n\o\p\q\r\s\t\u\v\w\x\y\z
%   Digits        \0\1\2\3\4\5\6\7\8\9
%   Exclamation   \!     Double quote  \"     Hash (number) \#
%   Dollar        \$     Percent       \%     Ampersand     \&
%   Acute accent  \'     Left paren    \(     Right paren   \)
%   Asterisk      \*     Plus          \+     Comma         \,
%   Minus         \-     Point         \.     Solidus       \/
%   Colon         \:     Semicolon     \;     Less than     \<
%   Equals        \=     Greater than  \>     Question mark \?
%   Commercial at \@     Left bracket  \[     Backslash     \\
%   Right bracket \]     Circumflex    \^     Underscore    \_
%   Grave accent  \`     Left brace    \{     Vertical bar  \|
%   Right brace   \}     Tilde         \~}
%
% \changes{v1.0}{2008/03/03}{Initial version}
% \changes{v1.1}{2008/03/05}{Wrapped user-provided Sage code in
% try/except clauses; plotting now has optional format argument.}
% \changes{v1.2}{2008/03/07}{Imagemagick option; better documentation}
% \changes{v1.3.1}{2008/03/10}{Internal variables renamed; fixed typos}
% \changes{v1.4}{2008/03/11}{MD5 fix, percent sign macro, CTAN upload}
% \changes{v2.0}{2008/12/16}{External Python scripts for parsing
% SageTeX-ified documents}
%
% \GetFileInfo{sagetexpackage.dtx}
%
% \DoNotIndex{\newcommand,\newenvironment,\the}
% 
% \newcommand{\ST}{\textsf{Sage\TeX}\xspace}
% \iffalse
% so I don't have to put \ or {} after \LaTeX:
% \fi
% \newcommand{\LTX}{\LaTeX\xspace}
%
% \newcommand{\TikZ}{Ti\emph{k}Z\xspace}
%
% \tikzstyle{box}=[draw, shape=rectangle, thick]
%
% \title{The \ST{} package\thanks{This document
%   corresponds to \textsf{sagetex}~\fileversion, dated \filedate.}}
% \author{Dan Drake (\texttt{ddrake@member.ams.org}) and others}
%
% \maketitle
%
% \section{Introduction}
%
% Why should the Haskell and R folks have all the fun?
% \href{http://www.haskell.org/haskellwiki/Literate_programming}{Literate
% Haskell} is a popular way to mix Haskell source code and \LTX
% documents. (Well, actually any kind of text or document, but here
% we're concerned only with \LTX.) You can even embed Haskell code in
% your document that writes part of your document for you. Similarly,
% the R statistical computing environment includes
% \href{http://tug.org/pracjourn/2008-1/zahn/}{Sweave}, which lets you
% do the same thing with R code and \LTX.
%
% The \ST package allows you to do (roughly) the same thing with the
% Sage mathematics software suite (see \url{http://sagemath.org}) and
% \LTX. (If you know how to write literate Haskell: the |\eval| command
% corresponds to |\sage|, and the |code| environment to the |sageblock|
% environment.) As a simple example, imagine in your document you are
% writing about how to count license plates with three letters and three
% digits. With this package, you can write something like this:
% \begin{quote}
%  |There are $26$ choices for each letter, and $10$ choices for|\\
%  |each digit, for a total of $26^3*10^3 = \sage{26^3*10^3}$|\\
%  |license plates.|
% \end{quote}
% and it will produce
% \begin{quote}
%   There are $26$ choices for each letter, and $10$ choices for each
%   digit, for a total of $\sage{26^3 * 10^3}$ license plates.
% \end{quote}
% The great thing is, you don't have to do the multiplication. Sage does
% it for you. This process mirrors one of the great aspects of \LTX:
% when writing a \LTX document, you can concentrate on the logical
% structure of the document and trust \LTX and its army of packages to
% deal with the presentation and typesetting. Similarly, with \ST, you
% can concentrate on the mathematical structure (``I need the product of
% $26^3$ and $10^3$'') and let Sage deal with the base-$10$ presentation
% of the number.
%
% A less trivial, and perhaps more useful example is plotting. You can
% include a plot of the sine curve without manually producing a plot,
% saving an EPS or PDF file, and doing the |\includegraphics| business
% with the correct filename yourself. If you write this:
% \begin{quote}
% |Here is a lovely graph of the sine curve:|
%
% |\sageplot{plot(sin(x), x, 0, 2*pi)}|
% \end{quote}
% in your \LTX file, it produces
% \begin{quote}
% Here is a lovely graph of the sine curve:
%
% \sageplot{plot(sin(x), x, 0, 2*pi)}
% \end{quote}
% Again, you need only worry about the logical/mathematical structure of
% your document (``I need a plot of the sine curve over the interval
% $[0, 2\pi]$ here''), while \ST{} takes care of the gritty details of
% producing the file and sourcing it into your document.
%
% \paragraph{But \texttt{\bslash sageplot} isn't magic} I just tried to
% convince you that \ST makes putting nice graphics into your document
% very easy; let me turn around and warn you that using graphics
% \emph{well} is not easy, and no \LTX package or Python script will
% ever make it easy. What \ST does is make it easy to \emph{use Sage} to
% create graphics; it doesn't magically make your graphics good,
% appropriate, or useful. (For instance, look at the sine plot above---I
% would say that a truly lovely plot of the sine curve would not mark
% integer points on the $x$-axis, but rather $\pi/2$, $\pi$, $3\pi/2$,
% and $2\pi$.)
%
% Till Tantau has some good commentary on the use of graphics in
% \href{http://www.ctan.org/tex-archive/graphics/pgf/}{section 6 of the
% \textsc{pgf} manual}. You should always give careful thought and
% attention to creating graphics for your document; I have in mind that
% a good workflow for using \ST for plotting is something like this:
%
% \begin{enumerate}
%   \item Figure out what sort of graphic you need to communicate your
%   ideas or information.
%   \item Fiddle around in Sage until you get a graphics object and set
%   of options that produce the graphic you need.
%   \item Copy those commands and options into \ST commands in your
%   \LTX document.
% \end{enumerate}
%
% The \ST{} package's plotting capabilities don't help you find those
% Sage commands to make your lovely plot, but they do eliminate the need
% to muck around with saving the result to a file, remembering the
% filename, including it into your document, and so on. In
% \autoref{s:usage}, we will see what what we can do with \ST.
%
%
% \section{Installation}
%
% The simplest way to ``install'' \ST is to copy the files |sagetex.sty|
% and |sagetex.py| into the same directory as your document. This will
% always work, as \LTX and Python search the current directory for
% files. It is also convenient for zipping up a directory to send to a
% colleague who is not yet enlightened enough to be using \ST.
%
% Rather than make lots of copies of those files, you can keep them in
% one place and update the TEXINPUTS and PYTHONPATH environment
% variables appropriately.
%
% Perhaps the best solution is to put the files into a directory
% searched by \TeX{} and friends, and then edit the |sagetex.sty| file
% so that the |.sage| files we generate update Python's path
% appropriately---look for ``Python path'' in |sagetex.sty|. This is
% suitable for a system-wide installation, or if you are the kind of
% person who keeps a |texmf| tree in your home directory.
%
%
% \section{Usage} \label{s:usage}
% 
% Let's begin with a rough description of how \ST works. Naturally the
% very first step is to put |\usepackage{sagetex}| in the preamble of
% your document. When you use macros from this package and run \LTX on
% your file, along with the usual zoo of auxiliary files, a |.sage| file
% is written with the same basename as your document. This is a Sage
% source file that uses the Python module from this package and when you
% run Sage on that file, it will produce a |.sout| file. That file
% contains \LTX code that, when you run \LTX on your source file again,
% will pull in all the results of Sage's computation.
%
% All you really need to know is that to typeset your document, you need
% to run \LTX, then run Sage, then run \LTX again. 
%
% Also keep in mind that everything you send to Sage is done within one
% Sage session. This means you can define variables and reuse them
% throughout your \LTX document; if you tell Sage that |foo| is
% $12$, then anytime afterwards you can use |foo| in your Sage code and
% Sage will remember that it's $12$---just like in a regular Sage
% session.
%
% Now that you know that, let's describe what macros \ST provides and
% how to use them. If you are the sort of person who can't be bothered
% to read documentation until something goes wrong, you can also just
% look through the |example.tex| file included with this
% package.\footnote{Then again, if you're such a person, you're probably
% not reading this, and are already fiddling with
% \texttt{example.tex}\dots}\\
%
% \noindent\fbox{\parbox{.97\textwidth}{\textbf{WARNING!} When you run
% \LTX on a file |foo.tex|, the file |foo.sage| is created---and will be
% \emph{automatically overwritten} if it already exists. If you keep
% Sage scripts in the same directory as your \ST-ified \LTX documents,
% use a different file name!}}
%
% \paragraph{The final option} On a similar note, \ST, like many \LTX
% packages, accepts the |final| option. When passed this option, either
% directly in the |\usepackage| line, or from the |\documentclass| line,
% \ST will not write a |.sage| file. It will try to read in the |.sout|
% file so that the \ST macros can pull in their results. However, this
% will not allow you to have an independent Sage script with the same
% basename as your document, since to get the |.sout| file, you need the
% |.sage| file.
%
% \subsection{Inline Sage}
%
% \DescribeMacro{\sage}
% \fbox{\texttt{\bslash sage}\marg{Sage code}}
%
% \noindent takes whatever Sage code you give it, runs Sage's |latex|
% function on it, and puts the result into your document. 
%
% For example, if you do |\sage{matrix([[1, 2], [3,4]])^2}|, then that
% macro will get replaced by
% \begin{quote}
% |\left(\begin{array}{rr}|\\
% |7 & 10 \\|\\
% |15 & 22|\\
% |\end{array}\right)|
% \end{quote}
% in your document---that \LTX code is exactly exactly what you get
% from doing
% \begin{center}
% |latex(matrix([[1, 2], [3,4]])^2)|
% \end{center}
% in Sage.
%
% Note that since \LTX will do macro expansion on whatever you give
% to |\sage|, you can mix \LTX variables and Sage variables! If
% you have defined the Sage variable |foo| to be $12$ (using, say, the
% |sageblock| environment), then you can do something like this:
% \begin{quote}
% |The prime factorization of the current page plus foo is|\\
% |$\sage{factor(foo + \thepage)}$|.
% \end{quote}
% Here, I'll do just that right now: the prime factorization of the
% current page plus $12$ is $\sage{factor(\thepage + 12)}$.
%
% The |\sage| command doesn't automatically use math mode for its
% output, so be sure to use dollar signs or a displayed math environment
% as appropriate.\\
%
% \DescribeMacro{\percent} If you are doing modular arithmetic or string
% formatting and need a percent sign in a call to |\sage| (or
% |\sageplot|), you can use |\percent|. Using a bare percent sign won't
% work because \LTX will think you're starting a comment and get
% confused; prefixing the percent sign with a backslash won't work
% because then ``|\%|'' will be written to the |.sage| file and Sage
% will get confused. The |\percent| macro makes everyone happy.
%
% Note that using |\percent| inside the verbatim-like environments
% described in \autoref{s:codeblockenv} isn't necessary; a literal
% ``\percent'' inside such an environment will get written, uh, verbatim
% to the |.sage| file.
%
% \subsection{Graphics and plotting}
%
% \noindent \DescribeMacro{\sageplot}
% \fbox{\texttt{\bslash sageplot}\oarg{ltx opts}\oarg{fmt}\{\meta{graphics
% obj}, \meta{keyword args}\}}
%
% \noindent plots the given Sage graphics object and runs an
% |\includegraphics| command to put it into your document. It does not
% have to actually be a plot of a function; it can be any Sage graphics
% object. The options are described in \autoref{t:sageplotopts}.
%
% \begin{table}[h]
%   \centering
%   \begin{tabular}{l p{8cm}}
%   Option & Description \\
%   \hline
%   \meta{ltx options} & Any text here is passed directly into the
%   optional arguments (between the square brackets) of an
%   |\includegraphics| command. If not specified,
%   ``|width=.75\textwidth|'' will be used.\\
%   \meta{fmt} & You can optionally specify a file extension here; Sage
%   will then try to save the graphics object to a file with extension
%   \emph{fmt}. If not specified, \ST\ will save to EPS and PDF files.\\
%   \meta{graphics obj} & A Sage object on which you can call |.save()|
%   with a graphics filename.\\
%   \meta{keyword args} & Any keyword arguments you put here will
%   all be put into the call to |.save()|.
%   \end{tabular}
%   \caption{Explanation of options for the \texttt{\bslash sageplot}
%   command.}
%   \label{t:sageplotopts}
% \end{table}
%
% This setup allows you to control both the Sage side of things, and the
% \LTX side. For instance, the command
% \begin{quote}
% |\sageplot[angle=30, width=5cm]{plot(sin(x), 0, pi), axes=False,|\\
% |chocolate=True}|
% \end{quote}
% will run the following command in Sage:
% \begin{quote}
% |sage: plot(sin(x), 0, pi).save(filename=autogen, axes=False,|\\
% |chocolate=True)|
% \end{quote}
% Then, in your \LTX file, the following command will be issued
% automatically:
% \begin{center}
% |\includegraphics[angle=30, width=5cm]{autogen}|
% \end{center}
% You can specify a file format if you like. This must be the
% \emph{second} optional argument, so you must use empty brackets if
% you're not passing anything to |\includegraphics|:
% \begin{center}
% |\sageplot[][png]{plot(sin(x), x, 0, pi)}|
% \end{center}
% The filename is automatically generated, and unless you specify a
% format, both EPS and PDF files will be generated. This allows you to
% freely switch between using, say, a DVI viewer (many of which have
% support for automatic reloading, source specials and make the writing
% process easier) and creating PDFs for posting on the web or emailing
% to colleagues.
%
% If you ask for, say, a PNG file, keep in mind that ordinary |latex|
% and DVI files have no support for PNG files; \ST detects this and will
% warn you that it cannot find a suitable file if using
% |latex|.\footnote{We use a typewriter font here to indicate the
% executables which produce DVI and PDF files, respectively, as
% opposed to ``\LTX'' which refers to the entire typesetting system.}
% If you use |pdflatex|, there will be no problems because PDF files can
% include PNG graphics.
%
% When \ST cannot find a graphics file, it inserts this into your
% document:
%
% \begin{center}
%   \framebox[2cm]{\rule[-1cm]{0cm}{2cm}\textbf{??}}
% \end{center}
% 
% \noindent That's supposed to resemble the image-not-found graphics
% used by web browsers and use the traditional ``\textbf{??}'' that \LTX
% uses to indicate missing references.
%
% You needn't worry about the filenames; they are automatically
% generated and will be put into the directory
% |sage-plots-for-filename.tex|. You can safely delete that directory
% anytime; if \ST can't find the files, it will warn you to run Sage to
% regenerate them.\\
%
% \noindent\fbox{\parbox{.97\textwidth}{\textbf{WARNING!} When you run
% Sage on your |.sage| file, all files in the
% \texttt{sage-plots-for-\meta{filename}.tex} directory \emph{will be
% deleted!} Do not put any files into that directory that you do not
% want to get automatically deleted.}}
%
% \paragraph{The epstopdf option} One of the graphics-related options
% supported by \ST is |epstopdf|. This option causes \ST to use the
% |epstopdf| command to convert EPS files into PDF files. Like with the
% |imagemagick| option, it doesn't check to see if the |epstopdf|
% command exists or add options: it just runs the command. This option
% was motivated by a bug in the matplotlib PDF backend which caused it
% to create invalid PDFs. Ideally, this option should never be
% necessary; if you do need to use it, file a bug!
%
% \subsubsection{3D plotting}
%
% Right now there is, to put it nicely, a bit of tension between the
% sort of graphics formats supported by |latex| and |pdflatex|, and the
% graphics formats supported by Sage's 3D plotting systems. \LTX is
% happiest, and produces the best output, with EPS and PDF files, which
% are vector formats. Tachyon, Sage's 3D plotting system, produces
% bitmap formats like BMP and PNG.
%
% Because of this, when producing 3D plots with |\sageplot|, \emph{you
% must specify a file format}. The PNG format is compressed and lossless
% and is by far the best choice, so use that whenever possible. (Right
% now, it is always possible.) If you do not specify a file format, or
% specify one that Tachyon does not understand, it will produce files in
% the Targa format with an incorrect extension and \LTX (both |latex|
% and |pdflatex|) will be profoundly confused. Don't do that.
% 
% Since |latex| does not support PNGs, when using 3D plotting (and
% therefore a bitmap format like PNG), \ST will \emph{always} issue a
% warning about incompatible graphics if you use |latex|, provided
% you've processed the |.sage| file and the PNG file exists. (Running
% |pdflatex| on the same file will work, since PDF files can include PNG
% files.) 
%
% \paragraph{The imagemagick option} As a response to the above issue,
% the \ST package has an |imagemagick| option. If you specify this
% option in the preamble of your document with the usual
% ``|\usepackage[imagemagick]{sagetex}|'', then when you are compiling
% your document using |latex|, any |\sageplot| command which requests a
% non-default format will cause the \ST Python script to convert the
% resulting file to EPS using the Imagemagick |convert| utility. It does
% this by executing ``|convert filename.EXT filename.eps|'' in a
% subshell. It doesn't add any options, check to see if the |convert|
% command exists or belongs to Imagemagick---it just runs the command.
%
% The resulting EPS files are not very high quality, but they will work.
% This option is not intended to produce good graphics, but to allow you
% to see your graphics when you use |latex| and DVI files while writing
% your document.
%
% \subsubsection{But that's not good enough!} \label{s:notgoodenough}
%
% The |\sageplot| command tries to be both flexible and easy to use, but
% if you are just not happy with it, you can always do things manually:
% inside a |sagesilent| environment (see the next section) you could do
% \begin{quote}
%   |your special commands|\\
%   |x = your graphics object|\\
%   |x.save(filename=myspecialfile.ext, options, etc)|
% \end{quote}
% and then, in your source file, do your own |\includegraphics| command.
% The \ST package gives you full access to Sage and Python and doesn't
% turn off anything in \LTX, so you can always do things manually.
%
% \subsection{Verbatim-like environments}
% \label{s:codeblockenv}
%
% The \ST package provides several environments for typesetting and
% executing blocks of Sage code.\\
%
% \DescribeEnv{sageblock} Any text between |\begin{sageblock}| and
% |\end{sageblock}| will be typeset into your file, and also written into
% the |.sage| file for execution. This means you can do something like
% this:
% \begin{quote}
% |\begin{sageblock}|\\
% |   var('x')|\\
% |   f = sin(x) - 1|\\
% |   g = log(x)|\\
% |   h = diff(f(x) * g(x), x)|\\
% |\end{sageblock}|
% \end{quote}
% and then anytime later write in your source file
% \begin{quote}
%   |We have $h(2) = \sage{h(2)}$, where $h$ is the derivative of|\\
%   |the product of $f$ and $g$.|
% \end{quote}
% and the |\sage| call will get correctly replaced by $\sage{
% diff((sin(x) - 1)*log(x), x)(1)}$. You can use any Sage or Python
% commands inside a |sageblock|; all the commands get sent directly to
% Sage.\\
%
% \iffalse meta-comment
%   Sadly, we can't use sageblock or similar environments in this file!
%   If you prefix the lines inside the environment with percent signs,
%   then those percent signs get written to your .sage file. If you
%   *don't* prefix the lines with percent signs, those lines get written
%   into the .sty or .py file. It's just too tricky to get docstrip and
%   the verbatim stuff to play nicely together. I'd have to redefine how
%   those environments work, and get them to strip off initial percents. 
% \fi
% 
% \DescribeEnv{sagesilent} This environment is like |sageblock|, but it
% does not typeset any of the code; it just writes it to the |.sage|
% file. This is useful if you have to do some setup in Sage that is not
% interesting or relevant to the document you are writing.\\
%
% \DescribeEnv{sageverbatim} This environment is the opposite of the one
% above: whatever you type will be typeset, but not written into the
% |.sage| file. This allows you to typeset psuedocode, code that will
% fail, or take too much time to execute, or whatever.\\
%
% \DescribeEnv{comment} Logically, we now need an environment that
% neither typesets nor executes your Sage code\ldots but the |verbatim|
% package, which is always loaded when using \ST, provides such an
% environment: |comment|. Another way to do this is to put stuff between
% |\iffalse| and |\fi|.\\
%
% \DescribeMacro{\sagetexindent} There is one final bit to our
% verbatim-like environments: the indentation. The \ST package defines a
% length |\sagetexindent|, which controls how much the Sage code is
% indented when typeset. You can change this length however you like
% with |\setlength|: do |\setlength{\sagetexindent}{6ex}| or whatever.
% 
%
% \section{Other notes}
%
% Here are some other notes on using \ST.
%
% \paragraph{Using Beamer} The \textsc{beamer} package does not play
% nicely with verbatim-like environments unless you ask it to. To use
% code block environments in a \textsc{beamer} presentation, do:
% \begin{quote}
%  |\begin{frame}[fragile]|\\
%  |\begin{sageblock}|\\
%  |# sage stuff|\\
%  |# more stuff \end{sageblock}|\\
%  |\end{frame}|\\
% \end{quote}
% For some reason, \textsc{beamer} inserts an extra line break at the
% end of the environment; if you put the |\end{sageblock}| on the same
% line as the last line of your code, it works properly. See section
% 12.9, ``Verbatim and Fragile Text'', in the \textsc{beamer} manual.
%
% Thanks to Franco Saliola for reporting this.
%
% \paragraph{Plotting from Mathematica, Maple, etc.} Sage can use
% Mathematica, Maple, and friends and can tell them to do plotting, but
% since it cannot get those plots into a Sage graphics object, you
% cannot use |\sageplot| to use such graphics. You'll need to use the
% method described in ``But that's not good enough!''
% (\hyperref[s:notgoodenough]{page~\pageref*{s:notgoodenough}}) with
% some additional bits to get the directory right---otherwise your file
% will get saved to someplace in a hidden directory.
%
% For Mathematica, you can do something like this inside a |sagesilent|
% or |sageblock| environment:
% \begin{quote}
%   |mathematica('myplot = commands to make your plot')|\\
%   |mathematica('Export["%s/graphicsfile.eps", myplot]' % os.getcwd())|
% \end{quote}
% then put |\includegraphics[opts]{graphicsfile}| in your file.
%
% For Maple, you'll need something like
% \begin{quote}
%   |maple('plotsetup(ps, plotoutput=`%s/graphicsfile.eps`, \|\\
%   |  plotoptions=`whatever`);' % os.getcwd())|\\
%   |maple('plot(function, x=1..whatever);')|
% \end{quote}
% and then |\includegraphics| as necessary.
%
% These interfaces, especially when plotting, can be finicky. The above
% commands are just meant to be a starting point.
%
% \subsection{Sending \ST files to others who don't use Sage}
%
% What can you do when sending a \LTX document that uses \ST to a
% colleague who doesn't use Sage?\footnote{Or who cannot use Sage, since
% currently \ST is not very useful on Windows.} The best option is to
% bring your colleague into the light and get him or her using Sage! But
% this may not be feasible, because some (most?) mathematicians are
% fiercely crotchety about their choice of computer algebra system, or
% you may be sending a paper to a journal or the arXiv, and such places
% will not run Sage just so they can typeset your paper---at least not
% until Sage is much closer to its goal of world domination.
%
% How can you send your \ST-enabled document to someone else who doesn't
% use Sage? The easiest way is to simply include with your document the
% following files:
% \begin{enumerate}
%   \item |sagetex.sty|
%   \item the generated |.sout| file
%   \item the \texttt{sage-plots-for-\meta{filename}.tex} directory and
%     its contents
% \end{enumerate}
% As long as |sagetex.sty| is available, your document can be typeset
% using any reasonable \LTX system. Since it is very common to include
% graphics files with a paper submission, this is a solution that should
% always work. (In particular, it will work with arXiv submissions.)
%
% There is another option, and that is to use the |makestatic.py| script
% included with \ST.
%
% Use of the script is quite simple. Copy it and |sagetexparse.py| to
% the directory with your document, and run
% \begin{quote}
%   |python makestatic.py inputfile [outputfile]|
% \end{quote}
% where |inputfile| is your document. (You can also set the executable
% bit of |makestatic.py| and use |./makestatic.py|.) You optionally
% specify |outputfile|; if you do so, the results will be written to
% that file. If the file exists, it won't be overwritten unless you also
% specify the |-o| switch.
%
% You will need to run this after you've compiled your document and run
% Sage on the |.sage| file. The script reads in the |.sout| file and
% replaces all the calls to |\sage| and |\sageplot| with their plain
% \LTX equivalent, and turns the |sageblock| and |sageverbatim|
% environments into |verbatim| environments. Any |sagesilent|
% environment is turned into a |comment| environment. The resulting
% document should compile to something identical, or very nearly so, to
% the original file.
%
% The parsing that |makestatic.py| does is pretty good, but not perfect.
% Right now it doesn't support having a comma-separated list of
% packages, so you can't have |\usepackage{sagetex, foo}|. You need to
% have just |\usepackage{sagetex}|. (Along with package options; those
% are handled correctly.) If you find other parsing errors, please let
% me know.
%
% \subsection{Extracting the Sage code from a document}
%
% This next script is probably not so useful, but having done the above,
% this was pretty easy. The |extractsagecode.py| script does the
% opposite of |makestatic.py|, in some sense: given a document, it
% extracts all the Sage code and removes all the \LTX.
% 
% Its usage is the same as |makestatic.py|.
%
% Note that the resulting file will almost certainly \emph{not} be a
% runnable Sage script, since there might be \LTX commands in it, the
% indentation may not be correct, and the plot options just get written
% verbatim to the file. Nevertheless, it might be useful if you just
% want to look at the Sage code in a file.
%
%
% \StopEventually{}
%
% \section{Implementation}
%
% There are two pieces to this package: a \LTX style file, and a
% Python module. They are mutually interdependent, so it makes sense to
% document them both here.
%
% \subsection{The style file}
%
% \iffalse
% tell docstrip to put code into the .sty file
%<*latex>
% \fi
%
% All macros and counters intended for use internal to this package
% begin with ``|ST@|''.
%
% \subsubsection{Initialization}
%
% Let's begin by loading some packages. The key bits of |sageblock| and
% friends are stol---um, adapted from the |verbatim| package manual. So
% grab the |verbatim| package.
%    \begin{macrocode}
\RequirePackage{verbatim}
%    \end{macrocode}
% Unsurprisingly, the |\sageplot| command works poorly without graphics
% support.
%    \begin{macrocode}
\RequirePackage{graphicx}
%    \end{macrocode}
% The |makecmds| package gives us a |\provideenvironment| which we need,
% and we use |ifpdf| and |ifthen| in |\sageplot| so we know what kind of
% files to look for.
%    \begin{macrocode}
\RequirePackage{makecmds}
\RequirePackage{ifpdf}
\RequirePackage{ifthen}
%    \end{macrocode}
%
% Next set up the counters and the default indent.
%    \begin{macrocode}
\newcounter{ST@inline}
\newcounter{ST@plot}
\setcounter{ST@inline}{0}
\setcounter{ST@plot}{0}
\newlength{\sagetexindent}
\setlength{\sagetexindent}{5ex}
%    \end{macrocode}
% Set up the file stuff, which will get run at the beginning of the
% document, after we know what's happening with the |final| option. 
% First, we open the |.sage| file:
%    \begin{macrocode}
\AtBeginDocument{\@ifundefined{ST@final}{%
\newwrite\ST@sf%
\immediate\openout\ST@sf=\jobname.sage%
%    \end{macrocode}
%
% \begin{macro}{\ST@wsf}
% We will write a lot of stuff to that file, so make a convenient
% abbreviation, then use it to put the initial commands into the |.sage|
% file. If you know what directory |sagetex.py| will be kept in, delete
% the |\iffalse| and |\fi| lines in the generated style file
% (\emph{don't} do it in the |.dtx| file) and change the directory
% appropriately. This is useful if you have a |texmf| tree in your home
% directory or are installing \ST system-wide; then you don't need to
% copy |sagetex.py| into the same directory as your document.
%    \begin{macrocode}
\newcommand{\ST@wsf}[1]{\immediate\write\ST@sf{#1}}%
\iffalse
%% To get .sage files to automatically change the Python path to find
%% sagetex.py, delete the \iffalse and \fi lines surrounding this and
%% change the directory below to where sagetex.py can be found.
\ST@wsf{import sys}%
\ST@wsf{sys.path.insert(0, 'directory with sagetex.py')}%
\fi%
\ST@wsf{import sagetex}%
\ST@wsf{_st_ = sagetex.SageTeXProcessor('\jobname')}}%
%    \end{macrocode}
% On the other hand, if the |ST@final| flag is set, don't bother with
% any of the file stuff, and make |\ST@wsf| a no-op.
%    \begin{macrocode}
{\newcommand{\ST@wsf}[1]{\relax}}}
%    \end{macrocode}
% \end{macro}
% Now we declare our options, which mostly just set flags that we check
% at the beginning of the document, and when running the |.sage| file.
% \changes{v2.0}{2008/04/04}{Add \texttt{epstopdf} option}
% \changes{v2.0}{2008/12/16}{Add \texttt{final} option}
%
% The |final| option controls whether or not we write the |.sage| file;
% the |imagemagick| and |epstopdf| options both want to write something
% to that same file. So we put off all the actual file stuff until the
% beginning of the document---by that time, we'll have processed the
% |final| option (or not) and can check the |\ST@final| flag to see what
% to do. (We must do this because we can't specify code that runs if an
% option \emph{isn't} defined.)
%
% For |final|, we set a flag for other guys to check, and if there's no
% |.sout| file, we warn the user that something fishy is going on.
%    \begin{macrocode}
\DeclareOption{final}{%
  \newcommand{\ST@final}{x}%
  \IfFileExists{\jobname.sout}{}{\AtEndDocument{\PackageWarningNoLine{sagetex}%
  {`final' option provided, but \jobname.sout^^Jdoesn't exist! No Sage
    input will appear in your document. Remove the `final'^^Joption and
    rerun LaTeX on your document}}}}
%    \end{macrocode}
% For |imagemagick|, we set two flags: one for \LTX and one for Sage.
% It's important that we set |ST@useimagmagick| \emph{before} the
% beginning of the document, so that the graphics commands can check
% that. We do wait until the beginning of the document to do file
% writing stuff.
%    \begin{macrocode}
\DeclareOption{imagemagick}{%
  \newcommand{\ST@useimagemagick}{x}%
  \AtBeginDocument{%
  \@ifundefined{ST@final}{%
    \ST@wsf{_st_.useimagemagick = True}}{}}}
%    \end{macrocode}
% For |epstopdf|, we just set a flag for Sage. Then, process the options.
%    \begin{macrocode}
\DeclareOption{epstopdf}{%
\AtBeginDocument{%
\@ifundefined{ST@final}{%
  \ST@wsf{_st_.useepstopdf = True}}{}}}
\ProcessOptions\relax
%    \end{macrocode}
% The |\relax| is a little incantation suggested by the ``\LaTeXe{} for
% class and package writers'' manual, section 4.7.
%
% Pull in the |.sout| file if it exists, or do nothing if it doesn't. I
% suppose we could do this inside an |AtBeginDocument| but I don't see
% any particular reason to do that. It will work whenever we load it.
%    \begin{macrocode}
\InputIfFileExists{\jobname.sout}{}{}
%    \end{macrocode}
%
% \subsubsection{The \texttt{\bslash sage} macro}
%
% \begin{macro}{\sage}
% This macro combines |\ref|, |\label|, and Sage all at once. First, we
% use Sage to get a \LTX representation of whatever you give this
% function. The Sage script writes a |\newlabel| line into the |.sout|
% file, and we read the output using the |\ref| command. Usually, |\ref|
% pulls in a section or theorem number, but it will pull in arbitrary
% text just as well.
%
% The first thing it does it write its argument into the |.sage| file,
% along with a counter so we can produce a unique label. We wrap a
% try/except around the function call so that we can provide a more
% helpful error message in case something goes wrong. (In particular, we
% can tell the user which line of the |.tex| file contains the offending
% code.) We can use |^^J| to put linebreaks into the |.sage| file, but
% \LTX wants to put a space after that, which is why we don't put the
% ``except'' on its own line here in the source.
%    \begin{macrocode}
\newcommand{\sage}[1]{\ST@wsf{%
try:^^J
 _st_.inline(\theST@inline, #1)^^Jexcept:^^J
 _st_.goboom(\the\inputlineno)}%
%    \end{macrocode}
% Our use of |\newlabel| and |\ref| seems awfully clever until you load
% the |hyperref| package, which gleefully tries to hyperlink the hell
% out of everything. This is great until it hits one of our special
% |\newlabel|s and gets deeply confused. Fortunately the |hyperref|
% folks are willing to accomodate people like us, and give us a
% |NoHyper| environment.
%    \begin{macrocode}
\begin{NoHyper}\ref{@sageinline\theST@inline}\end{NoHyper}%
%    \end{macrocode}
% Now check if the label has already been defined. (The internal
% implementation of labels in \LTX involves defining a macro called
% ``|r@@labelname|''.) If it hasn't, we set a flag so that we can tell
% the user to run Sage on the |.sage| file at the end of the run.
% Finally, step the counter.
%    \begin{macrocode}
\@ifundefined{r@@sageinline\theST@inline}{\gdef\ST@rerun{x}}{}%
\stepcounter{ST@inline}}
%    \end{macrocode}
% \end{macro}
% The user might load the |hyperref| package after this one (indeed, the
% |hyperref| documentation insists that it be loaded last) or not at
% all---so when we hit the beginning of the document, provide a dummy
% |NoHyper| environment if one hasn't been defined by the |hyperref|
% package.
%    \begin{macrocode}
\AtBeginDocument{\provideenvironment{NoHyper}{}{}}
%    \end{macrocode}
%
% \begin{macro}{\percent} 
% A macro that inserts a percent sign. This is more-or-less stolen from the
% \textsf{Docstrip} manual; there they change the catcode inside a group
% and use |gdef|, but here we try to be more \LaTeX y and use
% |\newcommand|.
%    \begin{macrocode}
\catcode`\%=12
\newcommand{\percent}{%}
\catcode`\%=14
%    \end{macrocode}
% \end{macro}
%
% \subsubsection{The \texttt{\bslash sageplot} macro and friends}
%
% Plotting is rather more complicated, and requires several helper
% macros that accompany |\sageplot|.
%
% \begin{macro}{\ST@plotdir}
% A little abbreviation for the plot directory. We don't use
% |\graphicspath| because it's
% \href{http://www.tex.ac.uk/cgi-bin/texfaq2html?label=graphicspath}{
% apparently slow}---also, since we know right where our plots are
% going, no need to have \LTX looking for them.
%    \begin{macrocode}
\newcommand{\ST@plotdir}{sage-plots-for-\jobname.tex}
%    \end{macrocode}
% \end{macro}
%
% \begin{macro}{\sageplot}
% \changes{v1.3}{2008/03/08}{Iron out warnings, cool \TikZ flowchart}
% This function is similar to |\sage|. The neat thing that we take
% advantage of is that commas aren't special for arguments to \LTX
% commands, so it's easy to capture a bunch of keyword arguments that
% get passed right into a Python function.
%
% This macro has two optional arguments, which can't be defined using
% \LTX's |\newcommand|; we use Scott Pakin's brilliant
% \href{http://tug.ctan.org/tex-archive/support/newcommand/}{|newcommand|}
% package to create this macro; the options I fed to his script were
% similar to this:
%\begin{center}
% |MACRO sageplot OPT[#1={width}] OPT[#2={notprovided}] #3|
%\end{center}
% Observe that we are using a Python script to write \LTX code which
% writes Python code which writes \LTX code. Crazy!
% 
% Here's the wrapper command which does whatever magic we need to get
% two optional arguments.
%    \begin{macrocode}
\newcommand{\sageplot}[1][width=.75\textwidth]{%
  \@ifnextchar[{\ST@sageplot[#1]}{\ST@sageplot[#1][notprovided]}}
%    \end{macrocode}
% The first optional argument |#1| will get shoved right into the
% optional argument for |\includegraphics|, so the user has easy control
% over the \LTX aspects of the plotting. We define a default size of
% $3/4$ the textwidth, which seems reasonable. (Perhaps a future version
% of \ST will allow the user to specify in the package options a set of
% default options to be used throughout.) The second optional argument
% |#2| is the file format and allows us to tell what files to look for.
% It defaults to ``notprovided'', which tells the Python module to
% create EPS and PDF files. Everything in |#3| gets put into the Python
% function call, so the user can put in keyword arguments there which
% get interpreted correctly by Python.
%
% \begin{macro}{\ST@sageplot}
% \changes{v2.0}{2008/12/16}{Change to use only keyword arguments: see issue
% 2 on bitbucket tracker}
% Let's see the real code here. We write a
% couple lines to the |.sage| file, including a counter, input line
% number, and all of the mandatory argument; all this is wrapped in
% another try/except. 
%    \begin{macrocode}
\def\ST@sageplot[#1][#2]#3{\ST@wsf{%
try:^^J
 _st_.plot(\theST@plot, format='#2', _p_=#3)^^Jexcept:^^J
 _st_.goboom(\the\inputlineno)}%
%    \end{macrocode}
% Now we include the appropriate graphics file. Because the user might
% be producing DVI or PDF files, and have supplied a file format or not,
% and so on, the logic we follow is a bit complicated.
% \autoref{f:sageplottree} shows what we do; for completeness---and
% because I think drawing trees with \TikZ is really cool---we show what
% |\ST@inclgrfx| does in \autoref{f:stig}. This entire complicated
% business is intended to avoid doing an |\includegraphics| command on a
% file that doesn't exist, and to issue warnings appropriate to the
% situation.
%
% \begin{figure}
%   \centering
%   \begin{tikzpicture}
%     \tikzstyle{level 1}=[sibling distance=6cm]
%     \tikzstyle{level 2}=[sibling distance=3cm]
%     \node [box] {DVI or PDF?}
%       child {node [box] {Format provided?}
%         child {node [box] {STig EPS}
%           edge from parent node[left] {no}}
%         child {node [box] {IM option set?}
%           child {node [box, text width=3cm] {Warn that DVI + PNG = bad}
%             edge from parent node[left] {no}}
%           child {node [box] {STig EPS}
%             edge from parent node[right] {yes}}
%           edge from parent node[right] {yes}}
%         edge from parent node[left] {DVI}}
%       child {node [box] {Format provided?}
%         child {node [box] {STig PDF}
%           edge from parent node[left] {no}}
%         child {node [box] {STig \texttt{\#2}}
%           edge from parent node[right] {yes}}
%         edge from parent node[right] {PDF}};
%   \end{tikzpicture}
%   \caption{The logic tree that \texttt{\bslash sageplot} uses to
%   decide whether to run \texttt{\bslash includegraphics} or to yell at
%   the user. ``Format'' is the \texttt{\#2} argument to \texttt{\bslash
%   sageplot}, ``STig ext''
%   means a call to \texttt{\bslash ST@inclgrfx} with ``ext'' as the
%   second argument, and ``IM'' is Imagemagick.}
%   \label{f:sageplottree}
% \end{figure}
%
% If we are creating a PDF, we check to see if the user asked for a
% different format, and use that if necessary:
%    \begin{macrocode}
\ifpdf
  \ifthenelse{\equal{#2}{notprovided}}%
    {\ST@inclgrfx{#1}{pdf}}%
    {\ST@inclgrfx{#1}{#2}}%
%    \end{macrocode}
% Otherwise, we are creating a DVI file, which only supports EPS. If the
% user provided a format anyway, don't include the file (since it won't
% work) and warn the user about this. (Unless the file doesn't exist, in
% which case we do the same thing that |\ST@inclgrfx| does.)
%    \begin{macrocode}
\else
  \ifthenelse{\equal{#2}{notprovided}}%
    {\ST@inclgrfx{#1}{eps}}%
%    \end{macrocode}
% If a format is provided, we check to see if we're using the
% imagemagick option. If not, we're going to issue some sort of warning,
% depending on whether the file exists yet or not.
%    \begin{macrocode}
    {\@ifundefined{ST@useimagemagick}%
      {\IfFileExists{\ST@plotdir/plot-\theST@plot.#2}%
        {\framebox[2cm]{\rule[-1cm]{0cm}{2cm}\textbf{??}}%
         \PackageWarning{sagetex}{Graphics file
         \ST@plotdir/plot-\theST@plot.#2\space on page \thepage\space
         cannot be used with DVI output. Use pdflatex or create an EPS
         file. Plot command is}}%
        {\framebox[2cm]{\rule[-1cm]{0cm}{2cm}\textbf{??}}%
         \PackageWarning{sagetex}{Graphics file
         \ST@plotdir/plot-\theST@plot.#2\space on page \thepage\space
         does not exist. Plot command is}%
         \gdef\ST@rerun{x}}}%
%    \end{macrocode}
% Otherwise, we are using Imagemagick, so try to include an EPS file
% anyway.
%    \begin{macrocode}
    {\ST@inclgrfx{#1}{eps}}}%
\fi
%    \end{macrocode}
% Finally, step the counter and we're done.
%    \begin{macrocode}
\stepcounter{ST@plot}}
%    \end{macrocode}
% \end{macro}
% \end{macro}
%
% \begin{macro}{\ST@inclgrfx}
% This command includes the requested graphics file (|#2| is the
% extension) with the requested options (|#1|) if the file exists. Note
% that it just needs to know the extension, since we use a counter for
% the filename.
%    \begin{macrocode}
\newcommand{\ST@inclgrfx}[2]{%
  \IfFileExists{\ST@plotdir/plot-\theST@plot.#2}%
    {\includegraphics[#1]{\ST@plotdir/plot-\theST@plot.#2}}%
%    \end{macrocode}
% If the file doesn't exist, we insert a little box to indicate it
% wasn't found, issue a warning that we didn't find a graphics file,
% then set a flag that, at the end of the run, tells the user to run
% Sage again.
%    \begin{macrocode}
    {\framebox[2cm]{\rule[-1cm]{0cm}{2cm}\textbf{??}}%
     \PackageWarning{sagetex}{Graphics file
     \ST@plotdir/plot-\theST@plot.#2\space on page \thepage\space does not
     exist. Plot command is}%
     \gdef\ST@rerun{x}}}
%    \end{macrocode}
% \autoref{f:stig} makes this a bit clearer. 
% \begin{figure}
%   \centering
%   \begin{tikzpicture}
%     \tikzstyle{level 1}=[sibling distance=4cm]
%     \node [box] {Does EXT file exist?}
%       child {node [box, text width = 2.125cm] {Warn user to rerun Sage}
%         edge from parent node[left] {no}}
%       child {node [box] {Use \texttt{includegraphics}}
%         edge from parent node[right] {yes}};
%   \end{tikzpicture}
%   \caption{The logic used by the \texttt{\bslash ST@inclgrfx}
%   command.}
%   \label{f:stig}
% \end{figure}
% \end{macro}
%
% \subsubsection{Verbatim-like environments}
%
% \begin{macro}{\ST@beginsfbl}
% This is ``begin |.sage| file block'', an internal-use abbreviation
% that sets things up when we start writing a chunk of Sage code to the
% |.sage| file. It begins with some \TeX{} magic that fixes spacing,
% then puts the start of a try/except block in the |.sage| file---this
% not only allows the user to indent code without Sage/Python
% complaining about indentation, but lets us tell the user where things
% went wrong. The last bit is some magic from the |verbatim| package
% manual that makes \LTX respect line breaks.
%    \begin{macrocode}
\newcommand{\ST@beginsfbl}{%
  \@bsphack\ST@wsf{%
_st_.blockbegin()^^Jtry:}%
  \let\do\@makeother\dospecials\catcode`\^^M\active}
%    \end{macrocode}
% \end{macro}
%
% \begin{macro}{\ST@endsfbl}
% The companion to |\ST@beginsfbl|. 
%    \begin{macrocode}
\newcommand{\ST@endsfbl}{%
\ST@wsf{except:^^J
 _st_.goboom(\the\inputlineno)^^J_st_.blockend()}}
%    \end{macrocode}
% \end{macro}
%
% Now let's define the ``verbatim-like'' environments. There are four
% possibilities, corresponding to two independent choices of typesetting
% the code or not, and writing to the |.sage| file or not.
%
% \begin{environment}{sageblock}
% This environment does both: it typesets your code and puts it into the
% |.sage| file for execution by Sage.
%    \begin{macrocode}
\newenvironment{sageblock}{\ST@beginsfbl%
%    \end{macrocode}
% The space between |\ST@wsf{| and |\the| is crucial! It, along with the
% ``|try:|'', is what allows the user to indent code if they like. This
% line sends stuff to the |.sage| file.
%    \begin{macrocode}
\def\verbatim@processline{\ST@wsf{ \the\verbatim@line}%
%    \end{macrocode}
% Next, we typeset your code and start the verbatim environment.
%    \begin{macrocode}
\hspace{\sagetexindent}\the\verbatim@line\par}%
\verbatim}%
%    \end{macrocode}
% At the end of the environment, we put a chunk into the |.sage| file
% and stop the verbatim environment.
%    \begin{macrocode}
{\ST@endsfbl\endverbatim}
%    \end{macrocode}
% \end{environment}
%
% \begin{environment}{sagesilent}
% This is from the |verbatim| package manual. It's just like the above,
% except we don't typeset anything.
%    \begin{macrocode}
\newenvironment{sagesilent}{\ST@beginsfbl%
\def\verbatim@processline{\ST@wsf{ \the\verbatim@line}}%
\verbatim@start}%
{\ST@endsfbl\@esphack}
%    \end{macrocode}
% \end{environment}
%
% \begin{environment}{sageverbatim}
% The opposite of |sagesilent|. This is exactly the same as the verbatim
% environment, except that we include some indentation to be consistent
% with other typeset Sage code.
%    \begin{macrocode}
\newenvironment{sageverbatim}{%
\def\verbatim@processline{\hspace{\sagetexindent}\the\verbatim@line\par}%
\verbatim}%
{\endverbatim}
%    \end{macrocode}
% \end{environment}
%
% Logically, we now need an environment which neither typesets
% \emph{nor} writes code to the |.sage| file. The verbatim package's
% |comment| environment does that.\\
%
% Now we deal with some end-of-file cleanup.
%
% We tell the Sage script to write some information to the |.sout| file,
% then check to see if |ST@rerun| ever got defined. If not, all the
% inline formulas and plots worked, so do nothing.
%    \begin{macrocode}
\AtEndDocument{\ST@wsf{_st_.endofdoc()}%
\@ifundefined{ST@rerun}{}%
%    \end{macrocode}
% Otherwise, we issue a warning to tell the user to run Sage on the
% |.sage| file. Part of the reason we do this is that, by using |\ref|
% to pull in the inlines, \LTX will complain about undefined references
% if you haven't run the Sage script---and for many \LTX users, myself
% included, the warning ``there were undefined references'' is a signal
% to run \LTX again. But to fix these particular undefined references,
% you need to run \emph{Sage}. We also suppressed file-not-found errors
% for graphics files, and need to tell the user what to do about that.
%
% At any rate, we tell the user to run Sage if it's necessary.
%    \begin{macrocode}
{\PackageWarningNoLine{sagetex}{There were undefined Sage formulas
and/or plots.^^JRun Sage on \jobname.sage, and then run
LaTeX on \jobname.tex again}}}
%    \end{macrocode}
%
%
% \subsection{The Python module}
%
% \iffalse
% Hey, docstrip! Stop putting code into the .sty file, and start
% putting it into the .py file.
%</latex>
%<*python>
% Thanks.
% \fi
%
% The style file writes things to the |.sage| file and reads them from
% the |.sout| file. The Python module provides functions that help
% produce the |.sout| file from the |.sage| file.
%
% \paragraph{A note on Python and \textsf{Docstrip}} There is one tiny
% potential source of confusion when documenting Python code with
% \textsf{Docstrip}: the percent sign. If you have a long line of Python
% code which includes a percent sign for string formatting and you break
% the line with a backslash and begin the next line with a percent sign,
% that line \emph{will not} be written to the output file. This is only
% a problem if you \emph{begin} the line with a (single) percent sign;
% there are no troubles otherwise.\\
%
% On to the code:
%
% The |sagetex.py| file is intended to be used as a module and doesn't
% do anything useful when called directly, so if someone does that, warn
% them. We do this right away so that we print this and exit before
% trying to import any Sage modules; that way, this error message gets
% printed whether you run the script with Sage or with Python.
%    \begin{macrocode}
import sys
if __name__ == "__main__":
  print("""This file is part of the SageTeX package.
It is not meant to be called directly.

This file will be used by Sage scripts generated from a LaTeX document
using the sagetex package. Keep it somewhere where Sage and Python can
find it and it will automatically be imported.""")
  sys.exit()
%    \end{macrocode}
% Import what we need:
%    \begin{macrocode}
from sage.misc.latex import latex
import os
import os.path
import hashlib
import traceback
import subprocess
import shutil
%    \end{macrocode}
% We define a class so that it's a bit easier to carry around internal
% state. We used to just have some global variables and a bunch of
% functions, but this seems a bit nicer and easier.
%    \begin{macrocode}
class SageTeXProcessor():
  def __init__(self, jobname):
    self.progress('Processing Sage code for %s.tex...' % jobname)
    self.didinitplot    = False
    self.useimagemagick = False
    self.useepstopdf    = False
    self.plotdir        = 'sage-plots-for-' + jobname + '.tex'
    self.filename       = jobname
%    \end{macrocode}
% Open a |.sout.tmp| file and write all our output to that. Then, when
% we're done, we move that to |.sout|. The ``autogenerated'' line is
% basically the same as the lines that get put at the top of preparsed
% Sage files; we are automatically generating a file with Sage, so it
% seems reasonable to add it.
%    \begin{macrocode}
    self.souttmp        = open(self.filename + '.sout.tmp', 'w')
    s = '% This file was *autogenerated* from the file ' + \
        os.path.splitext(jobname)[0] + '.sage.\n'
    self.souttmp.write(s)
%    \end{macrocode}
%
% \begin{macro}{progress}
% This function just prints stuff. It allows us to not print a
% linebreak, so you can get ``|start...|'' (little time spent
% processing) ``|end|'' on one line.
%    \begin{macrocode}
  def progress(self, t,linebreak=True):
    if linebreak:
      print(t)
    else:
      sys.stdout.write(t)
%    \end{macrocode}
% \end{macro}
%
% \begin{macro}{initplot}
% We only want to create the plots directory if the user actually plots
% something. This function creates the directory and sets the
% |didinitplot| flag after doing so. We make a directory based on the
% \LTX file being processed so that if there are multiple |.tex| files
% in a directory, we don't overwrite plots from another file.
%    \begin{macrocode}
  def initplot(self):
    self.progress('Initializing plots directory')
%    \end{macrocode}
% We hard-code the |.tex| extension, which is fine in the overwhelming
% majority of cases, although it does cause minor confusion when
% building the documentation. If it turns out lots of people use, say, a
% |ltx| extension or whatever, We could find out the correct extension,
% but it would involve a lot of irritating mucking around---on
% |comp.text.tex|, the best solution I found for finding the file
% extension is to look through the |.log| file.
%    \begin{macrocode}
    if os.path.isdir(self.plotdir):
      shutil.rmtree(self.plotdir)
    os.mkdir(self.plotdir)
    self.didinitplot = True
%    \end{macrocode}
% \end{macro}
%
% \begin{macro}{inline}
% This function works with |\sage| from the style file to put Sage
% output into your \LTX file. Usually, when you use |\label|, it
% writes a line such as
% \begin{center}
%   |\newlabel{labelname}{{section number}{page number}}|
% \end{center}
% to the |.aux| file. When you use the |hyperref| package, there are
% more fields in the second argument, but the first two are the same.
% The |\ref| command just pulls in what's in the first field of the
% second argument, so we can hijack this mechanism for our own nefarious
% purposes. The function writes a |\newlabel| line with a label made
% from a counter and the text from running Sage on |s|.
%
% We print out the line number so if something goes wrong, the user can
% more easily track down the offending |\sage| command in the source
% file.
%
% That's a lot of explanation for a very short function:
%    \begin{macrocode}
  def inline(self, counter, s):
    self.progress('Inline formula %s' % counter)
    self.souttmp.write('\\newlabel{@sageinline' + str(counter) + '}{{' + \
                 latex(s) + '}{}{}{}{}}\n')
%    \end{macrocode}
% We are using five fields, just like |hyperref| does, because that
% works whether or not |hyperref| is loaded. Using two fields, as in
% plain \LTX, doesn't work if |hyperref| is loaded.
% \end{macro}
%
% \begin{macro}{blockbegin}
% \begin{macro}{blockend}
% This function and its companion used to write stuff to the |.sout|
% file, but now they just update the user on our progress evaluating a
% code block.
%    \begin{macrocode}
  def blockbegin(self):
    self.progress('Code block begin...', False)
  def blockend(self):
    self.progress('end')
%    \end{macrocode}
% \end{macro} 
% \end{macro} 
%
% \begin{macro}{plot}
% I hope it's obvious that this function does plotting. As mentioned in
% the |\sageplot| code, we're taking advantage of two things: first,
% that \LTX doesn't treat commas and spaces in macro arguments
% specially, and second, that Python (and Sage plotting functions) has
% nice support for keyword arguments. The |#3| argument to |\sageplot|
% becomes |_p_| and |**kwargs| below.
%    \begin{macrocode}
  def plot(self, counter, _p_, format='notprovided', **kwargs):
    if not self.didinitplot:
      self.initplot()
    self.progress('Plot %s' % counter)
%    \end{macrocode}
% If the user says nothing about file formats, we default to producing
% PDF and EPS. This allows the user to transparently switch between
% using a DVI previewer (which usually automatically updates when the
% DVI changes, and has support for source specials, which makes the
% writing process easier) and making PDFs.\footnote{Yes, there's
% \texttt{pdfsync}, but full support for that is still rare in Linux, so
% producing EPS and PDF is the best solution for now.}
%    \begin{macrocode}
    if format == 'notprovided':
      formats = ['eps', 'pdf']
    else:
      formats = [format]
    for fmt in formats:
%    \end{macrocode}
% If we're making a PDF and have been told to use |epstopdf|, do so,
% then skip the rest of the loop.
%    \begin{macrocode}
      if fmt == 'pdf' and self.useepstopdf:
        epsfile = os.path.join(self.plotdir, 'plot-%s.eps' % counter)
        self.progress('Calling epstopdf to convert plot-%s.eps to PDF' % \
            counter)
        subprocess.check_call(['epstopdf', epsfile])
        continue
      plotfilename = os.path.join(self.plotdir, 'plot-%s.%s' % (counter, fmt))
      #print('  plotting %s with args %s' % (plotfilename, kwargs))
      _p_.save(filename=plotfilename, **kwargs)
%    \end{macrocode}
% If the user provides a format \emph{and} specifies the |imagemagick|
% option, we try to convert the newly-created file into EPS format.
%    \begin{macrocode}
      if format != 'notprovided' and self.useimagemagick:
        self.progress('Calling Imagemagick to convert plot-%s.%s to EPS' % \
          (counter, format))
        self.toeps(counter, format)
%    \end{macrocode}
% \end{macro}
%
% \begin{macro}{toeps}
% This function calls the Imagmagick utility |convert| to, well, convert
% something into EPS format. This gets called when the user has
% requested the ``|imagemagick|'' option to the \ST\ style file and is
% making a graphic file with a nondefault extension.
%    \begin{macrocode}
  def toeps(self, counter, ext):
    subprocess.check_call(['convert',\
      '%s/plot-%s.%s' % (self.plotdir, counter, ext), \
      '%s/plot-%s.eps' % (self.plotdir, counter)])
%    \end{macrocode}
% We are blindly assuming that the |convert| command exists and will do
% the conversion for us; the |check_call| function raises an exception
% which, since all these calls get wrapped in try/excepts in the |.sage|
% file, should result in a reasonable error message if something strange
% happens.
% \end{macro}
%
% \begin{macro}{goboom}
% When a chunk of Sage code blows up, this function bears the bad news
% to the user. Normally in Python the traceback is good enough for this,
% but in this case, we start with a |.sage| file (which is
% autogenerated) which itself autogenerates a |.py| file---and the
% tracebacks the user sees refer to that file, whose line numbers are
% basically useless. We want to tell them where in the \LTX file things
% went bad, so we do that, give them the traceback, and exit after
% removing the |.sout.tmp| file.
%    \begin{macrocode}
  def goboom(self, line):
    print('\n**** Error in Sage code on line %s of %s.tex! Traceback\
 follows.' % (line, self.filename))
    traceback.print_exc()
    print('\n**** Running Sage on %s.sage failed! Fix %s.tex and try\
 again.' % ((self.filename,) * 2))
    self.souttmp.close()
    os.remove(self.filename + '.sout.tmp')
    sys.exit(1)
%    \end{macrocode}
% \end{macro}
% 
% \begin{macro}{endofdoc}
% When we're done processing, we have some cleanup tasks. We
% want to put the MD5 sum of the |.sage| file that produced the |.sout|
% file we're about to write into the |.sout| file, so that external
% programs that build \LTX documents can determine if they need to call Sage
% to update the |.sout| file. But there is a problem: we write line
% numbers to the |.sage| file so that we can provide useful error
% messages---but that means that adding non-\ST text to your
% source file will change the MD5 sum, and your program will think it
% needs to rerun Sage even though none of the actual \ST macros
% changed.
%
% How do we include line numbers for our error messages but still allow
% a program to discover a ``genuine'' change to the |.sage| file? 
%
% The answer is to only find the MD5 sum of \emph{part} of the |.sage|
% file. By design, the source file line numbers only appear in calls to
% |goboom|, so we will strip those lines out. Basically we are doing
% \begin{center}
% \verb+grep -v '^ _st_.goboom' filename.sage | md5sum+
% \end{center}
% (In fact, what we do below produces exactly the same sum.) 
%    \begin{macrocode}
  def endofdoc(self):
    sagef = open(self.filename + '.sage', 'r')
    m = hashlib.md5()
    for line in sagef:
      if line[0:12] != ' _st_.goboom':
        m.update(line)
    s = '%' + m.hexdigest() + '% md5sum of corresponding .sage file\
 (minus "goboom" lines)\n'
    self.souttmp.write(s)
%    \end{macrocode}
% Now, we do issue warnings to run Sage on the |.sage| file and an
% external program might look for those to detect the need to rerun
% Sage, but those warnings do not quite capture all situations. (If
% you've already produced the |.sout| file and change a |\sage| call, no
% warning will be issued since all the |\ref|s find a |\newlabel|.)
% Anyway, I think it's easier to grab an MD5 sum out of the end of the
% file than parse the output from running |latex| on your file. (The
% regular expression |^%[0-9a-f]{32}%| will find the MD5 sum.)
%
% Now we are done with the |.sout.tmp| file. Close it, rename it, and
% tell the user we're done.
%    \begin{macrocode}
    self.souttmp.close()
    os.rename(self.filename + '.sout.tmp', self.filename + '.sout')
    self.progress('Sage processing complete. Run LaTeX on %s.tex again.' %\
             self.filename)
%    \end{macrocode}
% \end{macro}
%
% \iffalse
%</python>
%<*staticscript>
% \fi
%
% \section{Included Python scripts}
%
% Here we describe the Python code for |makestatic.py|, which removes
% \ST commands to produce a ``static'' file, and |extractsagecode.py|,
% which extracts all the Sage code from a |.tex| file.
%
% \subsection{makestatic.py}
%
% First, |makestatic.py| script. It's about the most basic, generic
% Python script taking command-line arguments that you'll find. The
% |#!/usr/bin/env python| line is provided for us by the |.ins| file's
% preamble, so we don't put it here.
%    \begin{macrocode}
import sys
import time
import getopt
import os.path
from sagetexparse import DeSageTex

def usage():
  print("""Usage: %s [-h|--help] [-o|--overwrite] inputfile [outputfile]

Removes SageTeX macros from `inputfile' and replaces them with the
Sage-computed results to make a "static" file. You'll need to have run
Sage on `inputfile' already.

`inputfile' can include the .tex extension or not. If you provide
`outputfile', the results will be written to a file of that name.
Specify `-o' or `--overwrite' to overwrite the file if it exists.

See the SageTeX documentation for more details.""" % sys.argv[0])

try:
  opts, args = getopt.getopt(sys.argv[1:], 'ho', ['help', 'overwrite'])
except getopt.GetoptError, err:
  print str(err)
  usage()
  sys.exit(2)

overwrite = False
for o, a in opts:
  if o in ('-h', '--help'):
    usage()
    sys.exit()
  elif o in ('-o', '--overwrite'):
    overwrite = True

if len(args) == 0 or len(args) > 2:
  print('Error: wrong number of arguments. Make sure to specify options first.\n')
  usage()
  sys.exit(2)

if len(args) == 2 and (os.path.exists(args[1]) and not overwrite):
  print('Error: %s exists and overwrite option not specified.' % args[1])
  sys.exit(1)

src, ext = os.path.splitext(args[0])
%    \end{macrocode}
% All the real work gets done in the line below. Sorry it's not more
% exciting-looking.
%    \begin{macrocode}
desagetexed = DeSageTex(src)
%    \end{macrocode}
% This part is cool: we need double percent signs at the beginning of
% the line because Python needs them (so they get turned into single
% percent signs) \emph{and} because \textsf{Docstrip} needs them (so the
% line gets passed into the generated file). It's perfect!
%    \begin{macrocode}
header = """\
%% SageTeX commands have been automatically removed from this file and
%% replaced with plain LaTeX. Processed %s.

""" % time.strftime('%a %d %b %Y %H:%M:%S', time.localtime())

if len(args) == 2:
  dest = open(args[1], 'w')
else:
  dest = sys.stdout

dest.write(header)
dest.write(desagetexed.result)
%    \end{macrocode}
%
% \iffalse
%</staticscript>
%<*extractscript>
% \fi
%
% \subsection{extractsagecode.py}
%
% Same idea as |makestatic.py|, except this does basically the opposite
% thing.
%    \begin{macrocode}
import sys
import time
import getopt
import os.path
from sagetexparse import SageCodeExtractor

def usage():
  print("""Usage: %s [-h|--help] [-o|--overwrite] inputfile [outputfile]

Extracts Sage code from `inputfile'.

`inputfile' can include the .tex extension or not. If you provide
`outputfile', the results will be written to a file of that name,
otherwise the result will be printed to stdout.

Specify `-o' or `--overwrite' to overwrite the file if it exists.

See the SageTeX documentation for more details.""" % sys.argv[0])

try:
  opts, args = getopt.getopt(sys.argv[1:], 'ho', ['help', 'overwrite'])
except getopt.GetoptError, err:
  print str(err)
  usage()
  sys.exit(2)

overwrite = False
for o, a in opts:
  if o in ('-h', '--help'):
    usage()
    sys.exit()
  elif o in ('-o', '--overwrite'):
    overwrite = True

if len(args) == 0 or len(args) > 2:
  print('Error: wrong number of arguments. Make sure to specify options first.\n')
  usage()
  sys.exit(2)

if len(args) == 2 and (os.path.exists(args[1]) and not overwrite):
  print('Error: %s exists and overwrite option not specified.' % args[1])
  sys.exit(1)

src, ext = os.path.splitext(args[0])
sagecode = SageCodeExtractor(src)
header = """\
%% This file contains Sage code extracted from %s.%s.
%% Processed %s.

""" % (src, ext, time.strftime('%a %d %b %Y %H:%M:%S', time.localtime()))

if len(args) == 2:
  dest = open(args[1], 'w')
else:
  dest = sys.stdout

dest.write(header)
dest.write(sagecode.result)
%    \end{macrocode}
%
% \iffalse
%</extractscript>
%<*parsermod>
% \fi
%
% \subsection{The parser module}
% 
% Here's the module that does the actual parsing and replacing. It's
% really quite simple, thanks to the awesome
% \href{http://pyparsing.wikispaces.com}{Pyparsing module}. The parsing
% code below is nearly self-documenting! Compare that to fancy regular
% expressions, which sometimes look like someone sneezed punctuation all
% over the screen.
%    \begin{macrocode}
from pyparsing import *
%    \end{macrocode}
% First, we define this very helpful parser: it finds the matching
% bracket, and doesn't parse any of the intervening text. It's basically
% like hitting the percent sign in Vim. This is useful for parsing \LTX
% stuff, when you want to just grab everything enclosed by matching
% brackets.
%    \begin{macrocode}
def skipToMatching(opener, closer):
  nest = nestedExpr(opener, closer)
  nest.setParseAction(lambda l, s, t: l[s:getTokensEndLoc()])
  return nest

curlybrackets = skipToMatching('{', '}')
squarebrackets = skipToMatching('[', ']')
%    \end{macrocode}
% Next, parser for |\sage| and |\sageplot| calls:
%    \begin{macrocode}
sagemacroparser = '\\sage' + curlybrackets('code')
sageplotparser = ('\\sageplot'
                 + Optional(squarebrackets)('opts') 
                 + Optional(squarebrackets)('format')
                 + curlybrackets('code'))
%    \end{macrocode}
%
% With those defined, let's move on to our classes.
%
% \begin{macro}{SoutParser}
% Here's the parser for the generated |.sout| file. The code below does
% all the parsing of the |.sout| file and puts the results into a
% dictionary. Notice that it's on the order of 10 lines of code---hooray
% for Pyparsing!
%    \begin{macrocode}
class SoutParser():
  def __init__(self, fn):
    self.label = {}
%    \end{macrocode}
% A label line looks like
% \begin{quote}
%  |\newlabel{@sageinline|\meta{integer}|}{|\marg{bunch of \LTX code}|{}{}{}{}}|
% \end{quote}
% which makes the parser definition below pretty obvious. We assign some
% names to the interesting bits so the |newlabel| method can make the
% \meta{integer} and \meta{bunch of \LTX code} into the keys and values
% of a dictionary. The |DeSageTeX| class then uses that dictionary to
% replace bits in the |.tex| file with their Sage-computed results.
%    \begin{macrocode}
    parselabel = ('\\newlabel{@sageinline'
                 + Word(nums)('num')
                 + '}{'
                 + curlybrackets('result')
                 + '{}{}{}{}}')
%    \end{macrocode}
% We tell it to ignore comments, and hook up the dictionary-making
% method. 
%    \begin{macrocode}
    parselabel.ignore('%' + restOfLine)
    parselabel.setParseAction(self.newlabel)
%    \end{macrocode}
% A |.sout| file consists of one or more such lines. Now go parse the
% file we were given.
%    \begin{macrocode}
    OneOrMore(parselabel).parseFile(fn)
%    \end{macrocode}
% Pyparser's parse actions get called with three arguments: the string
% that matched, the location of the beginning, and the resulting parse
% object. Here we just add a new key-value pair to the dictionary,
% remembering to strip off the enclosing brackets from the ``result''
% bit.
%    \begin{macrocode}
  def newlabel(self, s, l, t):
    self.label[int(t.num)] = t.result[1:-1]
%    \end{macrocode}
% \end{macro}
%
% \begin{macro}{DeSageTeX}
% Now we define a parser for \LTX files that use \ST commands. We assume
% that the provided |fn| is just a basename.
%    \begin{macrocode}
class DeSageTex():
  def __init__(self, fn):
    self.sagen = 0
    self.plotn = 0
    self.fn = fn
    self.sout = SoutParser(fn + '.sout')
%    \end{macrocode}
% Parse |\sage| macros. We just need to pull in the result from the
% |.sout| file and increment the counter---that's what |self.sage| does.
%    \begin{macrocode}
    smacro = sagemacroparser
    smacro.setParseAction(self.sage)
%    \end{macrocode}
% Parse the |\usepackage{sagetex}| line. Right now we don't support
% comma-separated lists of packages.
%    \begin{macrocode}
    usepackage = ('\\usepackage'
                 + Optional(squarebrackets)
                 + '{sagetex}')
    usepackage.setParseAction(replaceWith("""\\RequirePackage{verbatim}
\\RequirePackage{graphicx}"""))
%    \end{macrocode}
% Parse |\sageplot| macros.
%    \begin{macrocode}
    splot = sageplotparser
    splot.setParseAction(self.plot)
%    \end{macrocode}
% The printed environments (|sageblock| and |sageverbatim|) get turned
% into |verbatim| environments.
%    \begin{macrocode}
    beginorend = oneOf('begin end')
    blockorverb = 'sage' + oneOf('block verbatim')
    blockorverb.setParseAction(replaceWith('verbatim'))
    senv = '\\' + beginorend + '{' + blockorverb + '}'
%    \end{macrocode}
% The non-printed |sagesilent| environment gets commented out. We could
% remove all the text, but this works and makes going back to \ST
% commands (de-de-\ST{}ing?) easier.
%    \begin{macrocode}
    silent = Literal('sagesilent')
    silent.setParseAction(replaceWith('comment'))
    ssilent = '\\' + beginorend + '{' + silent + '}'
%    \end{macrocode}
% The |\sagetexindent| macro is no longer relevant, so remove it from
% the output (``suppress'', in Pyparsing terms). 
%    \begin{macrocode}
    stexindent = Suppress('\\setlength{\\sagetexindent}' + curlybrackets)
%    \end{macrocode}
% Now we define the parser that actually goes through the file. It just
% looks for any one of the above bits, while ignoring anything that
% should be ignored.
%    \begin{macrocode}
    doit = smacro | senv | ssilent | usepackage | splot | stexindent
    doit.ignore('%' + restOfLine)
    doit.ignore('\\begin{verbatim}' + SkipTo('\\end{verbatim}'))
    doit.ignore('\\begin{comment}' + SkipTo('\\end{comment}'))
%    \end{macrocode}
% We can't use the |parseFile| method, because that expects a ``complete
% grammar'' in which everything falls into some piece of the parser.
% Instead we suck in the whole file as a single string, and run
% |transformString| on it, since that will just pick out the interesting
% bits and munge them according to the above definitions.
%    \begin{macrocode}
    str = ''.join(open(fn + '.tex', 'r').readlines())
    self.result = doit.transformString(str)
%    \end{macrocode}
% That's the end of the class constructor, and it's all we need to do
% here. You access the results of parsing via the |result| string.
%
% We do have two methods to define. The first does the same thing that
% |\ref| does in your \LTX file: returns the content of the label and
% increments a counter.
%    \begin{macrocode}
  def sage(self, s, l, t):
    self.sagen += 1
    return self.sout.label[self.sagen - 1]
%    \end{macrocode}
% The second method returns the appropriate |\includegraphics| command.
% It does need to account for the default argument.
%    \begin{macrocode}
  def plot(self, s, l, t):
    self.plotn += 1
    if len(t.opts) == 0:
      opts = '[width=.75\\textwidth]'
    else:
      opts = t.opts[0]
    return ('\\includegraphics%s{sage-plots-for-%s.tex/plot-%s}' % 
      (opts, self.fn, self.plotn - 1))
%    \end{macrocode}
% \end{macro}
%
% \begin{macro}{SageCodeExtractor}
% This class does the opposite of the first: instead of removing Sage
% stuff and leaving only \LTX, this removes all the \LTX and leaves only
% Sage.
%    \begin{macrocode}
class SageCodeExtractor():
  def __init__(self, fn):
    smacro = sagemacroparser
    smacro.setParseAction(self.macroout)

    splot = sageplotparser
    splot.setParseAction(self.plotout)
%    \end{macrocode}
% Above, we used the general parsers for |\sage| and |\sageplot|. We
% have to redo the environment parsers because it seems too hard to
% define one parser object that will do both things we want: above, we
% just wanted to change the environment name, and here we want to suck
% out the code. Here, it's important that we find matching begin/end
% pairs; above it wasn't. At any rate, it's not a big deal to redo this
% parser.
%    \begin{macrocode}
    env_names = oneOf('sageblock sageverbatim sagesilent')
    senv = '\\begin{' + env_names('env') + '}' + SkipTo(
           '\\end{' + matchPreviousExpr(env_names) + '}')('code')
    senv.leaveWhitespace()
    senv.setParseAction(self.envout)

    doit = smacro | splot | senv

    str = ''.join(open(fn + '.tex', 'r').readlines())
    self.result = ''

    doit.transformString(str)

  def macroout(self, s, l, t):
    self.result += '# \\sage{} from line %s\n' % lineno(l, s)
    self.result += t.code[1:-1] + '\n\n'

  def plotout(self, s, l, t):
    self.result += '# \\sageplot{} from line %s:\n' % lineno(l, s)
    if t.format is not '':
      self.result += '# format: %s' % t.format[0][1:-1] + '\n'
    self.result += t.code[1:-1] + '\n\n'

  def envout(self, s, l, t):
    self.result += '# %s environment from line %s:' % (t.env,
      lineno(l, s))
    self.result += t.code[0] + '\n'
%    \end{macrocode}
% \end{macro}
%
% \iffalse
%</parsermod>
% \fi
%
% \section{Credits and acknowledgements}
%
% According to the original README file, this system was originally done
% by Gonzalo Tornaria and Joe Wetherell. Later Harald Schilly made some
% improvements and modifications. Almost all the examples in the
% |example.tex| file are from Harald.
%
% Dan Drake rewrote and extended the style file (there is almost zero
% original code there), made significant changes to the Python module,
% put both files into \textsf{Docstrip} format, and wrote all the
% documentation and extra Python scripts.
%
% Many thanks to Jason Grout for his numerous comments, suggestions, and
% feedback.
%
% \section{Copying and licenses}
%
% If you are unnaturally curious about the current state of the \ST
% package, you can visit \url{http://www.bitbucket.org/ddrake/sagetex/}.
% There is a Mercurial repository and other stuff there.
%
% As for the terms and conditions under which you can copy and modify
% \ST:
%
% The \emph{source code} of the \ST package may be redistributed and/or
% modified under the terms of the GNU General Public License as
% published by the Free Software Foundation, either version 2 of the
% License, or (at your option) any later version. To view a copy of this
% license, see \url{http://www.gnu.org/licenses/} or send a letter to
% the Free Software Foundation, Inc., 51 Franklin Street, Fifth Floor,
% Boston, MA 02110-1301, USA.
%
% The \emph{documentation} of the \ST package is licensed under the
% Creative Commons Attribution-Noncommercial-Share Alike 3.0 License. To
% view a copy of this license, visit
% \url{http://creativecommons.org/licenses/by-nc-sa/3.0/} or send a
% letter to Creative Commons, 171 Second Street, Suite 300, San
% Francisco, California, 94105, USA.
