% \iffalse meta-comment
%
% Copyright (C) 2008 by Dan Drake <ddrake@member.ams.org>
% -------------------------------------------------------
% 
% This program is free software: you can redistribute it and/or modify
% it under the terms of the GNU General Public License as published by
% the Free Software Foundation, either version 2 of the License, or (at
% your option) any later version.
% 
% This program is distributed in the hope that it will be useful, but
% WITHOUT ANY WARRANTY; without even the implied warranty of
% MERCHANTABILITY or FITNESS FOR A PARTICULAR PURPOSE.  See the GNU
% General Public License for more details.
%         
% You should have received a copy of the GNU General Public License
% along with this program.  If not, see <http://www.gnu.org/licenses/>
% 
% \fi
%
% \iffalse
%<*driver>
\ProvidesFile{sagetexpackage.dtx}
%</driver>
%<latex>\NeedsTeXFormat{LaTeX2e}
%<latex>\ProvidesPackage{sagetex}
%<*latex>
    [2008/03/05 v1.1 embedding Sage into LaTeX documents]
%</latex>
%
%<*driver>
\documentclass{ltxdoc}
\usepackage{sagetex}
\usepackage{hyperref}
\EnableCrossrefs         
\CodelineIndex
\RecordChanges
\begin{document}
  \DocInput{sagetexpackage.dtx}
  \PrintChanges
  \PrintIndex
\end{document}
%</driver>
% \fi
%
% \CheckSum{0}
%
% \CharacterTable
%  {Upper-case    \A\B\C\D\E\F\G\H\I\J\K\L\M\N\O\P\Q\R\S\T\U\V\W\X\Y\Z
%   Lower-case    \a\b\c\d\e\f\g\h\i\j\k\l\m\n\o\p\q\r\s\t\u\v\w\x\y\z
%   Digits        \0\1\2\3\4\5\6\7\8\9
%   Exclamation   \!     Double quote  \"     Hash (number) \#
%   Dollar        \$     Percent       \%     Ampersand     \&
%   Acute accent  \'     Left paren    \(     Right paren   \)
%   Asterisk      \*     Plus          \+     Comma         \,
%   Minus         \-     Point         \.     Solidus       \/
%   Colon         \:     Semicolon     \;     Less than     \<
%   Equals        \=     Greater than  \>     Question mark \?
%   Commercial at \@     Left bracket  \[     Backslash     \\
%   Right bracket \]     Circumflex    \^     Underscore    \_
%   Grave accent  \`     Left brace    \{     Vertical bar  \|
%   Right brace   \}     Tilde         \~}
%
%
% \changes{v1.0}{2008/03/03}{Initial version}
% \changes{v1.1}{2008/03/05}{Wrapped user-provided Sage code in
% try/except clauses; plotting now has optional format argument.}
%
% \GetFileInfo{sagetexpackage.dtx}
%
% \DoNotIndex{\newcommand,\newenvironment,\the}
% 
% \newcommand{\ST}{\textsf{sagetex}}
%
% \title{The \ST\ package\thanks{This document
%   corresponds to \textsf{sagetex}~\fileversion, dated \filedate.}}
% \author{Dan Drake (\texttt{ddrake@member.ams.org}) and others}
%
% \maketitle
%
% \section{Introduction}
%
% Why should the Haskell folks have all the fun? 
%
% \href{http://www.haskell.org/haskellwiki/Literate_programming}{Literate
% Haskell} is a popular way to mix Haskell source code and \LaTeX{}
% documents. (Well, actually any kind of text or document, but here
% we're concerned only with \LaTeX.) You can even embed Haskell code in
% your document that writes part of your document for you. 
%
% The \ST\ package allows you to do (roughly) the same with Sage and
% \LaTeX. (If you know how to write literate Haskell: the \verb|\eval|
% command corresponds to \verb|\sage|, and the \verb|code| environment
% to the \verb|sageblock| environment.) As a simple example, imagine in
% your document you are writing about how to count license plates with
% three letters and three digits. With this package, you can write
% something like this:
% \begin{quote}
%  \texttt{There are \$26\$ choices for each letter, and \$10\$ choices
%  for each digit, for a total of } \verb+$26^3*10^3 = \sage{26^3*10^3}$+
%  \texttt{license plates.}
% \end{quote}
% and it will produce
% \begin{quote}
%   There are $26$ choices for each letter, and $10$ choices for each
%   digit, for a total of $\sage{26^3 * 10^3}$ license plates.
% \end{quote}
% The great thing is, you don't have to do the multiplication. Sage does
% it for you. This process mirrors one of the great aspects of
% \LaTeX{}: when writing a \LaTeX{} document, you can concentrate on the
% logical structure of the document and trust \LaTeX{} and its army of
% packages to deal with the presentation and typesetting. Similarly,
% with \ST, you can concentrate on the mathematical
% structure (``I need the product of $26^3$ and $10^3$'') and let Sage
% deal with the base-$10$ presentation of the number.
%
% A less trivial, and perhaps more useful example is plotting. You can
% include a plot of the sine curve without manually producing a plot,
% saving an eps or pdf file, and doing the \verb+\includegraphics+
% business with the correct filename yourself. If you write this:
% \begin{quote}
% \texttt{Here is a lovely graph of the sine curve:}
%
% \verb+\sageplot{plot(sin(x), x, 0, 2*pi)}+
% \end{quote}
% in your \LaTeX{} file, it produces
% \begin{quote}
% Here is a lovely graph of the sine curve:
%
% \sageplot{plot(sin(x), x, 0, 2*pi)}
% \end{quote}
% Again, you need only worry about the logical/mathematical structure of
% your document (``I need a plot of the sine curve over the interval
% $[0, 2\pi]$ here''), while \ST\ takes care of the gritty details of
% producing the file and sourcing it into your document.
%
% \section{Usage}
% 
% Let's begin with a rough description of how \ST\ works.
% When you use macros from this package and run \LaTeX{} on your file,
% along with the usual zoo of auxiliary files, a |.sage| file is
% written. This is a Sage source file that uses the Python module from
% this package and when you run Sage on that file, it will produce a
% |.sout| file. That file contains \LaTeX{} code which, when you run
% \LaTeX{} on your source file again, will pull in all the results of
% Sage's computation.
%
% All you really need to know is that to typeset your document, you need
% to run \LaTeX, then run Sage, then run \LaTeX{} again. 
%
% Also keep in mind that everything you send to Sage is done within one
% Sage session. This means you can define variables and reuse them
% throughout your \LaTeX{} document; if you tell Sage that |foo| is
% $12$, then anytime afterwards you can use |foo| in your Sage code and
% Sage will remember that it's $12$---just like in a regular Sage
% session.
%
% Now that you know that, let's describe what macros \ST\ provides and
% how to use them.\\
%
% \iffalse meta-comment
% For some reason, getting a blackslash in a typewriter font to print
% inside an fbox is really hard. This code works; it's copied out of
% Scott Pakin's dtxtut.tex. Verbatim stuff doesn't work because it's
% fragile.
% \fi
% {\catcode`\|=0 \catcode`\\=12
% |gdef|bslash{\}}
%
% \subsection{Inline Sage and plotting}
%
% \DescribeMacro{\sage}
% \fbox{\texttt{\bslash sage}\marg{Sage code}}
%
% \noindent takes whatever Sage code you give it, runs Sage's |latex|
% function on it, and puts the result into your document. 
%
% For example, if you do |\sage{matrix([[1, 2], [3,4]])^2}|, then that
% macro will get replaced by
% \begin{quote}
% |\left(\begin{array}{rr}|\\
% |7 & 10 \\|\\
% |15 & 22|\\
% |\end{array}\right)|
% \end{quote}
% in your document---that \LaTeX{} code is exactly exactly what you get
% from doing
% \begin{quote}
% |latex(matrix([[1, 2], [3,4]])^2)|
% \end{quote}
% in Sage.
%
% Note that you can mix \LaTeX{} variables and Sage variables! If you
% have defined the Sage variable |foo| to be $12$ (using, say, the
% |sageblock| environment), then you can do something like this:
% \begin{quote}
% |The prime factorization of the current page plus foo is|\\
% |$\sage{factor(foo + \thepage)}$|.
% \end{quote}
% Here, I'll do just that right now: the prime factorization of the
% current page plus $12$ is $\sage{factor(\thepage + 12)}$.\\
%
% \noindent \DescribeMacro{\sageplot}
% \fbox{\texttt{\bslash sageplot}\oarg{ltx opts}\oarg{fmt}\{\meta{graphics
% obj}, \meta{keyword args}\}}
%
% \noindent plots the given Sage graphics object and runs an
% |\includegraphics| command to put it into your document. The options
% work as follows:
%
% \begin{table}[h]
%   \centering
%   \begin{tabular}{l p{8cm}}
%   Option & Description \\
%   \hline
%   \meta{ltx options} & Any text here is passed directly into the
%   optional arguments (between the square brackets) of an
%   |\includegraphics| command. If not specified,
%   ``|width=.75\textwidth|'' will be used.\\
%   \meta{fmt} & You can optionally specify a file extension here; Sage
%   will then try to save the graphics object to a file with extension
%   \emph{fmt}. If not specified, \ST\ will save to EPS and PDF files.\\
%   \meta{graphics obj} & A Sage object on which you can call |.save()|
%   with a graphics filename.\\
%   \meta{keyword args} & Any keyword arguments you put here will
%   all get put into the call to |.save()|.
%   \end{tabular}
%   \end{table}
% This setup allows you to control both the Sage side of things, and the
% \LaTeX{} side. For instance, the command
% \begin{quote}
% |\sageplot[angle=30, width=5cm]{plot(sin(x), 0, pi), axes=False,|\\
% |chocolate=True}|
% \end{quote}
% will run the following command in Sage:
% \begin{quote}
% \texttt{plot(sin(x), 0, pi).save(filename=autogen, axes=False,\\
% chocolate=True)}
% \end{quote}
% The filename is automatically generated, and unless you specify a
% format, both EPS and PDF files will be generated. This allows you to
% freely switch between using, say, a DVI viewer (many of which have
% support for source specials and make the writing process easier) and
% creating PDFs for posting on the web or emailing to colleagues. Then,
% in your \LaTeX{} file, the following command will be issued
% automatically:
% \begin{quote}
% |\includegraphics[angle=30, width=5cm]{autogen}|
% \end{quote}
% You can specify a file format if you like:
% \begin{quote}
% |\sageplot[][png]{plot(sin(x), x, 0, pi)}|
% \end{quote}
% If you ask for, say, a PNG file, keep in mind that ordinary
% |latex| and DVI files have no support for DVI files; \ST{} detects
% this and will warn you that it cannot find a suitable file if using
% |latex|; if you use |pdflatex|, there will be no problems because PDF
% files can include PNG graphics.
%
% When \ST{} cannot find a graphics file, it inserts this into your
% file:
%
% \centerline{\framebox[2cm]{\rule[-1cm]{0cm}{2cm}\textbf{??}}}
% 
% \noindent That's supposed to resemble the traditional ``\textbf{??}''
% that \LaTeX{} uses to indicate missing reference, and also the
% image-not-found graphics used by web browsers.
%
% You needn't worry about the filenames; they are automatically
% generated and will be put into the directory
% \texttt{sage-plots-for-filename.tex}. You can safely delete that
% directory anytime; if \ST\ can't find the files, it will warn you to
% run Sage to regenerate them.\\
%
% \noindent\fbox{\parbox{\textwidth}{\textbf{WARNING!} When you run Sage
% on your |.sage| file, all files in the
% \texttt{sage-plots-for-filename.tex} directory \emph{will be deleted!}
% Do not put any files into that directory that you do not want to get
% automatically deleted.}}
%
% 
% \subsection{Verbatim-like environments}
%
% The \ST\ package provides several environments for typesetting and
% executing Sage code.\\
%
% \DescribeEnv{sageblock} Any text between |\begin{sageblock}| and
% |\end{sageblock}| will be typset into your file, and also written into
% the |.sage| file for execution. This means you can do something like
% this:
% \begin{quote}
% |\begin{sageblock}|\\
% |   var('x')|\\
% |   f = sin(x) - 1|\\
% |   g = log(x)|\\
% |   h = diff(f(x) * g(x), x)|\\
% |\end{sageblock}|
% \end{quote}
% and then anytime later write in your source file
% \begin{quote}
%   \texttt{We have \$h(2) = }|\sage{h(2)}|\texttt{\$, where \$h\$ is
%   the derivative of the product of \$f\$ and \$g\$.}
% \end{quote}
% and the |\sage| call will get correctly replaced by $\sage{
% diff((sin(x) - 1)*log(x), x)(1)}$. You can use any Sage or Python
% commands inside a |sageblock|; all the commands get sent directly to
% Sage.
%
% \iffalse meta-comment
%   Sadly, we can't use sageblock or similar environments in this file!
%   If you prefix the lines inside the environment with percent signs,
%   then those percent signs get written to your .sage file. If you
%   *don't* prefix the lines with percent signs, those lines get written
%   into the .sty or .py file. It's just too tricky to get docstrip and
%   the verbatim stuff to play nicely together. I'd have to redefine how
%   those environments work, and get them to strip off initial percents. 
% \fi
% 
% \DescribeEnv{sagesilent} This environment is like |sageblock|, but it
% does not typeset any of the code; it just writes it to the |.sage|
% file. This is useful if you have to do some setup in Sage that is not
% interesting or relevant to the document you are writing.
%
% \DescribeEnv{sageverbatim} This environment is the opposite of the one
% above: whatever you type will be typeset, but not written into the
% |.sage| file. This allows you to typeset psuedocode, code that will
% fail, or take too much time to execute, or whatever.
%
% \DescribeEnv{comment} Logically, we now need an environment that
% neither typesets nor executes your Sage code\ldots but the |verbatim|
% package, which is always loaded when using \ST, provides such an
% environment: |comment|. Another way to do this is to put stuff between
% |\iffalse| and |\fi|.\\
%
% \DescribeMacro{\sagetexindent} There is one final bit to our
% verbatim-like environments: the indentation. The \ST\ package defines a
% length |\sagetexindent|, which controls how much the Sage code is
% indented when typeset. You can change this length however you like
% with |\setlength|: do |\setlength{\sagetexindent}{6ex}| or whatever.
% 
% \StopEventually{}
%
% \section{Implementation}
%
% There are two pieces to this package: a \LaTeX{} style file, and a
% Python module. They are mutually interdependent, so it makes sense to
% document them both here.
%
% \subsection{The style file}
%
% \iffalse
% tell docstrip to put code into the .sty file
%<*latex>
% \fi
%
% Let's begin by loading some packages. The key bits of |sageblock| and
% friends are stol---um, adapted from the |verbatim| package manual.
%    \begin{macrocode}
\RequirePackage{verbatim}
%    \end{macrocode}
% Unsurprisingly, the |sageplot| command works poorly without graphics
% support.
%    \begin{macrocode}
\RequirePackage{graphicx}
%    \end{macrocode}
% The |makecmds| package gives us a |provideenvironment| which we need,
% and we use |ifpdf| and |ifthen| in |sageplot| so we know what kind of
% files to look for.
%    \begin{macrocode}
\RequirePackage{makecmds}
\RequirePackage{ifpdf}
\RequirePackage{ifthen}
%    \end{macrocode}
% 
% Next set up the counters and the default indent.
%    \begin{macrocode}
\newcounter{@sage}
\newcounter{@sageplot}
\setcounter{@sage}{0}
\setcounter{@sageplot}{0}
\newlength{\sagetexindent}
\setlength{\sagetexindent}{5ex}
%    \end{macrocode}
%
% It's time to deal with files. Open the |.sage| file:
%    \begin{macrocode}
\newwrite\@sagefile
\immediate\openout\@sagefile=\jobname.sage
%    \end{macrocode}
% We will write a lot of stuff to that file, so make a convenient
% abbreviation, then use it to put the initial commands into the |.sage|
% file.
%    \begin{macrocode}
\newcommand{\@wsf}[1]{\immediate\write\@sagefile{#1}}
\@wsf{import sagetex}
\@wsf{sagetex.openout('\jobname')}
%    \end{macrocode}
% Pull in the |.sout| file if it exists, or do nothing if it doesn't. I
% suppose we could do this inside an |AtBeginDocument| but I don't see
% any particular reason to do that. It will work whenever we load it.
%    \begin{macrocode}
\InputIfFileExists{\jobname.sout}{}{}
%    \end{macrocode}
%
% Now let's define the cool stuff.
%
% \begin{macro}{\sage}
% This macro combines |\ref|, |\label|, and Sage all at once. First, we
% use Sage to get a \LaTeX{} representation of whatever you give this
% function. The Sage script writes a |\newlabel| line into the |.sout|
% file, and we read the output using the |\ref| command. Usually, |\ref|
% pulls in a section or theorem number, but it will pull in arbitrary
% text just as well.
%
% The first thing it does it write its argument into the |.sage| file,
% along with a counter so we can produce a unique label, and the line
% number in the input file so we can give helpful error messages to the
% user. We wrap a try/except around the function call so that we can
% provide a more helpful error message in case something goes wrong. (In
% particular, we can tell the user which line of the |.tex| file
% contains the offending code.)
%    \begin{macrocode}
\newcommand{\sage}[1]{%
\@wsf{try:}%
\@wsf{  sagetex.inline(\the@sage, \the\inputlineno, #1)}%
\@wsf{except:}%
\@wsf{  sagetex.goboom(\the\inputlineno)}%
%    \end{macrocode}
% Our use of |\newlabel| and |\ref| seems awfully clever until you load
% the |hyperref| package, which gleefully tries to hyperlink the hell
% out of everything. This is great until it hits one of our special
% |\newlabel|s and gets deeply confused. Fortunately the |hyperref|
% folks are willing to accomodate people like us, and give us a
% |NoHyper| environment.
%    \begin{macrocode}
\begin{NoHyper}\ref{@sagelabel\the@sage}\end{NoHyper}%
%    \end{macrocode}
% Now check to see if the label has already been defined. (The
% deep internal implementation of labels in \LaTeX{} involves defining a
% function ``|r@labelname|''.) If it hasn't, we set a flag so that we
% can tell the user to run Sage on the |.sage| file at the end of the
% run. Finally, step the counter.
%    \begin{macrocode}
\@ifundefined{r@@sagelabel\the@sage}{\gdef\@rerunsage{x}}{}%
\stepcounter{@sage}}
%    \end{macrocode}
% \end{macro}
% The user might load the |hyperref| package after this one (indeed, the
% |hyperref| documentation insists that it be loaded last) or not at
% all---so when we hit the beginning of the document, provide a dummy
% |NoHyper| environment if one hasn't been defined by the |hyperref|
% package.
%    \begin{macrocode}
\AtBeginDocument{\provideenvironment{NoHyper}{}{}}
%    \end{macrocode}
%
% \begin{macro}{\@plotdir}
% A little abbreviation for the plot directory. We don't use
% |\graphicspath| because it's
% \href{http://www.tex.ac.uk/cgi-bin/texfaq2html?label=graphicspath}{
% apparently slow}---also, since we know right where our plots are
% going, no need to have \LaTeX{} looking for them.
%    \begin{macrocode}
\newcommand{\@plotdir}{sage-plots-for-\jobname.tex}
%    \end{macrocode}
% \end{macro}
%
% \begin{macro}{\sageplot}
% This function is similar to |\sage|. The neat thing that we take
% advantage of is that commas aren't special for arguments to \LaTeX{}
% commands, so it's easy to capture a bunch of keyword arguments that
% get passed right into a Python function.
%
% This macro has two optional arguments, which can't be defined using
% \LaTeX's |\newcommand|; we use Scott Pakin's brilliant
% \href{http://tug.ctan.org/tex-archive/support/newcommand/}{\texttt{newcommand}}
% package to create this macro; the options I fed to his script were
% similar to this:
%\begin{quote}
% |MACRO sageplot OPT[#1={width}] OPT[#2={notprovided}] #3|
%\end{quote}
% Observe that we are using a Python script to write \LaTeX{} code which
% writes Python code which writes \LaTeX{} code. Crazy!
% 
% Here's the ``shell'' command which does whatever magic we need to get
% two optional arguments.
%    \begin{macrocode}
\newcommand{\sageplot}[1][width=.75\textwidth]{%
  \@ifnextchar[{\sageplot@ii[#1]}{\sageplot@ii[#1][notprovided]}%]
}
%    \end{macrocode}
% That percent sign followed by a square bracket seems necessary; I have
% no idea why.
%
% The first optional argument |#1| will get shoved right into the
% optional argument for |\includegraphics|, so the user has easy control
% over both the Sage and \LaTeX{} aspects of the plotting. We define a
% default size of $3/4$ the textwidth, which seems reasonable. The
% second optional argument |#2| is the file format and allows us to tell
% what files to look for. It defaults to ``notprovided'', which tells
% the Python module to create EPS and PDF files. Everything in |#3| gets
% put into the Python function call, so the user can put in keyword
% arguments there which get interpreted correctly by Python.
%
% Let's see the real code here. We write a couple lines to the
% |.sage| file, including a counter, input line number, and all of the
% mandatory argument; all this is wrapped in another try/except. 
%    \begin{macrocode}
\def\sageplot@ii[#1][#2]#3{%
\@wsf{try:}%
\@wsf{  sagetex.initplot('\jobname')}%
\@wsf{  sagetex.sageplot(\the@sageplot, \the\inputlineno, #3, format='#2')}%
\@wsf{except:}%
\@wsf{  sagetex.goboom(\the\inputlineno)}%
%    \end{macrocode}
% Now we include the appropriate graphics file. We use our own function
% for this; it checks to see if the file exists before running the
% appropriate |\includegraphics| command, and issues some warnings it
% the file doesn't exist. If we are creating a PDF, we check to see if
% the user asked for a different format, and use that if necessary.
%    \begin{macrocode}
\ifpdf
  \ifthenelse{\equal{#2}{notprovided}}%
    {\@sagetexincludegraphics{#1}{pdf}}%
    {\@sagetexincludegraphics{#1}{#2}}%
%    \end{macrocode}
% Otherwise, we are creating a DVI file, so we always look for EPS
% files. 
%    \begin{macrocode}
\else
  \@sagetexincludegraphics{#1}{eps}%
\fi
%    \end{macrocode}
% Finally, step the counter and we're done.
%    \begin{macrocode}
\stepcounter{@sageplot}}
%    \end{macrocode}
% \end{macro}

%
% \begin{macro}{\@sagetexincludegrphics}
% This command includes the requested graphics file with the requested
% options if the file exists (note that it just needs to know the
% extension, since we use a counter in the filename).
%    \begin{macrocode}
\newcommand{\@sagetexincludegraphics}[2]{%
  \IfFileExists{\@plotdir/plot-\the@sageplot.#2}%
    {\includegraphics[#1]{\@plotdir/plot-\the@sageplot.#2}}%
%    \end{macrocode}
% If the file doesn't exist, we insert a little box to indicate it
% wasn't found, issue a warning that we didn't find a graphics file,
% then set a flag that, at the end of the run, tells the user to run
% Sage again.
%    \begin{macrocode}
    {\framebox[2cm]{\rule[-1cm]{0cm}{2cm}\textbf{??}}%
     \PackageWarning{sagetex}{Graphics file
     \@plotdir/plot-\the@sageplot\space on page \thepage\space does not
     exist}}%
     \gdef\@rerunsage{x}}
%    \end{macrocode}
% \end{macro}
%
% \begin{macro}{\@beginsagefileblock}
% This is an internal-use abbreviation that sets things up when we start
% writing a chunk of Sage code to the |.sage| file. It begins with some
% \TeX{} magic that fixes spacing, then puts the start of a try/except
% block in the |.sage| file---this not only allows the user to indent
% code without Sage/Python complaining about indentation, but lets us
% tell the user where things went wrong. The last bit is some magic from
% the |verbatim| package manual that makes \LaTeX{} respect line breaks.
%    \begin{macrocode}
\newcommand{\@beginsagefileblock}{%
  \@bsphack%
  \@wsf{sagetex.blockbegin(\the\inputlineno)}%
  \@wsf{try:}%
  \let\do\@makeother\dospecials\catcode`\^^M\active}
%    \end{macrocode}
% \end{macro}
% 
% \begin{macro}{\@endsagefileblock}
% The companion to |\@beginsagefileblock|. 
%    \begin{macrocode}
\newcommand{\@endsagefileblock}{%
\@wsf{except:}%
\@wsf{  sagetex.goboom(\the\inputlineno)}%
\@wsf{sagetex.blockend()}}
%    \end{macrocode}
% \end{macro}
%
% Now let's define the ``verbatim-like'' environments. There are four
% possibilities, corresponding to two independent choices of
% typesetting the code or not, and writing to the |.sage| file or not.
%
% \begin{environment}{sageblock}
% This environment does both: it typesets your code and puts it into the
% |.sage| file for execution by Sage.
%    \begin{macrocode}
\newenvironment{sageblock}{\@beginsagefileblock%
%    \end{macrocode}
% The space between |\@wsf{| and |\the| is crucial! It, along with the
% ``|try:|'', is what allows the user to indent code if they like.
% This line sends stuff to the |.sage| file.
%    \begin{macrocode}
\def\verbatim@processline{\@wsf{ \the\verbatim@line}%
%    \end{macrocode}
% Next, we typeset your code and start the verbatim environment.
%    \begin{macrocode}
\hspace{\sagetexindent}\the\verbatim@line\par}%
\verbatim}%
%    \end{macrocode}
% At the end of the environment, we put a chunk into the |.sage| file
% and stop the verbatim environment.
%    \begin{macrocode}
{\@endsagefileblock\endverbatim}
%    \end{macrocode}
% \end{environment}
%
% \begin{environment}{sagesilent}
% This is from the |verbatim| package manual. It's just like the above,
% except we don't typeset anything.
%    \begin{macrocode}
\newenvironment{sagesilent}{\@beginsagefileblock%
\def\verbatim@processline{\@wsf{ \the\verbatim@line}}%
\verbatim@start}%
{\@endsagefileblock\@esphack}
%    \end{macrocode}
% \end{environment}
%
% \begin{environment}{sageverbatim}
% The opposite of |sagesilent|. This is exactly the same as the verbatim
% environment, except that we include some indentation to be consistent
% with other typeset Sage code.
%    \begin{macrocode}
\newenvironment{sageverbatim}{%
\def\verbatim@processline{\hspace{\sagetexindent}\the\verbatim@line\par}%
\verbatim}%
{\endverbatim}
%    \end{macrocode}
% \end{environment}
%
% Logically, we now need an environment which neither typesets
% \emph{nor} writes code to the |.sage| file. The verbatim package's
% |comment| environment does that.\\
%
% Now we deal with some end-of-file cleanup.
%
% We tell the Sage script to write some information to the |.sout| file,
% then check to see if |@rerunsage| ever got defined. If not, all the
% inline formulas and plots worked, so do nothing.
%    \begin{macrocode}
\AtEndDocument{\@wsf{sagetex.endofdocument()}%
\@ifundefined{@rerunsage}{}%
%    \end{macrocode}
% Otherwise, we issue a warning to tell the user to run Sage on the
% |.sage| file. Part of the reason we do this is that, by using |\ref|
% to pull in the inlines, \LaTeX{} will complain about undefined
% references if you haven't run the Sage script---and for many \LaTeX{}
% users, myself included, the warning ``there were undefined
% references'' is a signal to run \LaTeX{} again. But to fix these
% particular undefined references, you need to run \emph{Sage}. We also
% suppressed file-not-found errors for graphics files, and need to tell
% the user what to do about that.
%
% At any rate, we tell the user to run Sage if it's necessary.
%    \begin{macrocode}
{\PackageWarningNoLine{sagetex}{There were undefined Sage formulas
and/or plots}%
\PackageWarningNoLine{sagetex}{Run Sage on \jobname.sage, and then run
LaTeX on \jobname.tex again}}}
%    \end{macrocode}
%
%
% \subsection{The Python module}
%
% \iffalse
% Hey, docstrip! Stop putting code into the .sty file, and start
% putting it into the .py file.
%</latex>
%<*python>
% Thanks.
% \fi
%
% The style file writes things to the |.sage| file and reads them from
% the |.sout| file. The Python module provides functions that help
% produce the |.sout| file from the |.sage| file.
%
% We start with some imports and definitions of our global variables.
% This is a relatively specialized use of Sage, so using global variables
% isn't a bad idea.
%    \begin{macrocode}
from sage.misc.latex import latex
import sys
import os
import os.path
import hashlib
import traceback
initplot_done  = False
dirname        = None
filename       = ""
%    \end{macrocode}
%
% \begin{macro}{progress}
% This is just a cute little function for printing stuff. It allows us
% to not print a linebreak, so you can get ``|start...|'' (little time
% spent processing) ``|end|'' on one line.
%    \begin{macrocode}
def progress(t,linebreak=True):
  if linebreak:
    print(t)
  else:
    sys.stdout.write(t)
%    \end{macrocode}
% \end{macro}
%
% \begin{macro}{openout}
% This function opens a |.sout.tmp| file and writes all our output to
% that. Then, when we're done, we move that to |.sout|. The
% ``autogenerated'' line is basically the same as the lines that get put
% at the top of preparsed Sage files; we are automatically generating a
% file with Sage, so it seems reasonable to add it.
%    \begin{macrocode}
def openout(f):
  global filename
  filename = f
  global _file_
  _file_ = open(f + '.sout.tmp', 'w')
  s = '% This file was *autogenerated* from the file ' + \
        os.path.splitext(filename)[0] + '.sage.\n'
  _file_.write(s)
  progress('Processing Sage code for %s.tex...' % filename)
%    \end{macrocode}
% \end{macro}
%
% \begin{macro}{deltree}
% When we start plotting, we delete the entire plots directory if it
% exists. This function does that for us.
%    \begin{macrocode}
def deltree(root):
  for name in os.listdir(root):
    path = os.path.join(root, name)
    if os.path.isdir(path):
      deltree(path)
    else:
      os.remove(path)
  os.rmdir(root) 
%    \end{macrocode}
% \end{macro}
%
% \begin{macro}{initplot}
% We only want to create the plots directory if the user actually plots
% something. This function creates the directory and sets the
% |initplot_done| flag after doing so. We make a directory based on the
% \LaTeX{} file being processed so that if there are multiple |.tex|
% files in a directory, we don't overwrite plots from another file.
%    \begin{macrocode}
def initplot(f):
  global initplot_done
  if not initplot_done:
    progress('Initializing plots directory')
    global dirname
    dirname = 'sage-plots-for-' + f + '.tex'
    if os.path.isdir(dirname):
      deltree(dirname)
    os.mkdir(dirname)
    initplot_done = True
%    \end{macrocode}
% \end{macro}
%
% \begin{macro}{inline}
% This function works with |\sage| from the style file to put Sage
% output into your \LaTeX{} file. Usually, when you use |\label|, it
% writes a line such as
% \begin{quote}
%   |\newlabel{labelname}{{section number}{page number}}|
% \end{quote}
% to the |.aux| file. When you use the |hyperref| package, there are
% more fields in the second argument, but the first two are the same.
% The |\ref| command just pulls in what's in the first field, so we can
% hijack this mechanism for our own nefarious purposes. The function
% writes a |\newlabel| line with a label made from a counter and the
% text from running Sage on |s|. 
%
% We print out the line number so if something goes wrong, the user can
% more easily track down the offending |\sage| command in the source
% file.
%
% That's a lot of explanation for a very short function:
%    \begin{macrocode}
def inline(counter, line, s):
  progress('Inline formula %s, line %s' % (counter, line))
  _file_.write('\\newlabel{@sagelabel' + str(counter) + '}{{' + \
               latex(s) + '}{}{}{}{}}\n')
%    \end{macrocode}
% We are using five fields, just like |hyperref| does, because that
% works whether or not |hyperref| is loaded. Using two fields, as in plain
% \LaTeX, doesn't work if |hyperref| is loaded.
% \end{macro}
%
% \begin{macro}{blockbegin}
% \begin{macro}{blockend}
% This function and its companion used to write stuff to the |.sout|
% file, but now they just update the user on our progress evaluating a
% code block.
%    \begin{macrocode}
def blockbegin(line):
  progress('Code block on line %s begin...' % line, False)
def blockend():
  progress('end')
%    \end{macrocode}
% \end{macro} 
% \end{macro} 
%
% \begin{macro}{sageplot}
% Here we do plotting. As mentioned in the |\sageplot| code (note the
% backslash), we're taking advantage of two things: first, that
% \LaTeX{} doesn't treat commas and spaces in macro arguments specially,
% and second, that Python (and Sage plotting functions) has nice support
% for keyword arguments. The |#3| argument to |\sageplot| becomes
% |p| and |**kwargs| below.
%    \begin{macrocode}
def sageplot(counter, line, p, format='notprovided', **kwargs):
  global dirname
  progress('Plot %s, line %s' % (counter, line))
%    \end{macrocode}
% If the user says nothing about file formats, we default to producing
% PDF and EPS. This allows the user to transparently switch between
% using a DVI previewer (which usually automatically updates when the
% DVI changes, and has support for source specials, which makes the
% writing process easier) and making PDFs. 
%    \begin{macrocode}
  if format == 'notprovided':
    formats = ['eps', 'pdf']
  else:
    formats = [format]
  for fmt in formats:
    plotfilename = os.path.join(dirname, 'plot-%s.%s' % (counter, fmt))
    print('  plotting %s with args %s' % (plotfilename, kwargs))
    p.save(filename=plotfilename, **kwargs)
%    \end{macrocode}
% \end{macro}
%
% \begin{macro}{goboom}
% When a chunk of Sage code blows up, this function bears the bad news
% to the user. Normally in Python the traceback is good enough for this,
% but in this case, we start with a |.sage| file (which is
% autogenerated) which autogenerates a |.py| file---and the tracebacks
% the user sees refer to that file, whose line numbers are basically
% useless. We want to tell them where in the \LaTeX{} file things went
% bad, so we do that, give them the traceback, and exit after removing
% the |.sout.tmp| file.
%    \begin{macrocode}
def goboom(line):
  global filename
  print('\n**** Error in Sage code on line %s of %s.tex! Traceback follows.' % (line, filename))
  traceback.print_exc()
  print('\n**** Running Sage on %s.sage failed! Fix %s.tex and try again.' % (filename, filename))
  os.remove(filename + '.sout.tmp')
  sys.exit(1)
%    \end{macrocode}
% \end{macro}
% 
% \begin{macro}{endofdocument}
% When we're done processing, we have a couple little cleanup tasks. We
% find the MD5 sum of the |.sage| file that produced the |.sout| file
% we're about to write, and put that sum into the |.sout| file. This
% allows external programs that build \LaTeX{} documents to tell if they
% need to call Sage to update the |.sout| file. We do issue warnings to
% run Sage on the |.sage| file, but those warnings do not quite capture
% all situations, and anyway I think it's easier to grab the md5sum out
% of the end of the file than parse the output from running |latex| on
% your file. (The regular expression \verb|^%[0-9a-f]{32}%| will find
% the MD5 sum.)
%    \begin{macrocode}
def endofdocument():
  global filename
  dotsagesum = hashlib.md5(open(filename + '.sage', 'rb').read()).hexdigest()
  s = '%' + dotsagesum + '% md5sum of .sage file that produced this\n'
  _file_.write(s)
%    \end{macrocode}
% Now we are done with the |.sout| file. Close it, rename it, and tell
% the user we're done.
%    \begin{macrocode}
  _file_.close()
  os.rename(filename + '.sout.tmp', filename + '.sout')
  progress('Sage processing complete. Run LaTeX on %s.tex again.' % filename)
%    \end{macrocode}
% \end{macro}
%
% \section{Credits}
%
% According to the original README file, this system was originally
% done by Gonzalo Tornaria and Joe Wetherell. Later Harald Schilly made
% some improvements and modifications. Almost all the examples in the
% |example.tex| file are from Harald.
%
% Dan Drake rewrote the style file (there is almost zero original code
% there), made significant changes to the Python module, put both
% files into \textsf{Docstrip} format, and wrote all the documentation.
