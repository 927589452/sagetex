% \section{Implementation}
% \label{sec:implementation}
%
% There are two pieces to this package: a \LTX style file, and a
% Python module. They are mutually interdependent, so it makes sense to
% document them both here.
%
% \subsection{The style file}
% \label{sec:sty-file}
%
% \iffalse
% tell docstrip to put code into the .sty file
%<*latex>
% \fi
%
% All macros and counters intended for use internal to this package
% begin with ``|ST@|''.
%
% \subsubsection{Initialization}
%
% Let's begin by loading some packages. The key bits of |sageblock| and
% friends are stol---um, adapted from the |verbatim| package manual. So
% grab the |verbatim| package.
%    \begin{macrocode}
\RequirePackage{verbatim}
%    \end{macrocode}
% Unsurprisingly, the |\sageplot| command works poorly without graphics
% support.
%    \begin{macrocode}
\RequirePackage{graphicx}
%    \end{macrocode}
% The |makecmds| package gives us a |\provideenvironment| which we need,
% and we use |ifpdf| and |ifthen| in |\sageplot| so we know what kind of
% files to look for.
%    \begin{macrocode}
\RequirePackage{makecmds}
\RequirePackage{ifpdf}
\RequirePackage{ifthen}
%    \end{macrocode}
%
% Next set up the counters, default indent, and flags.
%    \begin{macrocode}
\newcounter{ST@inline}
\newcounter{ST@plot}
\setcounter{ST@inline}{0}
\setcounter{ST@plot}{0}
\newlength{\sagetexindent}
\setlength{\sagetexindent}{5ex}
\newif\ifST@paused
\ST@pausedfalse
%    \end{macrocode}
% Set up the file stuff, which will get run at the beginning of the
% document, after we know what's happening with the |final| option. 
% First, we open the |.sage| file:
%    \begin{macrocode}
\AtBeginDocument{\@ifundefined{ST@final}{%
\newwrite\ST@sf%
\immediate\openout\ST@sf=\jobname.sage%
%    \end{macrocode}
% \begin{macro}{\ST@wsf}
% We will write a lot of stuff to that file, so make a convenient
% abbreviation, then use it to put the initial commands into the
% |.sage| file. The hash mark below gets doubled when written to the
% file, for some obscure reason related to parameter expansion. It's
% valid Python, though, so I haven't bothered figuring out how to get
% a single hash. We are assuming that the extension is |.tex|; see the
% |initplot| documentation on page~\pageref{initplot} for discussion
% of file extensions. The ``|(\jobname.sage)|'' business is there
% because the comment below will get pulled into the autogenerated
% |.py| file (second order autogeneration!) and I'd like to reduce
% possible confusion if someone is looking around in those files.
%    \begin{macrocode}
\newcommand{\ST@wsf}[1]{\immediate\write\ST@sf{#1}}%
\ST@wsf{# This file (\jobname.sage) was *autogenerated* from the file \jobname.tex.}%
\ST@wsf{import sagetex}%
\ST@wsf{_st_ = sagetex.SageTeXProcessor('\jobname')}}%
%    \end{macrocode}
% On the other hand, if the |ST@final| flag is set, don't bother with
% any of the file stuff, and make |\ST@wsf| a no-op.
%    \begin{macrocode}
{\newcommand{\ST@wsf}[1]{\relax}}}
%    \end{macrocode}
% \end{macro}
% Now we declare our options, which mostly just set flags that we check
% at the beginning of the document, and when running the |.sage| file.
% \changes{v2.0}{2008/04/04}{Add \texttt{epstopdf} option}
% \changes{v2.0}{2008/12/16}{Add \texttt{final} option}
%
% The |final| option controls whether or not we write the |.sage| file;
% the |imagemagick| and |epstopdf| options both want to write something
% to that same file. So we put off all the actual file stuff until the
% beginning of the document---by that time, we'll have processed the
% |final| option (or not) and can check the |\ST@final| flag to see what
% to do. (We must do this because we can't specify code that runs if an
% option \emph{isn't} defined.)
%
% For |final|, we set a flag for other guys to check, and if there's no
% |.sout| file, we warn the user that something fishy is going on.
%    \begin{macrocode}
\DeclareOption{final}{%
  \newcommand{\ST@final}{x}%
  \IfFileExists{\jobname.sout}{}{\AtEndDocument{\PackageWarningNoLine{sagetex}%
  {`final' option provided, but \jobname.sout^^Jdoesn't exist! No Sage
    input will appear in your document. Remove the `final'^^Joption and
    rerun LaTeX on your document}}}}
%    \end{macrocode}
% For |imagemagick|, we set two flags: one for \LTX and one for Sage.
% It's important that we set |ST@useimagmagick| \emph{before} the
% beginning of the document, so that the graphics commands can check
% that. We do wait until the beginning of the document to do file
% writing stuff.
%    \begin{macrocode}
\DeclareOption{imagemagick}{%
  \newcommand{\ST@useimagemagick}{x}%
  \AtBeginDocument{%
  \@ifundefined{ST@final}{%
    \ST@wsf{_st_.useimagemagick = True}}{}}}
%    \end{macrocode}
% For |epstopdf|, we just set a flag for Sage. Then, process the options.
%    \begin{macrocode}
\DeclareOption{epstopdf}{%
\AtBeginDocument{%
\@ifundefined{ST@final}{%
  \ST@wsf{_st_.useepstopdf = True}}{}}}
\ProcessOptions\relax
%    \end{macrocode}
% The |\relax| is a little incantation suggested by the ``\LaTeXe{} for
% class and package writers'' manual, section 4.7.
%
% Pull in the |.sout| file if it exists, or do nothing if it doesn't. I
% suppose we could do this inside an |AtBeginDocument| but I don't see
% any particular reason to do that. It will work whenever we load it.
%    \begin{macrocode}
\InputIfFileExists{\jobname.sout}{}{}
%    \end{macrocode}
%
% The user might load the |hyperref| package after this one (indeed, the
% |hyperref| documentation insists that it be loaded last) or not at
% all---so when we hit the beginning of the document, provide a dummy
% |NoHyper| environment if one hasn't been defined by the |hyperref|
% package. We need this for the |\sage| macro below.
%    \begin{macrocode}
\AtBeginDocument{\provideenvironment{NoHyper}{}{}}
%    \end{macrocode}
%
% \subsubsection{The \texttt{\bslash sage} macro}
% \label{sec:sagemacro}
%
% \begin{macro}{\sage}
% This macro combines |\ref|, |\label|, and Sage all at once. First, we
% use Sage to get a \LTX representation of whatever you give this
% function. The Sage script writes a |\newlabel| line into the |.sout|
% file, and we read the output using the |\ref| command. Usually, |\ref|
% pulls in a section or theorem number, but it will pull in arbitrary
% text just as well.
%
% The first thing it does it write its argument into the |.sage| file,
% along with a counter so we can produce a unique label. We wrap a
% try/except around the function call so that we can provide a more
% helpful error message in case something goes wrong. (In particular, we
% can tell the user which line of the |.tex| file contains the offending
% code.) We can use |^^J| to put linebreaks into the |.sage| file, but
% \LTX wants to put a space after that, which is why we don't put the
% ``except'' on its own line here in the source.
%    \begin{macrocode}
\newcommand{\sage}[1]{\ST@wsf{%
try:^^J
 _st_.inline(\theST@inline, #1)^^Jexcept:^^J
 _st_.goboom(\the\inputlineno)}%
%    \end{macrocode}
% The |inline| function of the Python module is documented on page
% \pageref{inlinefn}. Back in \LTX-land: if paused, say so.
%    \begin{macrocode}
\ifST@paused
  \mbox{(Sage\TeX{} is paused)}%
%    \end{macrocode}
% Otherwise\ldots our use of |\newlabel| and |\ref| seems awfully clever
% until you load the |hyperref| package, which gleefully tries to
% hyperlink the hell out of everything. This is great until it hits one
% of our special |\newlabel|s and gets deeply confused. Fortunately the
% |hyperref| folks are willing to accomodate people like us, and give us
% a |NoHyper| environment.
%    \begin{macrocode}
\else
  \begin{NoHyper}\ref{@sageinline\theST@inline}\end{NoHyper}
%    \end{macrocode}
% Now check if the label has already been defined. (The internal
% implementation of labels in \LTX involves defining a macro called
% ``|r@@labelname|''.) If it hasn't, we set a flag so that we can tell
% the user to run Sage on the |.sage| file at the end of the run.
%    \begin{macrocode}
  \@ifundefined{r@@sageinline\theST@inline}{\gdef\ST@rerun{x}}{}
\fi
%    \end{macrocode}
% In any case, the last thing to do is step the counter.
%    \begin{macrocode}
\stepcounter{ST@inline}}
%    \end{macrocode}
% \end{macro}
%
% \begin{macro}{\percent} 
% A macro that inserts a percent sign. This is more-or-less stolen from the
% \textsf{Docstrip} manual; there they change the catcode inside a group
% and use |gdef|, but here we try to be more \LaTeX y and use
% |\newcommand|.
%    \begin{macrocode}
\catcode`\%=12
\newcommand{\percent}{%}
\catcode`\%=14
%    \end{macrocode}
% \end{macro}
%
% \subsubsection{The \texttt{\bslash sageplot} macro and friends}
% \label{sec:sageplotmacro}
%
% Plotting is rather more complicated, and requires several helper
% macros that accompany |\sageplot|.
%
% \begin{macro}{\ST@plotdir}
% A little abbreviation for the plot directory. We don't use
% |\graphicspath| because it's
% \href{http://www.tex.ac.uk/cgi-bin/texfaq2html?label=graphicspath}{
% apparently slow}---also, since we know right where our plots are
% going, no need to have \LTX looking for them.
%    \begin{macrocode}
\newcommand{\ST@plotdir}{sage-plots-for-\jobname.tex}
%    \end{macrocode}
% \end{macro}
%
% \begin{macro}{\ST@missingfilebox}
% The code that makes the ``file not found'' box. This shows up in a
% couple places below, so let's just define it once.
%    \begin{macrocode}
\newcommand{\ST@missingfilebox}{\framebox[2cm]{\rule[-1cm]{0cm}{2cm}\textbf{??}}}
%    \end{macrocode}
% \end{macro}
% \begin{macro}{\sageplot}
% \changes{v1.3}{2008/03/08}{Iron out warnings, cool \TikZ flowchart}
% This function is similar to |\sage|. The neat thing that we take
% advantage of is that commas aren't special for arguments to \LTX
% commands, so it's easy to capture a bunch of keyword arguments that
% get passed right into a Python function.
%
% This macro has two optional arguments, which can't be defined using
% \LTX's |\newcommand|; we use Scott Pakin's brilliant
% \href{http://tug.ctan.org/tex-archive/support/newcommand/}{|newcommand|}
% package to create this macro; the options I fed to his script were
% similar to this:
%\begin{center}
% |MACRO sageplot OPT[#1={width}] OPT[#2={notprovided}] #3|
%\end{center}
% Observe that we are using a Python script to write \LTX code which
% writes Python code which writes \LTX code. Crazy!
% 
% Here's the wrapper command which does whatever magic we need to get
% two optional arguments.
%    \begin{macrocode}
\newcommand{\sageplot}[1][width=.75\textwidth]{%
  \@ifnextchar[{\ST@sageplot[#1]}{\ST@sageplot[#1][notprovided]}}
%    \end{macrocode}
% The first optional argument |#1| will get shoved right into the
% optional argument for |\includegraphics|, so the user has easy control
% over the \LTX aspects of the plotting. We define a default size of
% $3/4$ the textwidth, which seems reasonable. (Perhaps a future version
% of \ST will allow the user to specify in the package options a set of
% default options to be used throughout.) The second optional argument
% |#2| is the file format and allows us to tell what files to look for.
% It defaults to ``notprovided'', which tells the Python module to
% create EPS and PDF files. Everything in |#3| gets put into the Python
% function call, so the user can put in keyword arguments there which
% get interpreted correctly by Python.
%
% \begin{macro}{\ST@sageplot}
% \changes{v2.0}{2008/12/16}{Change to use only keyword arguments: see issue
% 2 on bitbucket tracker}
% Let's see the real code here. We write a couple lines to the |.sage|
% file, including a counter, input line number, and all of the mandatory
% argument; all this is wrapped in another try/except.
%    \begin{macrocode}
\def\ST@sageplot[#1][#2]#3{\ST@wsf{try:^^J
 _st_.plot(\theST@plot, format='#2', _p_=#3)^^Jexcept:^^J
 _st_.goboom(\the\inputlineno)}%
%    \end{macrocode}
% The Python |plot| function is documented on page~\pageref{plotfn}.
%
% Now we include the appropriate graphics file. Because the user might
% be producing DVI or PDF files, and have supplied a file format or not,
% and so on, the logic we follow is a bit complicated.
% \autoref{f:sageplottree} shows what we do; for completeness---and
% because I think drawing trees with \TikZ is really cool---we show what
% |\ST@inclgrfx| does in \autoref{f:stig}. This entire complicated
% business is intended to avoid doing an |\includegraphics| command on a
% file that doesn't exist, and to issue warnings appropriate to the
% situation.
%
% \begin{figure}
%   \centering
%   \begin{tikzpicture}
%     \tikzstyle{level 1}=[sibling distance=6cm]
%     \tikzstyle{level 2}=[sibling distance=3cm]
%     \node [box] {DVI or PDF?}
%       child {node [box] {Format provided?}
%         child {node [box] {STig EPS}
%           edge from parent node[left] {no}}
%         child {node [box] {IM option set?}
%           child {node [box, text width=3cm] {Warn that DVI + PNG = bad}
%             edge from parent node[left] {no}}
%           child {node [box] {STig EPS}
%             edge from parent node[right] {yes}}
%           edge from parent node[right] {yes}}
%         edge from parent node[left] {DVI}}
%       child {node [box] {Format provided?}
%         child {node [box] {STig PDF}
%           edge from parent node[left] {no}}
%         child {node [box] {STig \texttt{\#2}}
%           edge from parent node[right] {yes}}
%         edge from parent node[right] {PDF}};
%   \end{tikzpicture}
%   \caption{The logic tree that \texttt{\bslash sageplot} uses to
%   decide whether to run \texttt{\bslash includegraphics} or to yell at
%   the user. ``Format'' is the \texttt{\#2} argument to \texttt{\bslash
%   sageplot}, ``STig ext''
%   means a call to \texttt{\bslash ST@inclgrfx} with ``ext'' as the
%   second argument, and ``IM'' is Imagemagick.}
%   \label{f:sageplottree}
% \end{figure}
%
% If we are creating a PDF, we check to see if the user asked for a
% different format, and use that if necessary:
%    \begin{macrocode}
\ifpdf
  \ifthenelse{\equal{#2}{notprovided}}%
    {\ST@inclgrfx{#1}{pdf}}%
    {\ST@inclgrfx{#1}{#2}}%
%    \end{macrocode}
% Otherwise, we are creating a DVI file, which only supports EPS. If the
% user provided a format anyway, don't include the file (since it won't
% work) and warn the user about this. (Unless the file doesn't exist, in
% which case we do the same thing that |\ST@inclgrfx| does.)
%    \begin{macrocode}
\else
  \ifthenelse{\equal{#2}{notprovided}}%
    {\ST@inclgrfx{#1}{eps}}%
%    \end{macrocode}
% If a format is provided, we check to see if we're using the
% imagemagick option. If not, we're going to issue some sort of warning,
% depending on whether the file exists yet or not.
%    \begin{macrocode}
    {\@ifundefined{ST@useimagemagick}%
      {\IfFileExists{\ST@plotdir/plot-\theST@plot.#2}%
        {\ST@missingfilebox%
         \PackageWarning{sagetex}{Graphics file
         \ST@plotdir/plot-\theST@plot.#2\space on page \thepage\space
         cannot be used with DVI output. Use pdflatex or create an EPS
         file. Plot command is}}%
        {\ST@missingfilebox%
         \PackageWarning{sagetex}{Graphics file
         \ST@plotdir/plot-\theST@plot.#2\space on page \thepage\space
         does not exist. Plot command is}%
         \gdef\ST@rerun{x}}}%
%    \end{macrocode}
% Otherwise, we are using Imagemagick, so try to include an EPS file
% anyway.
%    \begin{macrocode}
    {\ST@inclgrfx{#1}{eps}}}%
\fi
%    \end{macrocode}
% Step the counter and we're done with the usual work.
%    \begin{macrocode}
\stepcounter{ST@plot}}
%    \end{macrocode}
% \end{macro}
% \end{macro}
%
% \begin{macro}{\ST@inclgrfx}
% This command includes the requested graphics file (|#2| is the
% extension) with the requested options (|#1|) if the file exists. Note
% that it just needs to know the extension, since we use a counter for
% the filename. If we are paused, it just puts in a little box saying
% so.
%    \begin{macrocode}
\newcommand{\ST@inclgrfx}[2]{\ifST@paused
  \fbox{\rule[-1cm]{0cm}{2cm}Sage\TeX{} is paused; no graphic}
\else
  \IfFileExists{\ST@plotdir/plot-\theST@plot.#2}%
    {\includegraphics[#1]{\ST@plotdir/plot-\theST@plot.#2}}%
%    \end{macrocode}
% If the file doesn't exist, we insert a little box to indicate it
% wasn't found, issue a warning that we didn't find a graphics file,
% then set a flag that, at the end of the run, tells the user to run
% Sage again.
%    \begin{macrocode}
    {\ST@missingfilebox%
     \PackageWarning{sagetex}{Graphics file
     \ST@plotdir/plot-\theST@plot.#2\space on page \thepage\space does not
     exist. Plot command is}%
     \gdef\ST@rerun{x}}
\fi}
%    \end{macrocode}
% \autoref{f:stig} makes this a bit clearer. 
% \begin{figure}
%   \centering
%   \begin{tikzpicture}
%     \tikzstyle{level 1}=[sibling distance=4cm]
%     \node [box] {Paused?}
%       child {node [box] {Insert ``we're paused'' box}
%         edge from parent node[left] {yes}}
%       child {node [box] {Does EXT file exist?}
%         child {node [box, text width = 2.125cm] {Warn user to rerun Sage}
%           edge from parent node[left] {no}}
%         child {node [box] {Use \texttt{includegraphics}}
%           edge from parent node[right] {yes}}
%         edge from parent node[right] {no}};
%   \end{tikzpicture}
%   \caption{The logic used by the \texttt{\bslash ST@inclgrfx}
%   command.}
%   \label{f:stig}
% \end{figure}
% \end{macro}
%
% \subsubsection{Verbatim-like environments}
% \label{sec:verbatim-envs}
%
% \begin{macro}{\ST@beginsfbl}
% This is ``begin |.sage| file block'', an internal-use abbreviation
% that sets things up when we start writing a chunk of Sage code to
% the |.sage| file. It begins with some \TeX{} magic that fixes
% spacing, then puts the start of a try/except block in the |.sage|
% file---this not only allows the user to indent code without
% Sage/Python complaining about indentation, but lets us tell the user
% where things went wrong. The |blockbegin| and |blockend| functions
% are documented on page~\pageref{blocksbeginend}. The last bit is some
% magic from the |verbatim| package manual that makes \LTX respect
% line breaks.
%    \begin{macrocode}
\newcommand{\ST@beginsfbl}{%
  \@bsphack\ST@wsf{%
_st_.blockbegin()^^Jtry:}%
  \let\do\@makeother\dospecials\catcode`\^^M\active}
%    \end{macrocode}
% \end{macro}
%
% \begin{macro}{\ST@endsfbl}
% The companion to |\ST@beginsfbl|. 
%    \begin{macrocode}
\newcommand{\ST@endsfbl}{%
\ST@wsf{except:^^J
 _st_.goboom(\the\inputlineno)^^J_st_.blockend()}}
%    \end{macrocode}
% \end{macro}
%
% Now let's define the ``verbatim-like'' environments. There are four
% possibilities, corresponding to the two independent choices of
% typesetting the code or not, and writing to the |.sage| file or not.
%
% \begin{environment}{sageblock}
% This environment does both: it typesets your code and puts it into the
% |.sage| file for execution by Sage.
%    \begin{macrocode}
\newenvironment{sageblock}{\ST@beginsfbl%
%    \end{macrocode}
% The space between |\ST@wsf{| and |\the| is crucial! It, along with the
% ``|try:|'', is what allows the user to indent code if they like. This
% line sends stuff to the |.sage| file.
%    \begin{macrocode}
\def\verbatim@processline{\ST@wsf{ \the\verbatim@line}%
%    \end{macrocode}
% Next, we typeset your code and start the verbatim environment.
%    \begin{macrocode}
\hspace{\sagetexindent}\the\verbatim@line\par}%
\verbatim}%
%    \end{macrocode}
% At the end of the environment, we put a chunk into the |.sage| file
% and stop the verbatim environment.
%    \begin{macrocode}
{\ST@endsfbl\endverbatim}
%    \end{macrocode}
% \end{environment}
%
% \begin{environment}{sagesilent}
% This is from the |verbatim| package manual. It's just like the above,
% except we don't typeset anything.
%    \begin{macrocode}
\newenvironment{sagesilent}{\ST@beginsfbl%
\def\verbatim@processline{\ST@wsf{ \the\verbatim@line}}%
\verbatim@start}%
{\ST@endsfbl\@esphack}
%    \end{macrocode}
% \end{environment}
%
% \begin{environment}{sageverbatim}
% The opposite of |sagesilent|. This is exactly the same as the verbatim
% environment, except that we include some indentation to be consistent
% with other typeset Sage code.
%    \begin{macrocode}
\newenvironment{sageverbatim}{%
\def\verbatim@processline{\hspace{\sagetexindent}\the\verbatim@line\par}%
\verbatim}%
{\endverbatim}
%    \end{macrocode}
% \end{environment}
%
% Logically, we now need an environment which neither typesets
% \emph{nor} writes code to the |.sage| file. The verbatim package's
% |comment| environment does that.\\
%
% \subsubsection{Pausing \ST}
% \label{sec:pausing-sagetex}
%
% How can one have Sage to stop processing \ST output for a little
% while, and then start again? At first I thought I would need some sort
% of ``goto'' statement in Python, but later realized that there's a
% dead simple solution: write triple quotes to the |.sage| file to
% comment out the code. Okay, so this isn't \emph{really} commenting out
% the code; PEP 8 says block comments should use ``|#|'' and Sage will
% read in the ``commented-out'' code as a string literal. For the
% purposes of \ST, I think this is a good decision, though, since (1)
% the pausing mechanism is orthogonal to everything else, which makes it
% easier to not screw up other code, and (2) it will always work.
%
% This illustrates what I really like about \ST: it mixes \LTX and
% Sage/Python, and often what is difficult or impossible in one system
% is trivial in the other.
%
% \begin{macro}{sagetexpause}
% This macro pauses \ST by effectively commenting out code in the
% |.sage| file. When running the corresponding |.sage| file, Sage will
% skip over any commands issued while \ST is paused.
%    \begin{macrocode}
\newcommand{\sagetexpause}{\ifST@paused\relax\else
\ST@wsf{print 'SageTeX paused on \jobname.tex line \the\inputlineno'^^J"""}
\ST@pausedtrue
\fi}
%    \end{macrocode}
% \end{macro}
%
% \begin{macro}{sagetexunpause}
% This is the obvious companion to |\sagetexpause|.
%    \begin{macrocode}
\newcommand{\sagetexunpause}{\ifST@paused
\ST@wsf{"""^^Jprint 'SageTeX unpaused on \jobname.tex line \the\inputlineno'}
\ST@pausedfalse
\fi}
%    \end{macrocode}
% \end{macro}
%
% \subsubsection{End-of-document cleanup}
% \label{sec:end-of-doc-cleanup}
%
% We tell the Sage script to write some information to the |.sout| file,
% then check to see if |ST@rerun| ever got defined. If not, all the
% inline formulas and plots worked, so do nothing. We check to see if
% we're paused first, so that we can finish the triple-quoted string in
% the |.sage| file.
%    \begin{macrocode}
\AtEndDocument{\ifST@paused
\ST@wsf{"""^^Jprint 'SageTeX unpaused at end of \jobname.tex'}
\fi
\ST@wsf{_st_.endofdoc()}%
\@ifundefined{ST@rerun}{}%
%    \end{macrocode}
% Otherwise, we issue a warning to tell the user to run Sage on the
% |.sage| file. Part of the reason we do this is that, by using |\ref|
% to pull in the inlines, \LTX will complain about undefined references
% if you haven't run the Sage script---and for many \LTX users, myself
% included, the warning ``there were undefined references'' is a signal
% to run \LTX again. But to fix these particular undefined references,
% you need to run \emph{Sage}. We also suppressed file-not-found errors
% for graphics files, and need to tell the user what to do about that.
%
% At any rate, we tell the user to run Sage if it's necessary.
%    \begin{macrocode}
{\PackageWarningNoLine{sagetex}{There were undefined Sage formulas
and/or plots.^^JRun Sage on \jobname.sage, and then run
LaTeX on \jobname.tex again}}}
%    \end{macrocode}
%
%
% \subsection{The Python module}
% \label{sec:py-file}
%
% \iffalse
% Hey, docstrip! Stop putting code into the .sty file, and start
% putting it into the .py file.
%</latex>
%<*python>
% Thanks.
% \fi
%
% The style file writes things to the |.sage| file and reads them from
% the |.sout| file. The Python module provides functions that help
% produce the |.sout| file from the |.sage| file.
%
% \paragraph{A note on Python and \textsf{Docstrip}} There is one tiny
% potential source of confusion when documenting Python code with
% \textsf{Docstrip}: the percent sign. If you have a long line of Python
% code which includes a percent sign for string formatting and you break
% the line with a backslash and begin the next line with a percent sign,
% that line \emph{will not} be written to the output file. This is only
% a problem if you \emph{begin} the line with a (single) percent sign;
% there are no troubles otherwise.\\
%
% On to the code: the |sagetex.py| file is intended to be used as a
% module and doesn't do anything useful when called directly, so if
% someone does that, warn them. We do this right away so that we print
% this and exit before trying to import any Sage modules; that way, this
% error message gets printed whether you run the script with Sage or
% with Python.
%    \begin{macrocode}
import sys
if __name__ == "__main__":
  print("""This file is part of the SageTeX package.
It is not meant to be called directly.

This file will be automatically used by Sage scripts generated from a
LaTeX document using the SageTeX package.""")
  sys.exit()
%    \end{macrocode}
% Import what we need:
%    \begin{macrocode}
from sage.misc.latex import latex
import os
import os.path
import hashlib
import traceback
import subprocess
import shutil
%    \end{macrocode}
% We define a class so that it's a bit easier to carry around internal
% state. We used to just have some global variables and a bunch of
% functions, but this seems a bit nicer and easier.
%    \begin{macrocode}
class SageTeXProcessor():
  def __init__(self, jobname):
    self.progress('Processing Sage code for %s.tex...' % jobname)
    self.didinitplot = False
    self.useimagemagick = False
    self.useepstopdf = False
    self.plotdir = 'sage-plots-for-' + jobname + '.tex'
    self.filename = jobname
%    \end{macrocode}
% Open a |.sout.tmp| file and write all our output to that. Then, when
% we're done, we move that to |.sout|. The ``autogenerated'' line is
% basically the same as the lines that get put at the top of preparsed
% Sage files; we are automatically generating a file with Sage, so it
% seems reasonable to add it.
%    \begin{macrocode}
    self.souttmp = open(self.filename + '.sout.tmp', 'w')
    s = '% This file was *autogenerated* from the file ' + \
        os.path.splitext(jobname)[0] + '.sage.\n'
    self.souttmp.write(s)
%    \end{macrocode}
%
% \begin{macro}{progress}
% This function just prints stuff. It allows us to not print a
% linebreak, so you can get ``|start...|'' (little time spent
% processing) ``|end|'' on one line.
%    \begin{macrocode}
  def progress(self, t,linebreak=True):
    if linebreak:
      print(t)
    else:
      sys.stdout.write(t)
      sys.stdout.flush()
%    \end{macrocode}
% \end{macro}
%
% \begin{macro}{initplot}
% \phantomsection\label{initplot}
% We only want to create the plots directory if the user actually plots
% something. This function creates the directory and sets the
% |didinitplot| flag after doing so. We make a directory based on the
% \LTX file being processed so that if there are multiple |.tex| files
% in a directory, we don't overwrite plots from another file.
%    \begin{macrocode}
  def initplot(self):
    self.progress('Initializing plots directory')
%    \end{macrocode}
% We hard-code the |.tex| extension, which is fine in the overwhelming
% majority of cases, although it does cause minor confusion when
% building the documentation. If it turns out lots of people use, say, a
% |ltx| extension or whatever, We could find out the correct extension,
% but it would involve a lot of irritating mucking around---on
% |comp.text.tex|, the best solution I found for finding the file
% extension is to look through the |.log| file.
%    \begin{macrocode}
    if os.path.isdir(self.plotdir):
      shutil.rmtree(self.plotdir)
    os.mkdir(self.plotdir)
    self.didinitplot = True
%    \end{macrocode}
% \end{macro}
%
% \begin{macro}{inline}
% \phantomsection\label{inlinefn}
% This function works with |\sage| from the style file (see
% \autoref{sec:sagemacro}) to put Sage output into your \LTX file.
% Usually, when you use |\label|, it writes a line such as
% \begin{center}
%   |\newlabel{labelname}{{section number}{page number}}|
% \end{center}
% to the |.aux| file. When you use the |hyperref| package, there are
% more fields in the second argument, but the first two are the same.
% The |\ref| command just pulls in what's in the first field of the
% second argument, so we can hijack this mechanism for our own nefarious
% purposes. The function writes a |\newlabel| line with a label made
% from a counter and the text from running Sage on |s|.
%
% We print out the line number so if something goes wrong, the user can
% more easily track down the offending |\sage| command in the source
% file.
%
% That's a lot of explanation for a very short function:
%    \begin{macrocode}
  def inline(self, counter, s):
    self.progress('Inline formula %s' % counter)
    self.souttmp.write('\\newlabel{@sageinline' + str(counter) + '}{{' + \
                 latex(s) + '}{}{}{}{}}\n')
%    \end{macrocode}
% We are using five fields, just like |hyperref| does, because that
% works whether or not |hyperref| is loaded. Using two fields, as in
% plain \LTX, doesn't work if |hyperref| is loaded.
% \end{macro}
%
% \begin{macro}{blockbegin}
% \begin{macro}{blockend}
% \phantomsection\label{blocksbeginend}
% This function and its companion used to write stuff to the |.sout|
% file, but now they just update the user on our progress evaluating a
% code block. The verbatim-like environments of
% \autoref{sec:verbatim-envs} use these functions.
%    \begin{macrocode}
  def blockbegin(self):
    self.progress('Code block begin...', False)
  def blockend(self):
    self.progress('end')
%    \end{macrocode}
% \end{macro} 
% \end{macro} 
%
% \begin{macro}{plot}
% \phantomsection\label{plotfn}
% I hope it's obvious that this function does plotting. It's the Python
% counterpart of |\ST@sageplot| described in \autoref{sec:sageplotmacro}. As
% mentioned in the |\sageplot| code, we're taking advantage of two
% things: first, that \LTX doesn't treat commas and spaces in macro
% arguments specially, and second, that Python (and Sage plotting
% functions) has nice support for keyword arguments. The |#3| argument
% to |\sageplot| becomes |_p_| and |**kwargs| below.
%    \begin{macrocode}
  def plot(self, counter, _p_, format='notprovided', **kwargs):
    if not self.didinitplot:
      self.initplot()
    self.progress('Plot %s' % counter)
%    \end{macrocode}
% If the user says nothing about file formats, we default to producing
% PDF and EPS. This allows the user to transparently switch between
% using a DVI previewer (which usually automatically updates when the
% DVI changes, and has support for source specials, which makes the
% writing process easier) and making PDFs.\footnote{Yes, there's
% \texttt{pdfsync}, but full support for that is still rare in Linux, so
% producing EPS and PDF is the best solution for now.}
%    \begin{macrocode}
    if format == 'notprovided':
      formats = ['eps', 'pdf']
    else:
      formats = [format]
    for fmt in formats:
%    \end{macrocode}
% If we're making a PDF and have been told to use |epstopdf|, do so,
% then skip the rest of the loop.
%    \begin{macrocode}
      if fmt == 'pdf' and self.useepstopdf:
        epsfile = os.path.join(self.plotdir, 'plot-%s.eps' % counter)
        self.progress('Calling epstopdf to convert plot-%s.eps to PDF' % \
            counter)
        subprocess.check_call(['epstopdf', epsfile])
        continue
      plotfilename = os.path.join(self.plotdir, 'plot-%s.%s' % (counter, fmt))
      #print('  plotting %s with args %s' % (plotfilename, kwargs))
      _p_.save(filename=plotfilename, **kwargs)
%    \end{macrocode}
% If the user provides a format \emph{and} specifies the |imagemagick|
% option, we try to convert the newly-created file into EPS format.
%    \begin{macrocode}
      if format != 'notprovided' and self.useimagemagick:
        self.progress('Calling Imagemagick to convert plot-%s.%s to EPS' % \
          (counter, format))
        self.toeps(counter, format)
%    \end{macrocode}
% \end{macro}
%
% \begin{macro}{toeps}
% This function calls the Imagmagick utility |convert| to, well, convert
% something into EPS format. This gets called when the user has
% requested the ``|imagemagick|'' option to the \ST\ style file and is
% making a graphic file with a nondefault extension.
%    \begin{macrocode}
  def toeps(self, counter, ext):
    subprocess.check_call(['convert',\
      '%s/plot-%s.%s' % (self.plotdir, counter, ext), \
      '%s/plot-%s.eps' % (self.plotdir, counter)])
%    \end{macrocode}
% We are blindly assuming that the |convert| command exists and will do
% the conversion for us; the |check_call| function raises an exception
% which, since all these calls get wrapped in try/excepts in the |.sage|
% file, should result in a reasonable error message if something strange
% happens.
% \end{macro}
%
% \begin{macro}{goboom}
% When a chunk of Sage code blows up, this function bears the bad news
% to the user. Normally in Python the traceback is good enough for this,
% but in this case, we start with a |.sage| file (which is
% autogenerated) which itself autogenerates a |.py| file---and the
% tracebacks the user sees refer to that file, whose line numbers are
% basically useless. We want to tell them where in the \LTX file things
% went bad, so we do that, give them the traceback, and exit after
% removing the |.sout.tmp| file.
%    \begin{macrocode}
  def goboom(self, line):
    print('\n**** Error in Sage code on line %s of %s.tex! Traceback\
 follows.' % (line, self.filename))
    traceback.print_exc()
    print('\n**** Running Sage on %s.sage failed! Fix %s.tex and try\
 again.' % ((self.filename,) * 2))
    self.souttmp.close()
    os.remove(self.filename + '.sout.tmp')
    sys.exit(int(1))
%    \end{macrocode}
% We use |int(1)| above to make sure |sys.exit| sees a Python integer;
% see
% \href{http://trac.sagemath.org/sage_trac/ticket/2861#comment:5}{ticket
% \#2861}.
% \changes{v2.0.2}{2008/04/21}{Make sure sys.exit sees a Python integer.}
% \end{macro}
% 
% \begin{macro}{endofdoc}
% When we're done processing, we have some cleanup tasks. We
% want to put the MD5 sum of the |.sage| file that produced the |.sout|
% file we're about to write into the |.sout| file, so that external
% programs that build \LTX documents can determine if they need to call Sage
% to update the |.sout| file. But there is a problem: we write line
% numbers to the |.sage| file so that we can provide useful error
% messages---but that means that adding non-\ST text to your
% source file will change the MD5 sum, and your program will think it
% needs to rerun Sage even though none of the actual \ST macros
% changed.
%
% How do we include line numbers for our error messages but still allow
% a program to discover a ``genuine'' change to the |.sage| file? 
%
% The answer is to only find the MD5 sum of \emph{part} of the |.sage|
% file. By design, the source file line numbers only appear in calls to
% |goboom|, so we will strip those lines out. Basically we are doing
% \begin{center}
% \verb+grep -v '^ _st_.goboom' filename.sage | md5sum+
% \end{center}
% (In fact, what we do below produces exactly the same sum.) 
%    \begin{macrocode}
  def endofdoc(self):
    sagef = open(self.filename + '.sage', 'r')
    m = hashlib.md5()
    for line in sagef:
      if line[0:12] != ' _st_.goboom':
        m.update(line)
    s = '%' + m.hexdigest() + '% md5sum of corresponding .sage file\
 (minus "goboom" lines)\n'
    self.souttmp.write(s)
%    \end{macrocode}
% Now, we do issue warnings to run Sage on the |.sage| file and an
% external program might look for those to detect the need to rerun
% Sage, but those warnings do not quite capture all situations. (If
% you've already produced the |.sout| file and change a |\sage| call, no
% warning will be issued since all the |\ref|s find a |\newlabel|.)
% Anyway, I think it's easier to grab an MD5 sum out of the end of the
% file than parse the output from running |latex| on your file. (The
% regular expression |^%[0-9a-f]{32}%| will find the MD5 sum. Note that
% there are percent signs on each side of the hex string.)
%
% Now we are done with the |.sout.tmp| file. Close it, rename it, and
% tell the user we're done.
%    \begin{macrocode}
    self.souttmp.close()
    os.rename(self.filename + '.sout.tmp', self.filename + '.sout')
    self.progress('Sage processing complete. Run LaTeX on %s.tex again.' %\
             self.filename)
%    \end{macrocode}
% \end{macro}

% \endinput
%</python>
% Local Variables: 
% mode: doctex
% TeX-master: "sagetexpackage"
% End: 
